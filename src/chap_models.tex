\chapter{Models}
\label{ch:mod}

Having obtained spectroscopic observations of the sample stars, we now
would like to extract information from those observations about the
physical properties of these stars. We do this by comparing the
observations with published theoretical models. The complete theory of
stellar atmospheres is far beyond the scope of this work, but some of
the more important aspects of the models are presented below.

%%%%%%%%%%%%%%%%%%%%%%%%%%%%%%%%%%%%%%%%%%%%%%%
%%%%%%% SECTION = STELLAR ATMOSPHERES %%%%%%%%%
%%%%%%%%%%%%%%%%%%%%%%%%%%%%%%%%%%%%%%%%%%%%%%%
\section{Modelling stellar atmospheres}

There is today a vast variety of different computer codes available
for the modelling of stellar atmospheres (for a good, but now slightly
outdated, summary see \cite{hu:91}). Although these codes sometimes
vary substantially in their approach and/or emphasis they all more or
less trace their ancestry to the work of Uns\"old in the 1930s.

%%%%%%%%%%%%%%%%%%%%%%%%%%%%%%
\subsection{Basic assumptions}
The `classical stellar atmosphere theory' is based on several
assumptions, which are still commonly made in modern atmospheric
models:

\begin{itemize}

\item	\textbf{Plane-parallel geometry} assumes that the stellar
	atmosphere can be divided into plane-parallel layers (see
	figure \ref{fig:planepar}). This reduces the geometrical
	dimensions from 3 to 1 and simplifies numerical
	calculations. Such an assumption is only valid if the
	thickness of photosphere constitutes a small fraction of the
	stellar radius. For stars with extended atmospheres, the
	geometry is likely to play an important role, and more
	rigorous spherical models are preferable.

\begin{figure} %%%%%%%%%FIGURE
\epsfxsize=11cm
\setlength{\cen}{(\textwidth / 2) - (\epsfxsize / 2)}
\hspace{\cen}\epsfbox{fig_models/planepar.eps}
%\fnote{Scanned schematic of plane-parallel atmospheres from Kudritzki}
\caption[The geometry of plane-parallel and extended atmospheres]
{\fcfont The geometry of plane-parallel and extended atmospheres. From
Kudritzki, 1988.
}
\label{fig:planepar}
\end{figure}   %%%%%%%%%%%%%%%

\item 	\textbf{Steady State}, i.e. no time-dependent phenomena (like
	shocks or pulsations). 

\item	\textbf{Hydrostatic equilibrium} implies that gravitational
	forces are exactly balanced by gas pressure. There is no net
	flow of mass.

\item	\textbf{Radiative equilibrium} assumes that radiation is the
	only (or at least the dominant) carrier of energy through the
	atmosphere. Other processes which transport energy, in particular
	convection, are neglected.

\item	\textbf{Chemical homogeneity}. It is generally assumed that
	the chemical composition is constant throughout the stellar
	atmosphere. 

\end{itemize}

%%%%%%%%%%%%%%%%%%%%%%%%%%%%%%%
\subsection{Radiative transfer}

The general aim of a model atmosphere is to calculate the emergent
flux at the surface of a star. The interaction of radiation with the
stellar material can simply be considered as the sum of the losses and
gains in intensity when the radiation has travelled a length
$\mathrm{d}s$
%
\begin{equation}
\mathrm{d}I_\nu = - \kappa_\nu \rho I_\nu \mathrm{d}s
		  + j_\nu \rho \mathrm{d}s
\end{equation}
%
where $j_\nu$ and $k_\nu$ are the mass coefficients for emissivity and
absorption respectively. We now define the \emph{optical depth}
which is simply the amount of absorbing matter the radiation
encounters over a length $L$
%
\begin{equation}
\tau_\nu = \int_0^L\kappa_\nu \rho \mathrm{d}s
\end{equation}
%
and the \emph{source function} which describes the ratio of the
emissivity and absorption coefficients
%
\begin{equation}
S_\nu(\tau_\nu) = \frac{j_\nu}{k_\nu}\mathrm{.}
\end{equation}
%
By combining these three equations we get
%
\begin{equation}
\mathrm{d}I_\nu \mathrm{d}\tau_\nu = -I_\nu + S_\nu
\end{equation}
%
which is called the \emph{equation of radiative transfer} and
describes the change in intensity when radiation travels along an
infinitesimal length $\mathrm{d}\tau_\nu$. The formal solution of
this equation can be written as
%
\begin{equation}
F_\nu = 2\pi \int_0^\infty S_\nu(\tau_\nu)E_2(\tau_\nu)d\tau_\nu
\end{equation}
%
where $F_\nu$ is the total emergent flux at frequency $\nu$ and
$E_2(x)$ is the second exponential integral given by
%
\begin{equation}
E_2(x) = \int_1^\infty \frac{e^{-xt}}{t^2}dt\mathrm{.}
\end{equation}
%
The mass emissivity and absorption coefficients depend on details of
the interactions of photons with the stellar material, and can only be
found if we know the physical properties of this material. This is the
main difficulty in modelling a stellar photosphere: the radiation
field depends on the properties of the matter, but the state of the
material itself depends on the radiation field which, through
interaction with photons, heats it and determines its excitation and
ionisation characteristics.


%%%%%%%%%%%%%%%%%%%%%%%%%%%%%%%%%%%%%%%%%%%%
\subsection{Local Thermodynamic Equilibrium}

The properties of a gas are most easily found if the gas is in
\emph{thermodynamic equilibrium} (TE). In TE, the state of the gas (the
distribution of atoms over bound and free states) is described by the
equations of statistical mechanics, and can be uniquely defined by two
thermodynamic variables. In the physics of stellar atmospheres it is
customary to use the temperature, $T$, and the particle number density, $n$.

However, it is clear that a stellar atmosphere is far from
Thermodynamic Equilibrium. New energy is constantly being created in
the core and `pumped' into the atmosphere, and at the same time
photons are escaping from the surface (otherwise we wouldn't be able
to observe the star at all). Instead, since TE does not hold, we make
the assumption of \emph{Local Thermodynamic Equilibrium}, or LTE,
which means that state of the gas is at its equilibrium value as
determined by the \emph{local} values of $T(r)$ and $n(r)$.

More specifically, this means that for the local values of $T$ and $n$,

\renewcommand{\theenumi}{\roman{enumi}}
\renewcommand{\labelenumi}{(\theenumi)}
\begin{enumerate}
\item	the particle velocity distributions are Maxwellian
	\begin{equation}
	f(v)\mathrm{d}v = \biggl( \frac{m}{2\pi kT}\biggr)^{3/2}
		\exp\biggl(\frac{-mv^2}{2kT}\biggr)\mathrm{d}v
	\end{equation}
	where $m$ and $v$ are the particle mass and velocity and $k$
	is the Boltzmann constant.

\item	the excitation equilibrium is given by the Boltzmann
	distribution
	\begin{equation}
	\frac{n_j}{n_i} = \frac{g_j}{g_i}
	\exp\biggl[\frac{-(E_j - E_i)}{kT}\biggr]
	\end{equation}
	where $g_{i}$ is the statistical weight assigned to level
	$i$ to account for degenerate sublevels, and $E_{i}$ is the
	level energy (measured from the ground state).

\item	the ionisation equilibrium is given by Saha's equation
	\begin{equation}
	\frac{N_j}{N_{j+1}} = n_e\frac{U_j}{U_{j+1}}
		\biggl(\frac{h^2}{2\pi mkT}\biggr)^{3/2}
		\exp \biggl(\frac{\chi_{Ij}}{kT}\biggr)
	\end{equation}
	where $N_j$ is the total number density, and $\chi_{Ij}$ is the ionisation potential,
	of ionisation stage $j$. $U$ is the partition function, given by
	$U_j = \sum_1^\infty g_j \exp (-\chi_j/kT)$.

\item	the source function, $S_\nu$, is given by the Planck
	function, $B_\nu$.
\end{enumerate}
\renewcommand{\theenumi}{\arabic{enumi}}
\renewcommand{\labelenumi}{\theenumi.}

%%%%%%%%%%%%%%%%%%%%%%%%%%%%%%%%%%%%%%%%
\subsection{Departures from simple model}

With all the assumptions discussed in the previous sections is
relatively easy to compute the properties of a `simple' stellar
atmosphere, which in many cases will be fairly successful in
describing the gross properties of the observed stellar spectrum, like
the visual continuum. However, there are also many cases where these
assumptions fail completely.

In the late 1960s and early 1970s, important work was done on
developing computational procedures that relieved the need for
LTE. In this approach, usually referred to as non-LTE or nLTE, the
populations of atomic and ionic states are allowed to deviate from
their LTE values, yielding (one hopes) a more realistic description of
the atmosphere. However, non-LTE routines are both relatively complex and
computationally expensive, and often only a few selected ionic
levels are allowed to depart from LTE.

A lot of work in the 1990s has been on the creation of `unified' model
atmospheres which simultaneously and consistently treat the
photosphere and the stellar wind. Classically, the outer boundary of
the photosphere was considered to be in hydrostatic equilibrium
(something that clearly is not the case, especially if the stars are
experiencing mass loss). The emergent flux from the static-atmosphere
model was then imposed as the lower boundary of the expanding envelope
--- a ``core--halo'' model. In unified atmospheres this artificial
segregation is removed and the photosphere and expanding envelope are
treated self-consistently, taking into account opacity effects in the
envelope.

Also several of the `new-generation' models remove the assumption of
plane-parallel atmospheres, and include the 3-dimensional angle
dependence in the radiation-transfer equations.

However, perhaps the most important of all has been the
increased quality (and quantity) of atomic and molecular
data. Although it might appear from the visible spectrum (see
appendix \ref{app:10lac}) that hydrogen and helium are the dominant
sources of opacity in early-type stars, UV spectra of these same stars
show a wealth of lines of heavier atomic species. The combined
opacity from these sources can significantly alter the temperature
structure in the upper layers of the stellar photosphere (e.g. \cite{hu:98}).

A graphical summary of the physics of O-star atmospheres is shown in
figure \ref{fig:atmophys}.

\begin{figure} %%%%%%%%%FIGURE
\epsfxsize=11cm
\setlength{\cen}{(\textwidth / 2) - (\epsfxsize / 2)}
\hspace{\cen}\epsfbox{fig_models/atmophys.eps}
%\fnote{Scanned schematic of atmospheric physics from Kudritzki}
\caption[Physics of O-stars atmospheres]
{\fcfont The physics of hot star atmospheres. From
Kudritzki, 1988.
}
\label{fig:atmophys}
\end{figure}   %%%%%%%%%%%%%%%


%%%%%%%%%%%%%%%%%%%%%%%%%%%%%%%%%%%%%%%%%%%%%%%
%%%%%%%%%%% SECTION = Model Grid %%%%%%%%%%%%%%
%%%%%%%%%%%%%%%%%%%%%%%%%%%%%%%%%%%%%%%%%%%%%%%
\section{Model grid}

The grid of model spectra used in this work for the analyses of O
stars was primarily calculated by K. C. Smith (unpublished) at
UCL. Details of this grid can be found in \ybcite{sm:94}, and will
only be summarised here. The grid consists of over 2200 individual
model spectra calculated for several different values of four
independent variables, namely effective temperature, surface gravity,
helium number fraction and microturbulence. It spans a range in
effective temperatures, \teff, from 27~kK to 3.5~kK in steps of 1~kK,
and in surface gravity, \logg, from close to the effective Eddington
limit%
\footnote{The lowest gravity models are 2.7 dex and 3.3 dex for
effective temperatures of 27~kK and 40~kK respectively.}
%
to 3.7 dex (c.g.s.) at intervals of 0.1 dex, with additional
`main-sequence' models calculated for 4.0 and 4.5 dex at each
temperature. Each of these \teff--\logg\ grids is repeated for
helium number fractions of 0.05, 0.09 (solar), 0.20 and 0.30, and
for microturbulent velocities of 0, 5, 10 and 15~\kms.

\subsection{Atmospheric structure calculations}

The model atmospheres were calculated using \tlusty\ version 160
(\cite{tlusty1}; \cite{tlusty2}), assuming plane-parallel geometry and
hydrostatic and radiative equilibria. Only hydrogen and helium were
considered explicitly in the structure calculations, with 8 non-LTE
levels each of \hi\ and \heii\, and 14 non-LTE levels of \hei. In
addition to these, series of LTE hydrogenic levels were considered,
extending to principal quantum numbers, $n$~=~16, 16, and 32 for \hi,
\hei, and
\heii\ respectively.

\subsection{Synthetic spectra}

Using the atmospheric structures discussed above, line-formation
calculations were carried out using the codes \detail\ and
\surface\ (\cite{gi:81}; \cite{bu:84}; \cite{bu:85}).

\detail\ solves the coupled equations of statistical equilibrium and
radiative transfer, using the fixed atmospheric structure
calculated above. It uses the method of complete linearization as
formulated by \ybcite{au:74}. The advantage with this formulation is
that the computational timing is linearly proportional to the number
of frequency points. This, coupled with the fact that good agreement
between model and observed line profiles is very dependent on the
complexity of the model atoms used in this step (e.g. Herrero
\cite*{he:87hyd} and \cite*{he:87he}), meant that line-formation
calculations were carried out using an increased number of non-LTE
levels for all ions (see \cite{sm:94}, for specific details about the
model atoms used).

Finally, synthetic H/He spectra were computed using \surface\ for each
of the values of \teff, \logg, $y$ and microturbulence discussed
above. These grids of spectra were then loaded into an analysis package,
\prog{tgkiel}, written by K. C. Smith, that for a given value of
microturbulence will interpolate the grid of model spectra in all 3
dimensions (i.e. \teff,
\logg\ and $y$). These interpolated models were then used for the
analyses.

%%%%%%%%%%%%%%%%%%%%%%%%%%%%%%%%%%%%%%%%%
\subsection{Model predictions}

In subsequent chapters we compare these models to observed spectra; in
particular, we will concentrate on the equivalent widths of certain
\hei\ and \heii\ lines, and the profiles of \hi\
lines. Before attempting such analyses, it is useful to investigate the
theoretical behaviours of the lines in question. Figures
\ref{fig:cont:he1} and \ref{fig:cont:he2} show the equivalent width
behaviour predicted by the models for the helium lines considered in
the analyses. It is worth keeping in mind when selecting lines for
\teff--\logg\ fit-diagram techniques (as explained in chapter
\ref{ch:casestudy}), lines which show a low equivalent width
sensitivity to the model input parameters (i.e., \teff\ and \logg)
might `amplify' uncertainties in measured equivalent widths when used
in the analyses. This is discussed further in chapter
\ref{ch:casestudy}. Similarly, lines which have a high sensitivity
might emphasise missing physics or assumptions made in the
models. This is clearly seen for the \hei\ triplet lines, in
particular for \hei\ \lam4471\AA, where the models consistently fail
to reproduce observed equivalent widths (see also chapter \ref{ch:analyses}).

Although it has previously been demonstrated (e.g. \cite{sm:94}) that
these models are successful in reproducing, at least to an acceptable
level, certain H and He lines in spectra of hot stars, they still do
fail, sometimes catastrophically, in modelling others. In particular,
these hydrostatic models neglect the effects of the stellar wind
envelope surrounding the stars, and are unable to reproduce the
wind-induced emission features or distortions often observed in lines
such as the lower members of the hydrogen Balmer series (for example
\ha\ and \hb) and \heii~\lam4686\AA.

\begin{figure} %%%%%%%%%FIGURE
\epsfxsize=\figwidth
\setlength{\cen}{(\textwidth / 2) - (\epsfxsize / 2)}
\hspace{\cen}\epsfbox{fig_models/contours/he1_cont.eps}
\caption[Equivalent-width contours for \hei]
{\fcfont Equivalent-width contours for \hei\ lines for models with
solar helium abundance ($y = 0.09$). Contours are drawn every 20m\AA,
with thick lines every 100m\AA.  
}
\label{fig:cont:he1}
\end{figure}   %%%%%%%%%%%%%%%

\begin{figure} %%%%%%%%%FIGURE
\epsfxsize=\figwidth
\setlength{\cen}{(\textwidth / 2) - (\epsfxsize / 2)}
\hspace{\cen}\epsfbox{fig_models/contours/he2_cont.eps}
\caption[Equivalent-width contours for \heii]
{\fcfont Equivalent-width contours for \heii\ lines. Contours as in
previous figure. Note `edge effects' caused by interpolation defects
close to edge of the grid (determined by the Eddington Limit). 
}
\label{fig:cont:he2}
\end{figure}   %%%%%%%%%%%%%%%


%%%%%%%%%%%%%%%%%%%%%%%%%%%%%%%%%%%%%%%%%%%
\subsection{Effects of microturbulence}

Microturbulence was originally incorporated into curve-of-growth
analyses to explain the observed fact that equivalent widths of
certain lines are larger than those predicted by
models\footnote{Models here does not necessarily refer to computer
models, but to any predicted behaviour of the curve-of-growth} not
taking this parameter into account.

Lately there has been a lot of interest in microturbulence in O and
B-type stars. \ybcite{mcer:98} and \ybcite{sm:98} have both performed
in-depth analyses of the effects of microturbulence when comparing
models to observations. Both groups have found that including
microturbulence in the line formation calculations results in an
increaase in the model
\hei-line equivalent widths. This is an unavoidable consequence of
desaturation if only line formation is considered, but \acite{sm:98}
also showed that this is not true for all lines if microturbulence is
treated in both line formation \emph{and} statistical equilibrium
calculations. However, it still holds for some important lines
(e.g. \hei~\lam4921\AA\ and \lam4471\AA) commonly used in the analysis
of O stars.  This increase in equivalent width can lead to a
lower adopted helium abundance than if microturbulence is neglected,
something that could, at least partly, help resolve the so-called
`helium discrepancy' identified in OB-type stars by
\ybcite{he:92}. It also appears that the inclusion of microturbulence
has a greater effect on the \hei\ triplet lines than on the
singlet lines. This could help explain the `generalized dilution
effect' described by \ybcite{vo:89}. There is a more detailed
discussion of this in chapters \ref{ch:casestudy} and
\ref{ch:analyses}.

In the model grid used in this thesis%
\footnote{Work is underway to construct a grid which treats
microturbulence in both line-formation and statistical-equilibrium
calculations.}%
, microturbulence was only
included in the line-formation step of the modelling, and its effect
on the statistical-equilibrium was neglected. Microturbulence was
incorporated into the Voigt functions by including a non-thermal term
thus:
\begin{equation}
\Delta \lambda_\mathrm{D} = (\lambda/c)\sqrt{(2kT/Am_\mathrm{H}) +
v_\mathrm{turb}^2}
\end{equation}
where $c$ is the speed of light, $k$ is the Planks constant, $T$ is the
gas temperature, $A$ is the atomic mass (of He in this case),
$m_\mathrm{H}$ is the mass of an H atom, and $v_\mathrm{turb}$ is the
microturbulent velocity.

The effects of microturbulence for \hei\ lines can be seen in figure
\ref{fig:mturb:he1}. Including 15~\kms\ microturbulence can add almost
a third to the equivalent width of certain \hei\ lines. It is also
appears that microturbulence is slightly more important in
supergiants, with a low surface gravity, than it is for main-sequence
stars, with $\logg \sim 4.5$. The \heii\ lines are relatively
unaffected by the inclusion of microturbulence in the line formation
calculations only (maximum changes are only of the order of a few per
cent).

\begin{figure} %%%%%%%%%FIGURE
\epsfxsize=\figwidth
\setlength{\cen}{(\textwidth / 2) - (\epsfxsize / 2)}
\hspace{\cen}\epsfbox{fig_models/contours/he1_mturb.eps}
\caption[Microturbulence dependence for \hei]
{\fcfont Equivalent-width contours for \hei\ lines highlighting the
effects of the microturbulence parameter. Contours are drawn every 100m\AA.
Thick lines are for a microturbulent velocity of 15~\kms, and thin
lines for zero microturbulence.
}
\label{fig:mturb:he1}
\end{figure}   %%%%%%%%%%%%%%%

%\begin{figure} %%%%%%%%%FIGURE
%\epsfxsize=\figwidth
%\setlength{\cen}{(\textwidth / 2) - (\epsfxsize / 2)}
%\hspace{\cen}\epsfbox{fig_models/contours/he2_mturb.eps}
%\caption[Microturbulence dependence for \hei]
%{\fcfont Equivalent width contours for \heii\ lines highlighting the
%effects (i.e. none) of the microturbulence parameter. Contours as in previous
%figure.  
%}
%\label{fig:mturb:he2}
%\end{figure}   %%%%%%%%%%%%%%%

%%%%%%%%%%%%%%%%%%%%%%%%%%%%%%%%%%%%%%%%%%%%%%%%%%
\subsection{Rotational or macroturbulent broadening}

A subject which we will repeatedly come back to throughout this work
is the question of whether the dominant `macroscopic' broadening process
in O-type stars is rotational Doppler broadening, or some other
`macroturbulent' velocity field. We distinguish between `microscopic'
processes which act on, essentially, atomic scales (Stark broadening,
thermal broadening, etc.) or which otherwise affect the intrinsic line
profile (like microturbulence), and `macroscopic' processes which do
not affect the line strength, but only the observed line profile (like
rotation and macroturbulence).

To try to resolve this problem, we will in chapters \ref{ch:casestudy}
and \ref{ch:analyses} compare observations to model spectra convolved
with either a rotational Doppler function (\cite{gray} p. 374):
%
\begin{equation}
\mathrm{f}(\Delta x,\Delta\lambda,\epsilon) =
\frac{2(1-\epsilon)\sqrt{1-\Delta x^2} + \frac{1}{2} \pi \epsilon (1
-\Delta x^2)}{\pi \Delta \lambda (1 - \epsilon/3)} 
\label{eq:mod:rotation}
\end{equation}
%
or a Gaussian function which simulates a macroturbulent velocity field
in the stellar atmosphere:
%
\begin{equation}
\mathrm{f}(\Delta x,\Delta\lambda) =
\frac{\sqrt{\pi}}{\Delta\lambda \times \exp(-\Delta x^2)}
\end{equation}
%
where $\epsilon$ is the limb-darkening coefficient, $\Delta\lambda$ is
the apparent velocity shift. The value $x$ is the x-coordinate of an
imaginary strip (see figure \ref{fig:rot:gray}) of the stellar disk,
and, for rotation, is related to $\Delta\lambda$ through:
%
\begin{equation}
\Delta\lambda = v = x\Omega\sin i
\label{eq:dopshift}
\end{equation}
%
where $\Omega$ is the angular velocity and $i$ is the angle
of inclination as illustrated in figure \ref{fig:rot:gray}.

\begin{figure} %%%%%%%%%FIGURE
\epsfxsize=12cm
\setlength{\cen}{(\textwidth / 2) - (\epsfxsize / 2)}
\hspace{\cen}\epsfbox{fig_models/gray.eps}
\caption[Schematic of a rotating star]
{\fcfont Schematic of a rotating star. \textbf{Left diagram} illustrates
symbols discussed in text. The $y$ axis is chosen to make $\Omega$ lie
in the $y$ -- $z$ plane. \textbf{Right diagram} shows how the
apparent disk of a star can be thought of as a series of strips, each
with a Doppler shift according to equation \ref{eq:dopshift}. From Gray (1992).}

\label{fig:rot:gray}
\end{figure}   %%%%%%%%%%%%%%%

Figure \ref{fig:brfunc:comparison} shows the model \hei~\lam4143\AA\
line convolved with the rotational function (solid line) and the
macroturbulent function (dashed line). The rotational convolution gives
the line a much deeper rounder shape, whereas the macroturbulent
convolution gives a sharper and broader profile, almost a `v'-shape.

In an attempt at determing which of these line broadening functions
best reproduce the observed line profiles, model spectra convolved
with the above functions are compared to observations using a $\chi^2$
fitting procedure. This analysis is described in section
\ref{sec:casestudy:broadening}.

\begin{figure} %%%%%%%%%FIGURE
\epsfxsize=\figwidth
\setlength{\cen}{(\textwidth / 2) - (\epsfxsize / 2)}
\hspace{\cen}\epsfbox{fig_models/broadening/comparison.eps}
\caption[Comparison of rotational Doppler and macroturbulent broadening 
functions]
{\fcfont Comparison of rotational Doppler and macroturbulent broadening
functions. Solid lines show the rotationally broadened line profiles,
and the dashed lines are the line profile broadened by macroturbulence.}

\label{fig:brfunc:comparison}
\end{figure}   %%%%%%%%%%%%%%%

