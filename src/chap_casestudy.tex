\chapter{10 Lac --- a case study}
\label{ch:casestudy}


10 Lacertae (HD 214680) is one of a very few narrow-lined O-stars. Its
projected equatorial rotation velocity is only of the order 25--35
\kms\ (35 \kms\ in 
\cite{ho:xcorr}; 31~\kms\ in \cite{pe:xcorr}; see also table 
\ref{ta:10Lac_results}). This makes 10 Lac a good star to use as
a case study. The narrow lines make line identification relatively
easy, and many lines that are either badly blended or too broadened to
be reliably measured for most of the other O stars in the sample are easily
measured. This chapter will primarily look at the method used to find
the stellar parameters for the O stars in this thesis, and demonstrate
its use by applying it to 10 Lac. It is important that, although the
strength of this analysis is in its consistency, the method should
yield good results for individual stars. However, the results of a single
star should always be considered in the context of to the complete sample.

%%%%%%%%%%%%%%%%%%%%%%%%%%%%%%%%%%%%%%%%%%%%%%%%%%%%%%%%
%%%%%%%%%%%%%%% SECTION = Spec. Features %%%%%%%%%%%%%%%
%%%%%%%%%%%%%%%%%%%%%%%%%%%%%%%%%%%%%%%%%%%%%%%%%%%%%%%%
\section{Spectral features}

The 10~Lac spectrum is a good example of a typical late O-type
main-sequence spectrum\footnote{We are here only considering the
optical spectrum, but UV spectra of 10~Lac has also been studied in
detail by several authors (e.g., \cite{wa:96}).}. Figure
\ref{fig:10Lac:spec} shows some of the spectral features considered in
this thesis. The optical spectrum is dominated by strong, broad
hydrogen lines of the Balmer series and by the relatively strong \hei\
triplet lines (e.g. \lam\lam4026 and 4471\AA). The \heii\ lines are
relatively weak compared to the H and \hei\ triplet lines, although
these lines are considerably stronger in stars with higher effective
temperatures, \teff (i.e. earlier spectral types).

Also visible are many weaker lines, in particular lines from several
ions of C, N, O and Si. Only some of these are shown in the figure.
See appendix~\ref{app:10lac} for more complete diagrams. 

The 4630--4640--4650\AA\ region contains a wealth of C, N and O lines, and
the strong lines here are often used to check for carbon and nitrogen
abundance anomalies in these stars (see also appendix \ref{app:ocn}).

\begin{figure} %%%%%%%%%FIGURE
\epsfxsize=15cm
\setlength{\cen}{(\textwidth / 2) - (\epsfxsize / 2)}
\hspace{\cen}\epsfbox{fig_casestudy/10lac_lines.eps}
%\fnote{Include key regions of the 10~Lac spec. here.
%Could do 100\AA\ per strip, 3780 -- 3880, 3880 -- 3980}  
\caption[Selected wavelength regions from the spectrum of 10~Lac]
{\fcfont Selected wavelength regions from the spectrum of 10~Lac with
some important lines labelled. Several of the lines used in the
analyses are shown, as well as the 4630 -- 4640 -- 4650\AA\ CNO region
(only C and N lines are identified here). Interstellar lines are
labelled in \textit{italics}. The spectrum of 10~Lac is shown in its
entirety in appendix \ref{app:10lac}.}

\label{fig:10Lac:spec}
\end{figure}   %%%%%%%%%%%%%%%

%%%%%%%%%%%%%%%%%%%%%%%%%%%%%%%%%%%%%%%%%%%%%%%%%%%%%%%%
%%%%%%%%%%%%%%% SECTION = Stellar parameters %%%%%%%%%%%
%%%%%%%%%%%%%%%%%%%%%%%%%%%%%%%%%%%%%%%%%%%%%%%%%%%%%%%%
\section{Determining stellar parameters}

As a general aim we want to find the effective temperature, \teff, the
surface gravity, \logg, and the helium number fraction, $y$. These
parameters define the most important properties of the star, and are
intimately connected to other stellar parameters like the stellar mass
and age. (There is also a fourth parameter which turns out to have an
important effect on the final results: the microturbulence.) Using the
grid of pre-calculated models discussed in chapter
\ref{ch:mod} we want to find a method to estimate these parameters by
comparing the models to the observations.

In practice, we need to optimize pairs of parameters in tandem, since
rigorous multi-dimensional optimisation is impractical (the models do
not perfectly match the observations). After some initial testing, the
method that was found to yield the most consistent results for the
lines available was the construction of temperature -- gravity
diagrams; 2-D slices through the 3-D parameter space (4-D with
microturbulence) being explored. The measured equivalent widths of the
helium lines are traced in the diagrams as the locii of temperatures
and gravities of the models that reproduce the measured values. The
\heii\ lines are, in the parameter space explored, mostly sensitive to
the effective temperature, whereas the \hei\ lines are sensitive to
both \teff\ and \logg.

The plots give an intersection zone where most of the lines cross, and the
best-fit model is likely to fall somewhere inside this intersection
zone. Due to observational errors and noise, as well as inadequacies in the
models, there will always be a certain scatter of the
lines. The wrong choice of the microturbulence parameter or of the
helium abundance will also affect the plotted lines, and the aim of the
analysis is to find the values of the free parameters that
minimise the size of the intersection zone.

To constrain the solution further, a $\chi^2$ fitting technique is
applied to the wings of H10 \lam3797\AA, H9 \lam3835\AA, H8 \lam3889\AA\ and
\he\ \lam3970\AA. Unfortunately, the strong, wide hydrogen lines are often
contaminated by blends with other lines and by wind emission; however,
the pressure broadened wings are very sensitive to the surface
gravity. By fitting the wings (and removing blends where they occured)
consistent results could be obtained, even for the stars that show
strong wind contamination.

The steps above are then repeated for a range of values of $y$
(usually between 0.09 and 0.18 in steps of 0.01) and microturbulence
(for 0, 5, 10 and 15 \kms). The `best' solution is determined by using
an impact-parameter analysis to find the point on each diagram that
minimises the sum of the distances to individual lines. The best-fit
diagram in all 4 dimensions is the one with the overall smallest value
of the summed impact parameter. Figure \ref{fig:ewsummary}
demonstrates the method, and each step is described in more detail in
the following sections.

\begin{figure} %%%%%%%%%%%%%%%%%%% FIGURE
\begin{center}
\input{fig_casestudy/flowchart/flow.pic}
\end{center}
\caption[Method used to determine stellar parameters]
{\fcfont Summary of method used to determine stellar parameters}
\label{fig:ewsummary}
\end{figure} %%%%%%%%%%%%%%%%%%% END FIGURE

%%%%%%%%%%%%%%%%%%%%%%%%%%%%%%%%%%%%%%%%%%%%%%%%%%%%%%%%
%%%%%%%%%%%%%%% SECTION = EWs of He %%%%%%%%%%%%%%%%%%%%
%%%%%%%%%%%%%%%%%%%%%%%%%%%%%%%%%%%%%%%%%%%%%%%%%%%%%%%%
\section{Equivalent widths of helium lines}
\label{sec:ews}

There are a large number of helium lines in the observed spectrum.
However, a number of these are either too badly blended or too weak to be
measured reliably for all the stars in this dataset. The possible lines
available for analysis are shown and commented on in tables
\ref{ta:he1_lines} and \ref{ta:he2_lines}. Some of the possible \hei\ lines are very
insensitive to changes in the effective temperature and/or the surface
gravity. Including these lines in the analysis can degrade the quality
of the results rather than improve them, since even a small error in
measuring the equivalent width can give rise to large errors in both
\teff\ and \logg. The line that showed the least sensitivity to these parameters,
\hei~$\lambda$5016\AA, has been left out of the analysis. However, 
\hei~$\lambda$4713\AA, which is also fairly insensitive (see figure
\ref{fig:cont:he1}) has been included to see what effects 
this line might have on the analysis (see also fig
\ref{fig:ew_errors:he1}). In cases where this line gives discrepant
results it has been left out of the impact-parameter calculations.

\begin{table} %%%%%%%%%%%%%% HeI LINES %%%%%%%%%%%%%%%%%%%%%%%%%%55
\begin{center}
\caption{Observed \hei\ lines}
\begin{tabular}{rr@{ -- }lp{10cm}}
\hline\hline
\multicolumn{1}{c}{$\lambda$ (\AA)} & \multicolumn{2}{c}{Transition} &
Comment \\ 
\hline
3819.6 		& $2p~^3P^\circ$ & $6d~^3D$	& \\
3867.5 		& $2p~^3P^\circ$ & $6s~^3S$	& Blended with several weak lines\\
3964.7 		& $2s~^1S$ & $4p~^1P^\circ$	& Blended with \hi \\
\textbf{4009.3} & $2p~^1P^\circ$ & $7d~^1D$	& \\
\textbf{4026.2} & $2p~^3P^\circ$ & $5d~^3D$	& \\
4120.8 		& $2p~^3P^\circ$ & $5s~^3S$ 	& Blended with several strong lines\\
\textbf{4143.8} & $2p~^1P^\circ$ & $6d~^1D$	& \\
4169.0 		& $2p~^1P^\circ$ & $6s~^1S$	& Very weak and possibly blended\\
\textbf{4387.9} & $2p~^1P^\circ$ & $5d~^1D$	& \\
4437.5 		& $2p~^1P^\circ$ & $5s~^1S$	& Very weak and in the middle of diffuse
     						interstellar band.\\
\textbf{4471.5} & $2p~^3P^\circ$ & $4d~^3D$	& \\
\textbf{4713.1} & $2p~^3P^\circ$ & $4s~^3S$	& Model predictions
     						insensitive to \teff\
     						and \logg\\
\textbf{4921.9}	& $2p~^1P^\circ$ & $4d~^1D$	& \\
5015.7 		& $2s~^1S$ & $3p~^1P^\circ$ 	& Model predictions insensitive to
     						\teff\ and
     						\logg. Excluded since
     						it might introduce
     						large errors into
     						analysis.\\  
\textbf{5047.7} & $2p~^1P^\circ$ & $4s~^1S$	& \\
5875.6 		& $2p~^3P^\circ$ & $3d~^3D$ 	& Very strong line expected to be
						extremely sensitive to
						the dilution effect
						(discussed in text).\\
6678.1 		& $2p~^1P^\circ$ & $3d~^1D$	& Sometimes lost in interorder gap\\

\hline\hline
\end{tabular}
\label{ta:he1_lines}
\end{center}
\tnote{The lines used in this analysis are shown in \textbf{bold} type.}
\tref{Line identifications and wavelengths from
\ybcite{scholtz:lines}; Transitions from \ybcite{striganov}.}
\end{table} %%%%%%%%%%%%%%%%%%%%%%%%%%%%%%%%%%%%%%%%%%%%%%

\begin{table} %%%%%%%%%%%%%% HeII LINES %%%%%%%%%%%%%%%%%%%%%%%%%%
\begin{center}
\caption{Observed \heii\ lines}
\begin{tabular}{rr@{ -- }lp{10cm}}
\hline\hline
\multicolumn{1}{c}{$\lambda$ (\AA)} & \multicolumn{2}{c}{Transition} &
Comment \\ 
\hline
3968.4 		& $4f~^2F^\circ$ & $14g~^2G$	& Very weak. Blended with \hi \\
4025.6 		& $4f~^2F^\circ$ & $13g~^2G$	& Very weak. Blended
						with \hei~\lam4026\AA \\
4100.0 		& $4f~^2F^\circ$ & $12g~^2G$	& Blended with \hi\\
\textbf{4199.8}	& $4f~^2F^\circ$ & $11g~^2G$	& \\
4338.7 		& $4f~^2F^\circ$ & $10g~^2G$	& Blended with \hi\\
\textbf{4541.6} & $4f~^2F^\circ$ & $9g~^2G$	& \\
\textbf{4685.7} & $3d~^2D$ & $4f~^2F^\circ$	& Sometimes contaminated by wind emission\\
4859.3 		& $4f~^2F^\circ$ & $8g~^2G$	& Blended with \hi\\
\textbf{5411.5}	& $4f~^2F^\circ$ & $7g~^2G$	& \\
6683.2 		& $5g~^2G$ & $13h~^2H^\circ$	& Very weak\\
\hline\hline
\end{tabular}
\label{ta:he2_lines}
\end{center}
\tnote{The lines used in this analysis are shown in \textbf{bold} type.}
\tref{Line identifications and wavelengths from
\ybcite{scholtz:lines}; Transitions from \ybcite{striganov}.}
\end{table} %%%%%%%%%%%%%%%%%%%%%%%%%%%%%%%%%%%%%%%%%%%%%%

The lines originating in the $2~^3S$, $2~^1S$, $2p~^3P^\circ$ and
$2p~^1P^\circ$ levels (in order of decreasing sensitivity), are very
sensitive to the so-called \emph{generalized dilution effect}
discussed in detail by \ybcite{vo:89}. For stars where the
plane-parallel assumption is no longer accurate (most likely all stars
not belonging to luminosity class V), the models fail to reproduce the
strengths of these lines. The strongest lines of each series are the
most sensitive to this effect, and \acite{vo:89} found that the weaker
lines (like \lam\lam4713 and 4388\AA) reproduced observations
well. \ybcite{sm:98} showed that including microturbulence in both
statistical equilibrium and line formation calculations increased the
strength of the lines assumed to be affected by the dilution effect
more than some of the weaker lines. They suggested that
microturbulence could possibly explain the dilution effect. We have
only included microturbulence in the line formation calculations and,
according to \acite{sm:98}, this might not be enough to bring
\hei~\lam4471\AA\ into agreement with the observations.  However, we
have chosen to include this line to evaluate the effects of
microturbulence.

The spectra were measured using the Starlink data analysis program
\dipso\ (\cite{dipso}). In \dipso\ there are two basic ways of measuring the
equivalent width of an absorption (or emission line). One is using the
command \comm{ew}. Using a user defined continuum the program
calculates the equivalent width between two interactively defined
points by simple trapezoidal integration. This method is quick and
works very well for strong, unblended lines with a well defined
continuum. However, where the lines are blended or the signal-to-noise
is low, fitting a gaussian (using the
\dipso\ subroutine \elf) and using the equivalent width of the fit
seems to give better and more consistent results. This works well for
stars where the rotational broadening is large enough to give the
lines a broad profile that can be approximated well by a
gaussian. However, for the stars with narrow profiles the gaussians
are poor approximations of the line profiles (figure
\ref{fig:elf_fits}).

\begin{figure} %%%%%%%%%%%%%%%%%%% FIGURE

\epsfxsize=\figwidth
\setlength{\cen}{(\textwidth / 2) - (\epsfxsize / 2)}
\hspace{\cen}\epsfbox{fig_casestudy/elf_fits/elf_fits.eps}

\caption[\elf\ fits to selected lines in HD~214680 and HD~195592]
{\fcfont \elf\ fits to selected lines in HD~214680 and HD~195592. The
gaussians do not fit the narrow line profiles of the slow rotator
(HD~214680) very well, but reasonable fits are obtained for the
slightly broader lines in HD~195592.}
\label{fig:elf_fits}
\end{figure} %%%%%%%%%%%%%%%%%%% END FIGURE


Table \ref{ta:elf_vs_ew} shows a comparison of equivalent widths
measured using \elf\ and \comm{ew} for 10 Lac and for
HD~195592. HD~195592 is the third slowest rotator in the dataset,
with a \vsini\ of approximately 50 -- 60 \kms\ (\cite{sl:56};
\cite{co:vii}; this thesis). In the case of the slower rotators it
appears that there there is a clear tendency for
the equivalent widths measured using \elf\ to be smaller than those
measured using
\comm{ew}. However, this trend
disappears for HD~195592 where already the gaussians are a fairly good
approximation of the line shape. It also appears that the typical
differences in equivalent widths are of the order 5 to 10 per cent. By
comparing this to figures
\ref{fig:ew_errors:he1} and \ref{fig:ew_errors:he2} it appears that
this corresponds to a difference of less than 1000K in temperature and
0.1 dex in \logg\ for most lines. In the interest of maintaining
consistency, the \elf\ equivalent widths are used for the analysis,
but the uncertainty on the final results for 10~Lac (and the slow
rotators in general) is likely to be slightly larger than for the fast
rotators.

\begin{table} %%%%%%%%%%%%%%% ELF VS EW %%%%%%%%%%%%%%%%%%%%
\begin{center}
\caption{Comparison of the \dipso\ commands \elf\ and \comm{ew}}
\small
\begin{tabular}{c|rrr@{}l}
\multicolumn{5}{c}{HD 214680} \\ \hline\hline
Line			& ELF (m\AA)	& EW (m\AA)	& 
\multicolumn{2}{c}{$\mathrm{\frac{EW - ELF}{ELF} \times 100}$}\\
\hline
\hei\ \lam 4009\AA 	& 224		& 200 		& $-10.7$ \\
\hei\ \lam 4026\AA 	& 791		& 851 		& 7.5 \\
\hei\ \lam 4143\AA 	& 247		& 290 		& 17.4 \\
\hei\ \lam 4388\AA 	& 399		& 421 		& 5.5 \\
\hei\ \lam 4471\AA 	& 646		& 715 		& 10.6 \\
\hei\ \lam 4713\AA 	& 265		& 266 		& 0.3 \\
\hei\ \lam 4922\AA 	& 466		& 510 		& 9.4 \\
\hei\ \lam 5047\AA 	& 117		& 121 		& 3.4 \\
\heii\ \lam 4200\AA 	& 632		& 650 		& 2.8 \\
\heii\ \lam 4541\AA 	& 562		& 680 		& 20.9 \\
\heii\ \lam 4686\AA 	& 797		& 850 		& 6.6 \\
\heii\ \lam 5411\AA 	& 764		& 780 		& 2.0 \\
\hline
Mean 			&		&		& 6.3 & ~$\pm$~7.8 \\
\hline\hline
\multicolumn{4}{c}{} \\
\multicolumn{5}{c}{HD 195592} \\ \hline\hline
Line			& ELF (m\AA)	& EW (m\AA)	&
\multicolumn{2}{c}{$\mathrm{\frac{EW - ELF}{ELF} \times 100}$}\\
\hline
\hei\ \lam 4009\AA 	& 215		& 196 		& $-8.8$ \\
\hei\ \lam 4026\AA 	& 641		& 650 		& 1.4 \\
\hei\ \lam 4143\AA 	& 207		& 210 		& 1.4 \\
\hei\ \lam 4388\AA 	& 386		& 378 		& $-2.0$ \\
\hei\ \lam 4471\AA 	& 811		& 828		& 2.0\\
\hei\ \lam 4713\AA 	& 309		& 319		& 3.2\\
\hei\ \lam 4922\AA 	& 500		& 504 		& 0.8 \\
\hei\ \lam 5047\AA 	& 145		& 135 		& $-6.8$ \\
\heii\ \lam 4200\AA 	& 202		& 193 		& $-4.4$ \\
\heii\ \lam 4541\AA 	& 264		& 249 		& $-5.6$ \\
\heii\ \lam 4686\AA 	& \multicolumn{4}{c}{\emph{contaminated by emission}}  \\
\heii\ \lam 5411\AA 	& 364		& 362 		& $-0.5$ \\
\hline
Mean			&		&		& $-1.8$ & ~$\pm$~3.9 \\
\hline\hline
\end{tabular}
\label{ta:elf_vs_ew}
\end{center}
\tnote{Comparison of equivalent widths for 10~Lac and
HD~195592. HD~195592 is the third slowest rotator in the sample, and
with a relatively slow \vsini\ of approximately 50~\kms. It is clear
that the agreement between \elf\ and \comm{ew} is better for the
faster rotator.}
\end{table}   %%%%%%%%%%%%%%%%%%%%%%%%%%%%%%%%%%%%%%%%%%%%%%


%%%%%%%%%%%%%%%%%%%%%%%%%%%%%%%%%%%%%%%%%%%%%%%%%%%%%%%%
%%%%%%%%%%%%%%% SECTION = VSINI %%%%%%%%%%%%%%%%%%%%%%%%
%%%%%%%%%%%%%%%%%%%%%%%%%%%%%%%%%%%%%%%%%%%%%%%%%%%%%%%%
\section{Rotational velocity and macroturbulence}
\label{sec:casestudy:broadening}

The $\chi^2$ fitting of the hydrogen line wings requires an estimate
of the `macroscopic' and `instrumental' broadening of the lines
(at least if either of the two are significantly large; see figure
\ref{fig:h_sensitivity}) As it turns out, in the case of the
broad hydrogen wings, the contribution due to the instrumental
broadening is negligible, and rotation/macroturbulence only have to be
considered if the velocity is sufficiently high (negligible for
velocities less than 100 -- 120~\kms, so only important for the
fastest rotators). Rotation is unimportant for the \hei\ and \heii\
line statistics, since we use measured equivalent widths for these.

One way of determining the projected rotational velocity, \vsini\, in
the stars in this sample, would be to measure the FWHM of the
absorption lines of metallic species like C, N, Si and O. These lines
are relatively unaffected by atmospheric broadening mechanisms
compared to the hydrogen and helium lines. These metallic lines are,
however, usually weak compared to H and He. Although they are easy to
measure in the slow rotators like 10 Lac, in the faster rotators the
lines are broadened too much to be accurately measured with our
signal-to-noise.

Since consistency is paramount in this part of the analysis, a
different approach to finding the \vsini\ is needed that will work for
stars with broad lines and for stars with narrow lines. A method that
seems to work well is to perform a quick equivalent width analysis
similar to that described in section \ref{sec:ew_analysis} but without
the hydrogen line analysis. The results of this quick analysis are
used to fix a starting point for the effective temperature and surface
gravity. A $\chi^2$ fitting method is then use to find the values of
\vsini\ and $y$ in the models that will best fit the observed line
profiles of the helium lines. Since the $\chi^2$ value is dependent on
the number of `spectral bins', it will vary depending on how much of
the spectrum is examined. To make comparisons between individual lines
possible, we usually use the normalised-$\chi^2$, $\chi_\nu^2 =
\chi^2/\nu$, where $\nu$ is the \emph{number of degrees of freedom},
which is usually the number of bins minus two.

The effective temperature, \teff, and surface gravity, \logg, both
affect the widths of the lines. Varying $y$ only changes the strengths
of the lines and does not directly affect any of the broadening
functions. Thus it is possible to get a good fit for the
line broadening in a consistent way for all the \hei\ lines,
even when the initial estimates of the stellar parameters could be
slightly wrong.

\begin{table}
\begin{center}
\caption{Fitted values for \vsini\ and macroturbulence for 10~Lac}
\vspace{\abovecaptionskip}
\begin{tabular}{c|ccc|ccc}
\hline\hline
\hei\ $\lambda$	& \multicolumn{3}{c|}{\vsini} 	& \multicolumn{3}{c}{m.turb.} \\
(\AA)	& $v$ (\kms) & $y$ & $\chi_\nu^2$ & $v$ (\kms) & $y$ & $\chi_\nu^2$ \\
\hline
4009	& 61	& 0.07	& 1.548	& 50	& 0.07 	& 1.515\\
4026	& 21	& 0.09	& 2.122	& 14	& 0.10	& 2.148\\
4143	& 44	& 0.07	& 0.412	& 33	& 0.08	& 0.382\\
4388	& 23	& 0.08	& 0.718	& 16	& 0.08	& 0.732\\
4922	& 31	& 0.12	& 2.717	& 22	& 0.13	& 2.727\\
5047	& 29	& 0.10	& 0.222	& 22	& 0.10	& 0.212\\
\hline
\multicolumn{1}{r|}{Median:} 	& 30	& & & 22 \\
\hline\hline
\end{tabular}
\label{ta:10lacparams}
\end{center}

\tnote{Values are fitted for initial `guesstimate' of stellar
parameters (\teff~=~37~kK; \logg~=~4.3). Note that in this case, $y$ is
not an accurate representation of the actual helium number fraction of
the star, but indicates whether the model line for the given
parameters is too strong ($y~<~0.09$) or too weak
($y~>~0.09$). However, if all the lines are too weak it could indicate
that the star does have a helium abundance higher than the solar value
(but could also indicate the wrong choice of \teff\ and/or \logg).}
\end{table}

Two different line broadening functions are compared to the
observations, as described in chapter \ref{ch:mod}. From table
\ref{ta:10lacparams} it appears that there is no statistically significant
difference between the goodness of fit (given by the $\chi_\nu^2$
value) for the rotational and the macroturbulent broadening functions
for 10 Lac and we adopt the rotational function. The reasons for this
choice will be further discussed in chapter \ref{ch:analyses}; however
as noted above, the line broadening is unimportant for the final
results when the line broadening velocity is low.

\ybcite{ho:xcorr} found the \vsini\ of 10 Lac to be 35~$\pm$~14~\kms,
and other analyses of this star (see table \ref{ta:10Lac_results})
adopted values between 25 and 30 \kms. These values compare well with
the 30~\kms\ adopted here (typical uncertainties on velocity
measurements are of the order 10--15~\kms; see chapter \ref{ch:analyses}). 


%%%%%%%%%%%%%%%%%%%%%%%%%%%%%%%%%%%%%%%%%%%%%%%%%%%%%%%%
%%%%%%%%%%%%%%% SECTION = CHI-SQUARE %%%%%%%%%%%%%%%%%%%
%%%%%%%%%%%%%%%%%%%%%%%%%%%%%%%%%%%%%%%%%%%%%%%%%%%%%%%%
\section{$\chi^2$ maps of hydrogen line wings}
\label{sec:chi-sq}

The pressure-broadened wings of the hydrogen lines are used to fix the
surface gravity. They are relatively insensitive to changes in \teff\
and show almost no dependence on the helium abundance (see fig
\ref{fig:h_sensitivity}).

The equivalent width, $W$, is not a useful parameter for the hydrogen
lines since their cores are very sensitive to stellar wind
contamination and also because their broad wings make the measured $W$
very sensitive to continuum placement. Because of their broad
profiles, they are also often blended with other lines. However, it is
possible to extract useful information from these lines by
concentrating on the wings. The method developed for this thesis
involves extracting just the wings from the spectra by removing the
cores of the lines and any blends that might occur. Then the wings are
compared to the synthetic model spectra. For each grid-point (i.e.
value of \teff\ and \logg) a $\chi_\nu^2$ fit value is calculated as
in the previous section. These $\chi_\nu^2$ values are tabulated in a
two-dimensional `map' in the \teff - \logg\ plane and an automatic
routine traces the $\chi_\nu^2$ minima. The resulting locus of
points describes a line of best-fit values of \teff\ and
\logg.

This is a very time consuming (both human and cpu) and due to the \hi\
lines' insensitivity to the helium abundance, $y$, it usually only
calculated for $y$ = 0.09. However, if the star has a high helium
abundance (say $y~\geq~0.14$) then the $\chi_\nu^2$ map is
re-calculated.

\begin{figure} %%%%%%%%%%%%%%%%%%% FIGURE

\epsfxsize=\figwidth
\setlength{\cen}{(\textwidth / 2) - (\epsfxsize / 2)}
\hspace{\cen}\epsfbox{fig_casestudy/hydrogen/hydrogen.eps}

\caption[Sensitivity of \hi\ lines]
{\fcfont Model predictions of the sensitivity of H10~$\lambda$3797\AA\ to
physical parameters. Diagrams illustrate, from top to bottom,
sensitivity to changes in gravity, in helium number fraction
($y$), and in rotational broadening parameter (\vsini). It is
clear that of these, gravity is the most important parameter. Only
a small part of the wings is shown to emphasize differences between
models.}
\label{fig:h_sensitivity}
\end{figure} %%%%%%%%%%%%%%%%%%% END FIGURE

%%%%%%%%%%%%%%%%%%%%%%%%%%%%%%%%%%%%%%%%%%%%%%%%%%%%%%%%
%%%%%%%%%%%%%%% SECTION = Fit DIAGRAMS %%%%%%%%%%%%%%%%%%
%%%%%%%%%%%%%%%%%%%%%%%%%%%%%%%%%%%%%%%%%%%%%%%%%%%%%%%%
\section{Fit diagrams}
\label{sec:ew_analysis}

The measurements discussed in sections \ref{sec:ews} and
\ref{sec:chi-sq} are used to construct fit diagrams similar to those
used by, among others, \ybcite{he:92} and \ybcite{sm:94}. The loci of
closest equivalent widths for helium lines, and best-fit wings for
hydrogen lines, are plotted in the \teff\ -- \logg\ plane for a given
value of $y$ and microturbulence. The best-fit parameters are found by
calculating the point of closest approach to all the fitted lines. The
goodness-of-fit is estimated from the root-mean-square distance of the
point to the fit-lines. Diagrams are constructed for values of $y$
between 0.09 and 0.18 and for microturbulence-values of 0, 5, 10 and
15 \kms.

The fit diagram for 10 Lac is show in figure \ref{fig:10lac:ewd} and a
sequence of diagrams for varying $y$ and microturbulence is shown in
appendix \ref{app:osumm}. The best-fit value is found by defining a
grid of points that surrounds the lines on the fit diagram. For each
point in the grid, the shortest distance to each line is found, and
the sum of squares of these distances is calculated. The point in the
grid with the smallest sum of squares is taken to be the best-fit
point. There is a slight ambiguity in defining the distances between
point on the fit diagrams since the axes are not in the same
units. The effective temperatures will typically be of the order 30000
-- 40000, whereas the surface gravities are of the order 3.0 --
4.5. Unless these values are scaled, the temperatures will dominate in
the calculation of impact parameters. In an attempt at ``normalising''
the axes, the y-axis (\logg) is simply multiplied by 10 (and effective
temperatures are measured in kK). Thus both axes have units which are
of the same magnitude.

\begin{figure} %%%%%%%%%%%%%%%%%%% FIGURE
%\fnote{Fit diagram for 10 Lac}

\epsfxsize=12cm
\setlength{\cen}{(\textwidth / 2) - (\epsfxsize / 2)}
\hspace{\cen}\epsfbox{fig_casestudy/ewd.eps}

\caption[Fit diagram for 10 Lac]
{\fcfont The fit diagram for 10 Lac at $y = 0.09$. The diagram is
plotted in the same orientation as the Hertzsprung-Russel diagram,
with \logg\ ($\equiv$ luminosity) increasing downwards and
\teff\ increasing to the left. Each line on the plot represents the
locus of parameters for the model spectrum which best fits either the
equivalent width (for the helium lines) or the line-wing profile (for the
hydrogen lines).}
\label{fig:10lac:ewd}
\end{figure} %%%%%%%%%%%%%%%%%%% END FIGURE

From the fit diagram it appears that most lines are in fairly good
agreement, with none of the lines falling outside the plotted
area. There do, however, appear to be some lines that lie well
outside the $\pm$ 5\%  variations shown in figures
\ref{fig:ew_errors:he1} and \ref{fig:ew_errors:he2} (a discussion of
the uncertainties and these plots follows later). The \hei\ lines at
4009\AA\ and 4471\AA\ and the \heii\ line at 4200\AA\ all appear to
lie well outside the intersection zone of the greater number of
lines. The \hei\ lines are both among the lines in table
\ref{ta:elf_vs_ew} with the largest differences in measured equivalent
widths. \hei\ $\lambda$4009\AA\ is a weak line, and appears to be
blended in all the observed stars, and this blend could lead to the
observed over-estimation of the measured equivalent width. \hei\
$\lambda$4471\AA\ on the other hand, is a relatively strong line, and
the measured equivalent width is in this case under-estimated. This is a
triplet line with a strong forbidden component. There are also
several blends in the line wings. Because this line has a strong but
narrow line profile, and because of the resolved forbidden component,
it is very poorly fitted using \elf. The blends in the line wings make
reliable measurements using \comm{ew} difficult. The \comm{ew}
measurements (715~m\AA) are significantly larger than the \elf\
estimates (646~m\AA), and in the model spectrum for \teff ~=~38~kK and
\logg ~=~4.2 the equivalent width between 4466\AA\ and 4476\AA\ is
766~m\AA. However in the unblended region between 4468\AA\ and
4474\AA\ the model equivalent width is only 663~m\AA, so it appears
that the model is not in disagreement with the observations.

In the case of \heii\ $\lambda$4200\AA\ (4199.8\AA), the disagreement
between model and observations is most likely caused by a blend with
a N\scr{3}\ line at 4200.0\AA. Using \elf\ to fit \heii\
$\lambda$4200\AA\ and both N\scr{3}\ lines at 4195.7\AA\ and 4200.0\AA\
(see figure \ref{fig:he2_4200fit}) yields an equivalent width of 433
m\AA\ which is in good agreement with the 454 m\AA\ predicted by the model.

\begin{figure} %%%%%%%%%%%%%%%%%%% FIGURE

\epsfxsize=12cm
\setlength{\cen}{(\textwidth / 2) - (\epsfxsize / 2)}
\hspace{\cen}\epsfbox{fig_casestudy/he2_4200/he2_4200_fit.eps}

\caption[\elf\ fit to \heii\ $\lambda$4200\AA]
{\fcfont \elf\ fit used to estimate equivalent width of \heii\
$\lambda$4200\AA. The dotted line shows the \elf\ fit to the \heii\
line component, and the dashed line shows the fit to the combined line
profile. The \comm{ew} measurements included the blend with N\scr{3}
and overestimated the equivalent width of the \heii\ line. Using \elf\
we can get a far better measurement, although because the line
profiles are not well fitted by Gaussians it is likely that the \heii\
line strength is underestimated in this figure.}
\label{fig:he2_4200fit}
\end{figure} %%%%%%%%%%%%%%%%%%% END FIGURE

It is difficult to give an accurate breakdown of the uncertainties that might
be included in the analysis. Obviously the method is limited by the
physics of the models and these results are difficult to quantify. It
is slightly easier to look at the possible errors on the
measurements. In the process of measuring the equivalent widths, the
main pitfall is the determination of the continuum. For some lines
this is relatively easy; however, when the lines are in a part of the
spectrum with many other lines it can be very difficult. Nonetheless, by
looking back at table \ref{ta:elf_vs_ew} (where we compare equivalent
widths measured independently using two different methods and
repeating the steps of spectral normalisation), it seems that the
uncertainties on the equivalent widths are within a few per cent for most
lines, and that an error estimate of $\pm 5\%$ is probably quite
reasonable. Figures \ref{fig:ew_errors:he1} and
\ref{fig:ew_errors:he2} show the effects of adding these errors to the
\hei\ and \heii\ lines respectively. The change in position of the
plotted lines is more or less of the order of the minimum spread seen
in the fit-diagrams (e.g. HD~195592, HD~202124; see fig
\ref{fig:allewd}). This could imply that we have underestimated the
uncertainty on the equivalent width measurements, but more likely it is
down to a combination of uncertainties in measurements and
omissions/simplifications in the models.

\begin{figure} %%%%%%%%%%%%%%%%%%% FIGURE
%\fnote{Errors on \hei\ lines}

\epsfxsize=12cm
\setlength{\cen}{(\textwidth / 2) - (\epsfxsize / 2)}
\hspace{\cen}\epsfbox{fig_casestudy/errors_he1.eps}

\caption[Equivalent widths of \hei\ lines plotted with errors]
{\fcfont These diagrams show the \hei\ lines plotted as they would
appear on a fit diagram using the equivalent widths predicted by the
models for the adopted parameters of 10 Lac (solid lines). Error
estimates of $\pm 5$ per cent on the equivalent widths are shown as
dashed lines (+5\%) and dotted lines (-5\%).}
\label{fig:ew_errors:he1}
\end{figure} %%%%%%%%%%%%%%%%%%% END FIGURE

\begin{figure} %%%%%%%%%%%%%%%%%%% FIGURE
%\fnote{Errors on \heii\ lines}

\epsfxsize=12cm
\setlength{\cen}{(\textwidth / 2) - (\epsfxsize / 2)}
\hspace{\cen}\epsfbox{fig_casestudy/errors_he2.eps}

\caption[Equivalent widths of \heii\ lines plotted with errors]
{\fcfont These diagrams show the \heii\ lines plotted as they would
appear on a fit diagram using the equivalent widths predicted by the
models for the adopted parameters of 10 Lac (solid lines). Error
estimates of $\pm 5$ per cent on the equivalent widths are shown as
dashed lines (+5\%) and dotted lines (-5\%).}
\label{fig:ew_errors:he2}
\end{figure} %%%%%%%%%%%%%%%%%%% END FIGURE



%%%%%%%%%%%%%%%%%%%%%%%%%%%%%%%%%%%%%%%%%%%%%%%%%%%%%%%%
%%%%%%%%%%%%%%% SECTION = RESULTS %%%%%%%%%%%
%%%%%%%%%%%%%%%%%%%%%%%%%%%%%%%%%%%%%%%%%%%%%%%%%%%%%%%%
\section{Physical parameters of 10 Lac}

The results of the impact parameter analysis are listed in table
\ref{ta:10Lac:imp}. From this table it appears that both temperature
and gravity are well constrained by the equivalent width
diagrams. It is also clear that the fits are significantly better for
high microturbulent velocities of the order 10-15 \kms. The parameter
$p$ listed in the table is the root-mean-square distance to the lines
as discussed in section \ref{sec:ew_analysis}. The smallest value of
$p$ is for $y$~=~09 and $\xi$~=~15~\kms, and we adopt these values,
and \teff~=~38,000~K and \logg~=~4.2 as the best-fit parameters for
10~Lac. Comparisons of model spectra for these parameters and the
observations are shown in figures \ref{fig:model_fit_he1} and
\ref{fig:model_fit_he2+h}.

The \hei\ lines fit the observed equivalent widths very well, but for
all lines apart from 4009\AA\ and 4143\AA\ the model line profiles are
too weak in the cores. Lowering the \vsini\ by 10 -- 15\kms\ resolves
this, but the profiles then appear too narrow. The model \heii\ lines
are consistently too narrow and the line at 4200\AA\ is also clearly
blended as discussed earlier. The \hi\ lines fit the line wings very
well, but the model cores are clearly too weak and too narrow. In
addition to this, the red wing of H10~\lam3797\AA\ does not fit
the observations. It is possible that the normalisation at this point
is poor. This line appears in the first echelle order observed and the
normalisation could be affected by edge effects (see also chapter
\ref{ch:obs}).


%%%%%%%%%%%%%%%% TABLE %%%%%%%%%%%%%%
\begin{table}
\begin{center}
\caption[Impact-parameter analysis for 10 Lac]{Impact parameter
	analysis for 10 Lac.}
\vspace{\abovecaptionskip}
\begin{tabular}{cr|ccc}
\hline\hline
$y$	& $\xi$	(\kms) & \teff\ (kK)	& \logg	& $p$ \\
\hline
09	&  0	& 37.6	& 4.21	& 1.439	\\
	&  5	& 37.7	& 4.24	& 1.437 \\
	& 10	& 37.7	& 4.22	& 1.376 \\
	& 15	& 37.6	& 4.16	& 1.136 \\
&&&&\\
10	&  0	& 37.6	& 4.20	& 1.419 \\
	&  5	& 37.5	& 4.16	& 1.430 \\
	& 10	& 37.5	& 4.18	& 1.363 \\
	& 15	& 37.5	& 4.09	& 1.142 \\
&&&&\\
11	&  0	& 37.8	& 4.20	& 1.459 \\
	&  5	& 37.8	& 4.18	& 1.463 \\
	& 10	& 37.0	& 4.02	& 1.167 \\
	& 15	& 37.5	& 4.03	& 1.281 \\
\hline\hline
\end{tabular}
\label{ta:10Lac:imp}
\end{center}
\tnote{$\xi$ is the microturbulence and $p$ is
	a number representing the goodness of fit. It is given by the
	sum of the distances squared. A small $p$ represents a tight
	clustering of the lines in the diagram, with $p = 0$ if all
	the lines intersect at a single point.}
\end{table}
%%%%%%%%%%%%%% END TABLE %%%%%%%%%%%


\begin{figure} %%%%%%%%%%%%%%%%%%% FIGURE
\epsfxsize=15cm
\setlength{\cen}{(\textwidth / 2) - (\epsfxsize / 2)}
\hspace{\cen}\epsfbox{fig_casestudy/model_fit_he1.eps}
\caption[10 Lac \hei\ model fits]
{\fcfont Model fits to \hei\ lines in 10~Lac.}
\label{fig:model_fit_he1}
\end{figure} %%%%%%%%%%%%%%%%%%% END FIGURE

\begin{figure} %%%%%%%%%%%%%%%%%%% FIGURE
\epsfxsize=15cm
\setlength{\cen}{(\textwidth / 2) - (\epsfxsize / 2)}
\hspace{\cen}\epsfbox{fig_casestudy/model_fit_he2+h.eps}
\caption[10 Lac \heii\ and \hi\ model fits]
{\fcfont Model fits to \heii\ and \hi\ lines in 10~Lac.}
\label{fig:model_fit_he2+h}
\end{figure} %%%%%%%%%%%%%%%%%%% END FIGURE


%%%%%%%%%%%%%%%%%%%%%%%%%%%%%%%%%%%%%%%%%%%%%%%%%%%%%%%%
%%%%%%%%%%%%%%% SECTION = DISCUSSION %%%%%%%%%%%%%%%%%%%
%%%%%%%%%%%%%%%%%%%%%%%%%%%%%%%%%%%%%%%%%%%%%%%%%%%%%%%%
\section{Discussion}
\label{sec:casestudy:discussion}

As shown in table \ref{ta:10Lac_results}, our results are in good
agreement with \ybcite{sc:88}, but disagree with
\ybcite{gr:92}. \acite{gr:92} put this difference down to the
inclusion of \heii\ $\lambda$4686\AA\ in \acite{sc:88} We have also
included this line; however it is in good agreement with all the other
lines. Excluding this line does not appear to affect the analysis; the
impact parameter results for 10~Lac exluding \hei\ \lam4686\AA\ are
\teff~=~37.7~kK and \logg~=~4.16 (for $y=0.09$; $\xi =
15$~\kms). Including this line gave \teff~=~37.6~kK. Clearly \heii\
$\lambda$4686\AA\ has no effect on the final results in this case.

The likely cause of the differences (also noted for other stars, see
\ref{ch:analyses}) must then be put down to differences in the
atmospheric models and fitting methods used. In particular,
\acite{gr:92} used the cores of \ha, \hb\ and \hg\ and the \hei\ lines
\lam4471\AA\ and \lam4922\AA\ to determine temperatures. The cores of
the hydrogen lines are likely to be filled-in by emission from the
stellar wind, although from the plots in \acite{gr:92} this would
appear to have the effect of increasing the adopted effective
temperatures, contrary to what would be required to bring their
results into agreement with ours. The line \hei\ \lam4471\AA\
consistently predicts lower temperature than the rest in our study.
It is possible that this line could have affected their results
(although \acite{gr:92} note that their models also fail to reproduce
the strength of this line). It is also worth noting that \acite{gr:92}
used line-blanketed atmospheric models, and
\ybcite{hu:98} showed that when comparing analyses using H/He models to
analyses using line-blanketed models, the H/He model analyses
\emph{overestimate} the derived temperatures. \acite{hu:98} found
differences between the approaches of the order 2.5~kK, however he
only `analysed' synthetic spectra, not actual observations.


%%%%%%%%%%%%%%%% TABLE %%%%%%%%%%%%%%
\begin{table}
\begin{center}
\caption{Published parameters for 10~Lac}
\vspace{\abovecaptionskip}
\label{ta:10Lac_results}
\begin{tabular}{r|cccc}
\hline\hline
       & This study	& Sch\"onberner et al. & Grigsby et al. & Herrero et al. \\
\hline
\teff  & 38,000 K	& 38,000 K & 30,000 K & 37,500 K\\
\logg  & 4.20		& 4.25     & 4.00     & 4.00    \\
$y$    & 0.09		& 0.09     & 0.10     & 0.10    \\
\vsini & 30 \kms	& 25 \kms  & 30 \kms  & 50 \kms \\
\hline\hline
\end{tabular}
\end{center}
\tref{\ybcite{sc:88}; \ybcite{gr:92}; \ybcite{he:92}}
\end{table}
%%%%%%%%%%%%%% END TABLE %%%%%%%%%%%








