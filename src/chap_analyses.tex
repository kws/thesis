\chapter{O Star Analyses}
\label{ch:analyses}

In this chapter, we apply the methods described in chapter
\ref{ch:casestudy} to all of the WHT O stars. Briefly, this 
involves estimating  \teff\ and \logg. Using model spectra
computed for these parameters, we can find the line broadening velocity for
each star. These results are discussed in section
\ref{sec:broadening}. Next, we measure equivalent widths and fit the
hydrogen line wings using a $\chi^2$ fitting procedure to construct
fit-diagrams in the \teff ~--~\logg\ plane. Series of these diagrams,
constructed for different values of the helium number fraction, $y$,
and the microturbulence, $\xi$, are used to find the values of these
parameters that best fit all the measured lines. The appearance of the
fit diagrams, model spectra and peculiarities of each star are
briefly discussed in section \ref{sec:analyses}. The fit diagrams for
all the sample stars are shown in figure \ref{fig:allewd} at the end
of this chapter.

As explained in chapter \ref{ch:casestudy}, the stellar parameters are
found by using an automated routine to find the point of closest
approach to all the lines in the fit diagrams. This routine sets up a
grid surrounding the intersection zone, and calculates the sum of
impact parameters for each grid point. This method works well when the
intersection zone is close to the centre of the
diagrams. Unfortunately, quite often the stellar parameters are
towards one of the edges of the grid of models. At these points unfortunate
`edge effects' can occur. Notably, lines that fall outside the grid
(for higher values of \logg) are treated as horizontal lines at the
bottom of the diagrams. These lines are, however, easy to identify and
remove from the analysis. Worse is the tendency of lines to cluster
towards the low \logg\ end of the diagrams. The grid of models goes as
close to the Eddington limit as was computationally possible, and at
these low values for \logg\ the equivalent width changes rapidly with
$g$. This effect is clearly apparent in the fit diagrams for the star
BD+36~4063 (figure \ref{fig:ewd:bd+36}).

\begin{figure} %%%%%%%%%%%%%%%%%%% FIGURE
%\fnote{Flowchart demonstrating the general method for fixing parameters}
\epsfxsize=\figwidth
\setlength{\cen}{(\textwidth / 2) - (\epsfxsize / 2)}
\hspace{\cen}\epsfbox{fig_app_osumm/ewd/BD+364063.eps}
\caption[Fit diagrams for BD+36~4063]
{\fcfont Series of fit diagrams for BD+36~4063 for varying values of
the helium number fraction, $y$, and the microturbulence, $\xi$. Note
the tendency for the \hei\ lines to cluster towards the lower \logg\
(top) end of the grid as $y$ increases. This can confuse the impact
parameter calculating routines.}
\label{fig:ewd:bd+36}
\end{figure} %%%%%%%%%%%%%%%%%%% END FIGURE

The impact parameter routines are also fairly sensitive to outlier
lines, so they do include some automatic rejection routines. These
routines work by first finding the point closest to all the lines,
then rejecting those lines that are further than some arbitrary value
from that point and re-calculating. If the first point is close to the
`correct' value, then this method usually works well. However, if the
starting point is wrong, rejecting the outliers often rejects reliable
lines, and the solution gets even worse. Because of this sensitivity
to outliers, the lines that are more or less consistently in
disagreement with the other lines, like \hei~$\lambda$4471\AA\ and
\heii~$\lambda$4686\AA, have manually been excluded from
the analyses where necessary.


%%%%%%%%%%%%%%%%%%%%%%%%%%%%%%%%%%%%%%%%%%%%%%%%%%%%%%%%
%%%%%%%%%%%%%%% SECTION = Line broadening profiles %%%%%
%%%%%%%%%%%%%%%%%%%%%%%%%%%%%%%%%%%%%%%%%%%%%%%%%%%%%%%%
\section{Line broadening functions}
\label{sec:broadening}

The velocity characterising the best-fit line-broadening function in
each star was found, as described in chapter \ref{ch:casestudy}, by
creating a synthetically broadened model spectrum and using
statistical methods to find the best-fit broadening function and
velocity parameter. The median of the velocities found from the \hei\
lines was adopted as the broadening velocity of the star.

\begin{table} %%%%%%% TABLE %%%%%%%%%%%%%%%
\begin{center}
\caption{Example of line broadening analysis.}
\label{ta:broadening}
\vspace{\abovecaptionskip}
\small
\begin{tabular}{c !{\hspace{18pt}} ccc !{\hspace{18pt}} ccc}
\multicolumn{7}{c}{HD~12323 -- ON9~V}\\
\hline\hline
\hei\ $\lambda$ & \multicolumn{3}{c}{\vsini}   & \multicolumn{3}{c}{macroturb.} \\
(\AA) & $v$ (\kms) & $y$ & $\chi^2$ & $v$ (\kms) & $y$ & $\chi^2$ \\
\hline
4009    & 153	& 0.12	& 0.491	& 124	& 0.13	& 0.491 \\
4026    & 118	& 0.13	& 0.424	& 88	& 0.14	& 0.586 \\
4143    & 142	& 0.17	& 0.191	& 115	& 0.20	& 0.179 \\
4388    & 117	& 0.10	& 0.459	& 90	& 0.11	& 0.521 \\
4922    & 128	& 0.13	& 0.373	& 95	& 0.14	& 0.498 \\
5047    & 134	& 0.11	& 0.231	& 117	& 0.14	& 0.253 \\
\hline
Median:	& 131   &      	& 0.399 & 105   &       & 0.495 \\
\hline\hline
&\\

\multicolumn{7}{c}{HD 218195 -- O9~III}\\
\hline\hline
\hei\ $\lambda$ & \multicolumn{3}{c}{\vsini} & \multicolumn{3}{c}{macroturb.} \\
(\AA)   & $v$ (\kms) & $y$ & $\chi^2$ & $v$ (\kms) & $y$ & $\chi^2$ \\
\hline
4009    & 132	& 0.13	& 0.500	& 104 	& 0.14 & 0.318 \\
4026    & 72 	& 0.10 	& 1.011 & 53	& 0.10 & 0.986 \\
4143    & 85 	& 0.08 	& 0.320 & 70 	& 0.09 & 0.184 \\
4388    & 78 	& 0.08	& 0.493	& 62 	& 0.08	& 0.391 \\
4922    & 70 	& 0.11	& 0.321	& 56 	& 0.11 & 0.216 \\
5047    & 65 	& 0.07	& 0.201	& 54 	& 0.07	& 0.181 \\
\hline
Median:	& 75 	&       & 0.407	& 59	&       & 0.267 \\
\hline\hline
&\\

\multicolumn{7}{c}{HD~30614 -- O9.5~Ia}\\
\hline\hline
\hei\ $\lambda$ & \multicolumn{3}{c}{\vsini} & \multicolumn{3}{c}{macroturb.} \\
(\AA)   & $v$ (\kms) & $y$ & $\chi^2$ & $v$ (\kms) & $y$ & $\chi^2$ \\
\hline
4009    & 110	& 0.11	& 0.232	& 94	& 0.12	& 0.190 \\
4026    & 103	& 0.22	& 0.468	& 76	& 0.22	& 0.411 \\
4143    & 108	& 0.11	& 0.249	& 92	& 0.12	& 0.168 \\
4388    & 120	& 0.15	& 0.216	& 95	& 0.16	& 0.124 \\
4922    & 107	& 0.22	& 0.310	& 88	& 0.24	& 0.199 \\
5047    & 105	& 0.12	& 0.268	& 80	& 0.13  & 0.173 \\ 
\hline
Median:	& 108	&       & 0.259	& 90    &       & 0.182 \\
\hline\hline
\end{tabular}
\end{center}
\end{table} %%%%%%%%%%% END TABLE %%%%%%%%%%%

Table \ref{ta:broadening} shows the results of the line broadening
analyses for three `typical' stars. They are representative of the
complete sample, in the sense that the inferred rotational velocities
are always higher than the inferred macroturbulent velocities. This
effect is intrinsic to these functions; the macroturbulent function is
shallower and broader than a rotational function with the same
velocity (see chapter \ref{ch:mod}).

The stars in table \ref{ta:broadening} are also `typical' in that the
rotational broadening function fits better for the main-sequence star, and
the macroturbulent broadening function fits better for all the other
stars (see table
\ref{ta:sample_vsini} for results for the complete sample). This is
true for all the stars, except AE~Aur (HD~34078; O9.5~V), 10~Lac
(HD~214680; O9~V) and the two stars with the broadest lines: HD~13745
(O9.7~II) and HD~16429 (O9.5 II). For the two main-sequence stars there is
no real difference in the reduced $\chi^2$ values for the two fits,
and neither broadening function fits the observations particularly
well. Unfortunately there are only 4 main-sequence stars in the
dataset, so the sample is too small to draw any firm
conclusions. However, it is not unreasonable to assume that the
dominating line-broadening effect in main-sequence stars is rotation,
whereas in the giants and supergiants, with their large, extended
atmospheres, there could be other velocity fields contributing to the
line broadening.

There are several theoretical reasons why an O star should slow down
as it ages. \ybcite{co:vii} have outlined several such
theories. Firstly, as a star evolves, its radius increases, and from
conservation of angular momentum, it should slow down. Secondly, a
star may also lose angular moment because of its stellar
wind.\footnote{This is only the case if the material remains somehow
bound to the photosphere, for example through a magnetic field. See
discussions in \ybcite{la:98}, \ybcite{ma:98} and \ybcite{me:98}.}
\ybcite{pe:xcorr} has calculated the wind-related ``spin-down'' time
of a O9 main sequence star. It turns out that this is more or less six
times longer than the star's H-burning lifetime. However, there is
observational and theoretical evidence that as a star evolves away
from the main sequence, its mass-loss rate might increase and thus the
rate of spin-down from this process should also increase.

There is also a lack of narrow-lined O supergiants. In the
comprehensive survey of O stars by \ybcite{ho:xcorr} the
slowest-rotating O supergiant had a \vsini~=~68~\kms. The
narrowest-lined O supergiant in the present sample (HD~195592) has a
macroturbulent velocity of 54~\kms\ (the best-fit rotational profile
for this star is 70~\kms, in reasonable agreement with the published
values shown in table \ref{ta:sample_vsini}). This clearly indicates
that there must be some other effect apart from rotation which
contributes to the line broadening in evolved O stars. We have
therefore chosen to adopt rotational profiles for all main-sequence
stars and macroturbulent profiles for the rest of the sample, except
for the two stars with the broadest lines where it appears that rotation
is dominating. 

Such a generalisation might however be dangerous. Since there appears
to be a total lack of zero-velocity O stars (main-sequence and
evolved; see discussions in \cite{pe:xcorr} and \acite{ho:xcorr}), and
10~Lac and AE~Aur are both among the narrowest-lined O stars known, it
is not unreasonable to assume that these stars might be rotating, but
that they are seen pole-on. If this is the case, then the observed
broadening can only be caused by some other broadening mechanism. It
is also not unreasonable to assume that this is the same mechanism as
seen in the evolved stars, only not as strong. In this case, at least
for these two stars, the lines should have the shape of the
macroturbulent broadening function. However, since neither profile
fits well, we will use the rotational broadening functions for all
main-sequence stars as discussed above. In any case, the broadening
functions do not affect our derived parameters since these rely on
equivalent widths for the helium lines, and the broadening in the
hydrogen lines is dominated by the intrinsic line-width.

Estimating an uncertainty on our adopted broadening-velocities is
complicated due to the lack of published velocities to compare
with. No other measurements using the macroturbulent profile are
available for comparison. However, several of the sample stars have a
\vsini\ between 90 -- 95~\kms\ listed in \ybcite{ho:xcorr}. The range in
our adopted macroturbulent velocities for the same stars is 65 -
85~\kms (only one of these stars has a macroturbulent velocity higher
than 70~\kms). Computing a synthetic line profile with a \vsini\ of
93~\kms\ and then fitting this synthetic line with a macroturbulent
profile, yields a macroturbulent velocity of 73~\kms, which is in very
good agreement with our adopted velocities. An estimate of the errors
on the macroturbulent velocities is then about $\pm$~10~\kms\ (and
probably the same for the rotational velocities). This spread in
velocities is slightly smaller than seen in the variations between the
individual lines in each spectrum, and might as such be a slightly
conservative estimate.

\begin{table} %%%%%%% TABLE %%%%%%%%%%%%%%%
\begin{center}
\caption{Line broadening velocities}
\label{ta:sample_vsini}
\vspace{\abovecaptionskip}
\begin{tabular}{llccccrr}
\hline\hline
HD/BD & \multicolumn{1}{c}{Spectral} & \multicolumn{4}{c}{\vsini\
        (\kms)} & \multicolumn{2}{c}{this thesis} \\ &
        \multicolumn{1}{c}{Type} & UF & CE & P96 & H97 & rot. $v$ & mac. $v$ \\
\hline
10125 	& O9.7 II 	&  95 	& ...	& ...	& ...	& 161	& {\bf 132} \\
12323	& ON9 V		& 130	& 100	& 113	& 131	& {\bf 131} &  105 \\
13745 	& O9.7 II((n))	& 185	& ...	& 168	& 169	& {\bf 176} &  151 \\
16429 	& O9.5 II((n))	& 165	& 140	& ...	& ...	& {\bf 216} &  175 \\
30614 	& O9.5 Ia  	&  80	&  85	& 115	& 129	& 108	& {\bf  90} \\
34078	& O9.5 V	&  25	&  27	&  25	&  30	& {\bf 30} 	&  41 \\
36486 	& O9.5 II	& 145	& 109	& ...	& 144	& 153	& {\bf 126} \\
37742	& O9.7 Ib	& 135	& 110	& 123	& 124	& 120	& {\bf 109} \\
188209	& O9.5 Iab	&  80	&  70	&  87	&  92	&  79	& {\bf  65} \\
189957	& O9.5 III	&  90	& ...	&  89	&  92	& 106	& {\bf  85} \\
191781	& ON9.7 Iab	&  80	& ...	& ...	& ...	& 110	& {\bf  89} \\
194280	& OC9.7 Iab	& ...	& ...	& ...	& ...	& 120	& {\bf 101} \\
195592	& O9.7 Ia	&  85	&  60	& ...	& ...	&  68	& {\bf  54} \\
201345	& ON9 V		&  95	&  54	&  87	&  91	& {\bf 109}	&  82\\
202124	& O9.5 Iab	&  95	&  60	& 137	& 115	& 111	& {\bf 93} \\
207198	& O9 Ib-II	&  80	&  70	&  91	&  91	&  78	& {\bf 67} \\
209975	& O9.5 Ib	&  40	& ...	&  90	&  95	&  86	& {\bf 69} \\
210809	& O9 Iab	& 120	& 101	& 118	& 117	& 112	& {\bf 89} \\
214680	& O9 V		&  30	&  32	&  31	&  35	& {\bf 30}	&  22\\
218195	& O9 III	&  85	& ...	& ...	&  70	&  75	& {\bf 59} \\
218915 	& O9.5 Iab	& 115	& 110	&  74	&  94	&  81	& {\bf 68} \\
225160	& O8 Ib(f)	& 105	&  95	& ...	& ...	& 124	& {\bf 109} \\
+36 4063& ON9.7 I	& ...	& ...	& ...	& ...	& 106	& {\bf 84} \\
\hline\hline 
\end{tabular}
\end{center}
\tnote{The adopted broadening velocities are indicated by
\textbf{bold} text.}
\tref{UF -- \ybcite{uesugi}, CE -- \ybcite{co:vii}, P96 --
\ybcite{pe:xcorr}, H97 -- \ybcite{ho:xcorr}}
\end{table} %%%%%%%%%%%%%%%%%%%%%%%%%%%%

%%%%%%%%%%%%%%%%%%%%%%%%%%%%%%%%%%%%%%%%%%%%%%%%%%%%%%%%
%%%%%%%%%%%%%%% SECTION = Individual analyses %%%%%%%%%%
%%%%%%%%%%%%%%%%%%%%%%%%%%%%%%%%%%%%%%%%%%%%%%%%%%%%%%%%
\section{Notes on individual stars}
\label{sec:analyses}

This section contains brief summaries of the properties of the
individual stars in the sample. As mentioned earlier, the fit diagrams
for all the stars are shown in figure \ref{fig:allewd}, and fit
diagrams, results, and selected regions of the optical spectra are
shown in appendix~\ref{app:osumm}.

\subsection{HD 10125} %%%%%%%%%%%%%%
\label{analysis:hd010125}
This star was classified O9.7~II by \ybcite{wa:76}.  The spectrum of
this star appears normal, with a moderate wind contamination of the
\hi\ lines typical of its spectral type. There appears to be some weak
emission in the unidentified $\lambda\lambda$4465, 4505\AA\ doublet
(this doublet is briefly discussed in section
\ref{analysis:hd195592}). The fit diagram shows little scatter of the
lines, and gives \teff~=~32.5~kK,
\logg~=~3.3 for $y$~=~0.15. None of the \hei\ triplet lines
($\lambda\lambda$4026,~4471,~4713\AA) are in very good agreement with
the other lines on the fit diagrams, but the only two lines that are
in great disagreement are \hei~$\lambda$4471\AA\ and
\hei~$\lambda$4713\AA. \hei~$\lambda$4713\AA, as discussed in
section \ref{sec:ews}, is very insensitive to changes in \teff\ and
\logg, and the model spectrum fits the observations for this line
well. For \hei~$\lambda$4471\AA\ the synthetic spectrum is
consistently too weak for most of the stars in the sample. This is
undoubtedly due to the `generalized dilution effect' which was shown
by \ybcite{vo:89} to be very strong for this line (see also discussion
in section \ref{sec:ews}).


\subsection{HD 12323}  %%%%%%%%%%%%%%
\label{analysis:hd012323}
HD~12323 is among the ON main-sequence (ON9~V) stars classified by
\ybcite{wa:76}. The line profiles in this star are best fitted with a
rotational broadening function, and we have derived a rotational
velocity of 131~\kms. This corresponds well with the velocities found
by other authors (see table \ref{ta:sample_vsini}). The fit diagrams
show a fairly large scatter in the parameters predicted by the \hei\
lines. At surface gravities higher than \logg~=~3.7 there are models
calculated only for \logg~=~4.0 and 4.5 (below \logg~=~3.7 the grid is
calculated with a step size of 0.1~dex). This increased step-size,
coupled with the fact that the lines get more sensitive to changes in
equivalent width (see figures \ref{fig:cont:he1} and
\ref{fig:cont:he2}) is probably the cause of the large scatter. The
solution, however, is well constrained by the \hi\ and
\heii\ lines.

In this case, the impact parameter analysis has to be applied very
carefully, since the automatic procedures can very easily be fooled by
the horizontal appearance of lines that predict surface gravities
higher than the scope of the model grid. These lines have to be
removed before running the impact parameter routines. The smallest
scatter is found for $y$~=~0.17, and the predicted parameters are
\teff~=~36~kK and
\logg~=~4.1. The synthetic spectrum fits the observations well,
although the \hi\ line cores are consistently predicted too weak. This
is the opposite of what is observed for most stars in the sample,
where the observed cores are filled in by emission from the stellar wind. The
only ways of matching the observed \hi\ line profiles are by either
lowering the temperature and surface gravity of the models until they
fall well outside the O star range, or by lowering the rotational
velocity. Neither of these options is likely to give realistic
results, and they fail to reproduce the observed profiles of the \hei\
and \heii\ lines. Lowering the helium abundance to a solar value
($y$~=~0.09) has only a negligible effect on the \hi\ lines.

The \heii\ line strengths are well matched by the models, but in the
case of \heii~$\lambda$4200\AA\ the blend with N\scr{3} as discussed
in chapter \ref{ch:casestudy} is very obvious. \heii~$\lambda$4686\AA\
is also, unlike what is noted for most other stars in the sample,
predicted too weak.

\subsection{HD 13745}  %%%%%%%%%%%%%%
\label{analysis:hd013745}

This star is of spectral type O9.7 II((n)) (\cite{wa:76}), where `((n))'
indicates that the star has broad lines. The exact criterion used by
\ybcite{wa:71supp} and by the same author in subsequent papers, was that
the lines $\lambda\lambda$4116 and 4121\AA\ (see appendix
\ref{app:osumm}) are just merged (using spectra with a spectral
dispersion of 63~\AA~mm$^{-1}$). This star has, in fact, the second
highest broadening velocity of the sample, with a rotational velocity
of 176~\kms. It has also the highest helium abundance of the
sample. The impact parameter analysis indicates a helium abundance as
high as $y$~=~0.20, and also \teff~=~33~kK and \logg~=~3.3. The fit
diagram is fairly well constrained, with only the \hei\ triplet lines
$\lambda\lambda$4471,~4713\AA\ in serious disagreement. Comparisons of
synthetic and observed spectra are complicated by the fast rotation of
this star. At these high velocities the lines are very broad and it is
difficult to identify a continuum. Of the `problem' \hei\ lines,
$\lambda$4471\AA\ is predicted too weak, whereas $\lambda$4713\AA\
fits well. This line probably only appears to disagree with the other
lines in the fit diagram, due to its low sensitivy to \teff\ and \logg.

Of the other lines, the \heii\ lines fit moderately well, although
$\lambda$4686\AA\ is predicted too strong. This could be due to wind
emission contaminating this line. This is quite commonly observed in
giants and supergiants in this region of the H--R diagram.


\subsection{HD 16429}  %%%%%%%%%%%%%%
\label{analysis:hd016429}

HD~16429 was noted by \ybcite{wa:76} to be nitrogen deficient. He
classified it as O9.5~II((n)) where again ((n)) denotes very broad
lines. It has the broadest lines the sample with a
\vsini~=~216~\kms. The \hi\ lines show clearly asymmetric line
profiles (see plots on page
\pageref{sec:summary_hd016429} in appendix
\ref{app:osumm}) where the red wing is much steeper than the blue wing, 
probably caused by filling-in by emission in the stellar wind.

The fit diagrams for this star shows a large scatter in the \teff\ --
\logg\ plane. The helium abundance appears to be high, and the impact
parameter analysis indicates that the inferred value can be higher
than 0.15; however, this is probably caused by the tendency for lines
to cluster by the edge of the grid -- in this case the lower limit for
\logg\ which is closely related to the Eddington limit. The best
estimate of $y$ is approximately 0.12. At this value, only a few of
the plots for the \hei\ lines have reached the edge of the grid and
the intersection between the \hi\ and \heii\ lines is centered between
the \hei\ lines. For this helium abundance, the parameters predicted
by the models are \teff~=~35~kK and \logg~=~3.4.

The model spectrum reproduces the observations well. Most of the \hei\
lines fit, with only \hei~$\lambda$4026\AA\ and $\lambda$4471\AA\
being predicted too strong and too weak respectively.  It is not
unusual for the triplet lines not to fit well, and $\lambda$4471\AA\
is almost always predicted too weak.

The \heii\ lines all fit resonably well. \heii~$\lambda$4200\AA\ fits
poorly in the blue wing. The $\lambda$4541\AA\ model appears too
strong. However, continuum placement is difficult with such broad
lines, and $\lambda$4541\AA\ is in a very `crowded' part of the
spectrum. \heii~$\lambda$4686\AA\ could possibly show some signs of
wind contamination, although it also possible that, like
$\lambda$4541\AA\, the rectification could be imperfect.

The \hi\ lines fit very well, although they are all badly blended at
these high line-broadening velocities.

\subsection{HD 30614}  %%%%%%%%%%%%%%
\label{analysis:hd030614}

\ybcite{wa:76} classified $\alpha$~Camelopardalis (HD~30614) as O9.5~Ia, and
included it in his list of stars with morphologically normal CNO spectra.  
However,
\ybcite{bi:82} proposed that it might be an ON star. From the line
broadening analysis, we have adopted a macroturbulent broadening
velocity of 90~\kms\ for this star, a value which falls more or less
in the middle of the distribution of \vsini\ values published by other
authors (see table \ref{ta:sample_vsini}). The spectrum is typical of
a luminous star, with strong H$\alpha$ emission;
C\scr{3}~$\lambda$5696\AA\ and the unidentified lines at
$\lambda\lambda$4486, 4504\AA\ are also in emission. The 4630 -- 4640
-- 4650\AA\ CNO complex appears normal with no indication of unusually
strong nitrogen lines.

The fit diagram clusters very tightly with only
\hei~$\lambda\lambda$4471, 4713\AA\ and \heii~$\lambda$4686\AA\ giving
discrepant results. \heii~$\lambda$4686\AA\ is probably contaminated
by emission from the wind. The fit diagrams indicate a moderately high
helium fraction of 0.13. The physical parameters predicted by the
models for this value of $y$ are \teff~=~33~kK and \logg~=~3.1. The
resulting model spectra fit the observations well for most
lines. \hei~$\lambda$4471\AA\ is predicted too weak, and
\heii~$\lambda$4686\AA\ is predicted too strong, in agreement with the
assumption that it is contaminated by wind
emission. \heii~$\lambda$4200\AA\ appears to be predicted too weak,
but this is probably due to the blend with N\scr{3}.

\subsection{HD 34078}  %%%%%%%%%%%%%%
\label{analysis:hd034078}

HD~34078 (AE~Aurigae), is an O9.5~V star (\cite{wa:73}) with a
spectrum that appears very similar to that of 10~Lac (see chapter
\ref{ch:casestudy}). It has a rotational broadening velocity similar
to that of 10~Lac (30~\kms) and consequently the problems with
measuring and fitting the lines for 10~Lac discussed in chapter
\ref{ch:casestudy} apply to AE~Aur as well. 

The fit diagram shows even a bit more scatter in the predicted
parameters than those for 10~Lac, although the \hi\ lines are all in
fairly good agreement. The maximum deviation in \logg\ predicted by
the \hi\ lines is about $\pm 0.15$, and the spread in \teff\ from the
\hei\ lines at the adopted value of \logg~=~4.2 is about 6~kK. This is
greatly reduced by the \heii\ lines, where the spread is only
3~kK. \heii~$\lambda$4200\AA\ predicts a slightly higher temperature
than the other \heii\ lines. This is probably because the equivalent
width has been overestimated as a result of the blend with
N\scr{3}. Figure \ref{fig:ew_errors:he2} shows that an increase in
equivalent width gives a higher temperature for the \heii\ lines, so
the error on the temperature estimate caused by the spread of the
lines is probably less than $\pm 1$~kK.

For a helium fraction of $y$~=~0.09 we find \teff~=~37~kK and
\logg~=~4.2. For these values the model spectrum fits the observed
spectrum reasonably well, and matches the observed equivalent widths
more or less within the errors on the measurements, which are probably
similar to those of 10~Lac (see chapter \ref{ch:casestudy}).

For the \hei\ lines, most of the cores of the lines appear to
weak. This could be due to a too high value for the rotational
broadening velocity, or possibly due to use of a wrong broadening
function. The model reproduces the \heii\ line profiles well, with the
exception of \heii~$\lambda$4686\AA\ for which, like in 10~Lac, the
model profile appears too narrow. The \hi\ lines also have the same
problems as in 10~Lac, where the model profile fits the wings very
well, but the cores are too weak.

%\note{Voels et al. 1989, p1082, refers to Herrero 1987: consistently
%poor fits at the centres of hydrogen lines for stars with low
%rotational veolicty, arise because pressure broadening is ignored in
%the statsitcal equilibrium calculations. I presume this is the case
%for our models as well?}

\subsection{HD 36486}  %%%%%%%%%%%%%%
\label{analysis:hd036486}
$\delta$ Orionis (HD~36486), one of the Orion Belt stars, is
classified as O9.5 II by \ybcite{wa:76} with a note that it shows
signs of moderate morphological nitrogen deficiency. It has quite a
high macroturbulent broadening velocity of 126~\kms.

The fit diagrams show an unusually large spread in the plots of the
\hi\ lines. In particular, H10~$\lambda$3797\AA\ predicts a far higher
value of the surface gravity than the rest of the \hi\ lines. The fit
diagrams suggest a helium fraction of 0.10, with \teff~=~34~kK and
\logg~=~3.4. The model spectrum for these parameters fits well, with
the notable exception of the \hi\ lines. The red wing of these lines
all have a very broad shape, with the wings far weaker than what is
predicted by the models. This is probably also the reason why
H10~$\lambda$3797\AA, the line with the strongest blend in the blue
wing, predicts a very different surface gravity than the rest of the
\hi\ lines. The blend makes the area that is suitable for fitting very
small. It also appears that the red-wing asymmetry also could be
present in some of the \hei\ lines. The \heii~$\lambda$4200\AA\ line
fits far better than what is observed for most of the other stars. If
we assume that this line normally appears to be stronger than the
model predictions because of the nitrogen blends, then this supports
Walborn's (\citeyear{wa:76}) classification of this star as nitrogen
deficient.
% Could asymmetry be caused by binary contamination?


\subsection{HD 37742}  %%%%%%%%%%%%%%
\label{analysis:hd037742}
$\zeta$ Orionis (HD~37742) is the second Orion Belt star of the
sample. The third, $\epsilon$~Orionis (HD~37128), is a B0~Ia
star, and falls outside the scope of this thesis. $\zeta$~Ori has,
like $\delta$~Ori, relatively broad lines, and we adopt a
macroturbulent velocity of 109~\kms\ for this star. This star is
classified O9.7~Ib, and it and the other Orion Belt stars were
classified and discussed in \ybcite{wa:76}. Walborn also noted that
all three stars were moderately nitrogen deficient.

The fit diagram is well confined, with only \heii~$\lambda$4686\AA\
diverging from the other lines. The closest clustering of the plotted
lines is found for $y$~=~0.10, with \teff~=~34~kK and \logg~=~3.3. The
model spectra fit well. Some asymmetries like in $\delta$~Ori are 
possibly present in the \hi\ lines, although they do not appear as
strong. \hei~$\lambda$4713\AA\ appears to have a `bump' in the red
wing, which is also possibly present in some of the other \hei\
lines. This star shows very little absorption in
\heii~$\lambda$4200\AA\ from the blend with N\scr{3}, consistent
with Walborn's N-deficient classification.


\subsection{HD 188209}  %%%%%%%%%%%%%%
\label{analysis:hd188209}

HD~188209 was classified O9.5 Iab by \ybcite{wa:76}, and we have
adopted a macroturbulent broadening velocity of 65~\kms. The fit diagrams
for this star are very tight, and the impact parameter solution is at
the point where the \heii\ lines turn over from being temperature to
gravity sensitive. The adopted solution for $y$~=~0.14 is
\teff~=~33.0~kK and \logg~=~3.1. The model spectrum fits the
observations very well, with only a very few
exceptions. \hei~$\lambda$4471\AA\ is, as usual, predicted too weak,
\heii~$\lambda$4200\AA\ shows signs of blending and
\heii~$\lambda$4686\AA\ is probably contaminated by
emission. H$\alpha$ is not in emission (above the continuum), but it
is almost entirely filled in. The model \hi\ lines fit the
observations well, with the exception of \hi~$\lambda$3935\AA\ which
has broader wings than predicted by the models. Usually this would be
indicative of a higher surface gravity, but in this case could be
caused by blends.

\subsection{HD 189957}  %%%%%%%%%%%%%%
\label{analysis:hd189957}
This star is an O9.5~III (\cite{wa:73}), and the line profiles show
very clearly the macroturbulent v-shape. We derived a
macroturbulent broadening velocity of 85~\kms.
% Check that v-shape is discussed somewhere

The fit diagrams show some scatter, especially H9~\lam3835\AA\
which predicts a higher \logg\ than the other \hi\ lines. The model profiles
confirm this, showing that the wings of this line are predicted too
narrow. The impact parameter analysis suggests that $y$~=~0.11 with
\teff~=~33~kK and \logg~=~3.5 is the best solution for this star. The
model profiles are not very good, however. All the \heii\ lines are
predicted too weak, something that usually suggests that the
temperature is estimated slightly too low. The equivalent widths match
well, so it could be a matter of using a slightly too high
line-broadening velocity or even that the data rectification is
slightly off.


\subsection{HD 191781}  %%%%%%%%%%%%%%
\label{analysis:hd191781}
Classified ON9.7~Iab by \ybcite{wa:76}, this star shows all the signs
of a `typical' ON star. N\scr{3}~$\lambda$4640\AA\ is clearly much
stronger than the C blends at 4650\AA\ (see figure \ref{fig:on_oc}). We
find a macroturbulent broadening of 89~\kms\ for this star.

The fit diagrams are very tight with all lines in good agreement, with
the exception of the \hei\ triplet lines and the blended
$\lambda$4009\AA\ line. We find a high helium fraction of $y$~=~0.16
for this star, with \teff=~31~kK and \logg~=~3.1. The model profiles
fit the observations well, and, after carefully using \elf\ to fit and
remove the blends from \hei~$\lambda$4009\AA, this line also
fits. \hei~$\lambda$4143\AA\ is, as usual, modelled too weak. For
\heii~$\lambda$4200\AA\ the model appears too weak, but this line is
blended with a N\scr{3} line, which we can expect to be quite strong
in this star. The \hi\ lines all appear weaker than the model
profiles. This star has clear emission in H$\alpha$ and the
discrepancies between the model and observations are probably due to
wind effects in these lines.

\subsection{HD 194280}  %%%%%%%%%%%%%%
\label{analysis:hd194280}
HD~194280 is the only OC star in the sample. It was classified
OC9.7~Iab by \ybcite{wa:76}. Comparing the spectra of HD~191781 and
HD~194280 (figure \ref{fig:on_oc}) we can see some clear differences.
The N\scr{3} lines at \lam\lam4634.2, 4640.6 and 4641.9\AA\ (and
possibly also the N\scr{2} \lam4630.5\AA\ line) are much stronger in
the ON star HD~191781, whereas the C\scr{3} lines at \lam\lam4647.4
and 4650.3\AA\ are much stronger in the OC star. Most of the other
lines appear to be of similar strenght in the two spectra. The spectra
of these two stars in the range 3800 -- 6200\AA\ are included in
appendix \ref{app:ocn}.

\begin{figure}
\epsfxsize=15cm
\epsfbox{fig_analyses/on_oc/on_oc.eps}
\caption[The spectra of the ON star HD~191781 and the OC star
HD~194280]{\fcfont The \lam4650\AA\ spectra of the ON star HD~191781 and the OC
star HD~194280}
\label{fig:on_oc}
\end{figure}

The fit diagrams for this star show a well defined solution for $y$=0.09 with
\teff~=~32~kK and \logg~=~3.2.


\subsection{HD 195592}  %%%%%%%%%%%%%%
\label{analysis:hd195592}
This star is undoubtedly  very luminous. The spectrum shows a
large number of emission lines, including
Si\scr{3}~$\lambda$3806.5\AA, the unidentified lines at
$\lambda\lambda$4486, 4504\AA, C\scr{3}~$\lambda$5695.9\AA, and of
course several of the \hi\ lines, including H$\alpha$. There also
appear to be several other weak emission lines, some of which are
shown in figure \ref{fig:hd195592_emission}.

\begin{figure} %%%%% FIGURE
\epsfxsize=15cm
\epsfbox{fig_analyses/hd195592/hd195592.eps}
\caption[Emission lines in HD~195592]
{\fcfont Some lines in the optical spectrum of HD~195592. H$\alpha$ has been
plotted on a different scale than the other lines. To obtain good S/N
we had to saturate this line (see also fig \ref{fig:splif_wide}).}
\label{fig:hd195592_emission}
\end{figure} %%%%%%

This star has relatively narrow lines (see also table
\ref{ta:elf_vs_ew} in chapter \ref{ch:casestudy}), and 
we have adopted a macroturbulent broadening profile of 54~\kms. The
fit diagrams show a very tight clustering of the lines, and the impact
parameter analysis yields a best estimate at $y$~=~0.12 with
\teff~=~31~kK and \logg~=~3.0. The model profiles fit well, but the
\hi\ lines show clear effects of emission filling in the cores, and
\heii~$\lambda$4686\AA\ is not visible as an absorption feature (there
is some slight emission at the centre of the line).

\subsection{HD 201345}  %%%%%%%%%%%%%%
\label{analysis:hd201345}

This is the second ON main-sequence star in the sample. It was
classified as ON9~V by \ybcite{wa:76}, and shows the same
characteristic strong nitrogen and weak carbon and oxygen lines as
HD~12323.  It is a relatively fast rotator, with a \vsini\ of
109~\kms.

The fit diagrams show quite large scatter for the lines, possibly due
to errors introduced into the measurements of the equivalent widths
because of the fast rotation (like unidentified blends etc.). The
impact parameter analysis suggests a helium fraction $y$~=~0.14, with
\teff~=~36~kK and \logg~=~3.9. For these parameters, the model
profiles fit well, with a few exceptions. As usual, the 
\hei~$\lambda$4471\AA\ model line is too weak, and so is, very unusually,
\heii~$\lambda$4686\AA. This line is only modelled too weak for the
main-sequence stars, and it is possible this is due to shortcomings in
the models.


\subsection{HD 202124}  %%%%%%%%%%%%%%
\label{analysis:hd202124}
For HD~202124, classified O9.5~Iab by \ybcite{wa:76}, we have adopted
a macroturbulent broadening velocity of 93~\kms. The lines on the fit
diagrams are all in good agreement, apart from
\heii~$\lambda$4686\AA. However, this star shows emission in H$\alpha$
and it is thus likely that also \heii~$\lambda$4686\AA\ will be
contaminated by emission. Indeed, by comparing with models for our
adopted parameters of $y$~=~0.13, \teff~=~34~kK and \logg~=~3.2, it is
clear that this line is clearly filled-in. All other lines fit well,
with the usual exception of \hei~$\lambda$4471\AA. The \hi\ line cores
are all predicted too strong, probably due to wind contamination in
these lines.


\subsection{HD 207198}  %%%%%%%%%%%%%%
\label{analysis:hd207198}
HD~207198 (O9~Ib--II; \cite{wa:76}) has an interesting spectrum,
showing emission in several lines, including the unidentified doublet
at $\lambda\lambda$4486, 4505\AA\ and
C\scr{3}~$\lambda$5695.9\AA. There also seems to be some emission in
the 4630 -- 4640\AA\ region. We find \teff~=~36~kK and
\logg~=~3.4 for $y$~=~0.12. Comparing the model with the observed
spectrum of this star, it looks like there is a wavelength shift in
\heii~$\lambda$4686\AA\ between the model and the observations;
however it is more likely that it just appears that way due to wind
contamination in this line (figure \ref{fig:hd207198}).

\begin{figure} %%%%% FIGURE
\epsfxsize=\figwidth
\setlength{\cen}{(\textwidth / 2) - (\epsfxsize / 2)}
\hspace{\cen}\epsfbox{fig_analyses/hd207198/hd207198.eps}
\caption[Unusual wavelength shift of \heii~\lam4686\AA\ in HD~207198]
{\fcfont The \heii\ lines with model fits for HD~207198. The unusual
apparent wavelength shift of \heii~\lam4686\AA\ is probably caused by
wind contamination (see text).}
\label{fig:hd207198}
\end{figure} %%%%%%

\subsection{HD 209975}  %%%%%%%%%%%%%%
\label{analysis:hd209975}
HD~209975 was classified O9.5~Ib by \ybcite{wa:76}. It has a
macroturbulent broadening velocity of 69~\kms, and from the fit
diagrams appears to have a solar helium abundance. The fit diagrams are
well confined, and most lines are in good agreement.
\heii~$\lambda$4686\AA\ predicts a temperature a few thousand Kelvin
colder than the other \heii\ lines. The model spectrum for $y$~=~0.09,
\teff~=~35~kK and \logg~=~3.4 fits well, with only
\hei~$\lambda$4471\AA\ and \heii~$\lambda$4686\AA\ predicted too weak
and too strong respectively. Some of the \hi\ lines show signs of weak
filling-in in the red wing.

\subsection{HD 210809}  %%%%%%%%%%%%%%
\label{analysis:hd210809}

The fit diagrams for this star show a small amount of scatter. Most of
the \hei\ singlets are in good agreement, with the exception of
\hei~$\lambda$4009\AA\ which predicts a slightly lower
temperature/higher surface gravity. This is most likely due to an
overestimation of this line's equivalent width resulting from a
blend. \heii~$\lambda$4686\AA\ fits at a temperature several thousand
K colder than the other \heii\ lines, which is not unexpected for a
O9~Iab star. The spectrum shows clear emission in H$\alpha$ and it is
likely that \heii~$\lambda$4686\AA\ is also contaminated by stellar
wind emission. The impact parameter suggests a helium fraction of
0.13, but the results are inconclusive for the temperature and
gravity. With the inclusion of \heii~$\lambda$4686\AA\ we get
\teff~=~35~kK and
\logg~=~3.4. The model spectrum for these parameters fits reasonably
well, but the \heii\ lines are all predicted too weak. By ignoring
$\lambda$4686\AA\ in the analysis, we find a temperature of
\teff~=~36~kK and \logg~=~3.3. This solves the problem with the weak
\heii\ lines and the overall fit is much better, althought the adopted
uncertainties encompass both solutions. 
% (if that is true)

\subsection{HD 214680}  %%%%%%%%%%%%%%
\label{analysis:hd214680}

This is 10~Lac, and the analysis of this star is discussed in detail
in chapter \ref{ch:casestudy}.

\subsection{HD 218195}  %%%%%%%%%%%%%%
\label{analysis:hd218195} 

This star was classified O9~III by \ybcite{wa:76}, and he also noted
that the spectrum, although not an ON spectrum, shows signs of
moderate nitrogen enhancement.

The fit diagram for this star is fairly well defined, and the impact
parameter analysis suggests a helium number fraction, $y$~=~0.12 with
\teff~=~37~kK and \logg~=~3.6. The model spectrum fits the
observations well, with the usual exception of
\hei~$\lambda$4471\AA. The N\scr{3} blend in \heii~$\lambda$4200\AA\
appears strong, in agreement with Walborn's classification of the star
as nitrogen enhanced. \heii~$\lambda$4686\AA\ shows no signs of
filling-in, but appears to have some extra absorption in the red wing.

\subsection{HD 218915}  %%%%%%%%%%%%%%
\label{analysis:hd218915}

HD~218195 was classified O9.5~Iab by \ybcite{wa:76}. The fit diagram
appears tight, but shows quite a large scatter in the \heii\ lines. We
have adopted $y$~=~0.10, \teff~=~34.0 and \logg~=~3.2 as the
parameters for this stars. The model spectrum fits the observations
well, although \heii~$\lambda$4686\AA\ show signs of filling-in. 

\subsection{HD 225160}  %%%%%%%%%%%%%%
\label{analysis:hd225160}

HD~225160 is the only O8 star in the sample (O8 Ib(f) --
\cite{wa:76}). The fit diagram is fairly tight, but the weak \hei\
lines are in disagreement with the other lines. Both
\hei~$\lambda$4009\AA\ and $\lambda$5047\AA\ predict lower \teff\
(models underestimate the equivalent width) than the rest of the
\hei\ lines (with the exception of \hei\ \lam4471\AA which
consistently is predicted too weak). This star has very broad lines
and we have adopted a macroturbulent broadening velocity of 109~\kms.

The model spectrum for the adopted parameters of $y$~=~0.15,
\teff~=~38~kK and \logg~=~3.4 appears to fit well with the exception of
\heii~$\lambda$4686\AA\ which is completely filled in. All the \hi\
lines also show clear signs of filling-in, and the star has several
strong emission lines. There also appears to be some emission in the
4630~--~4640\AA\ region, possibly due to N\scr{3}.

\subsection{BD +36 4063}  %%%%%%%%%%%%%%
\label{analysis:bd+364063}

This star is the only star in the sample that was not originally
classified by Walborn. It was first classified as ON by \ybcite{ma:89}
(ON9.7~I), and appears similar to that of HD~191781 only with even
weaker carbon lines. This star is also a binary (Howarth, private
communication), and unfortunately the spectrum is of quite low S/N
compared to the other stars in the sample because the star is much
fainter. However, the fit diagrams appear quite well confined,
although the \heii\ and \hi\ lines show slightly larger scatter than
the norm.

The model spectrum for $y$~=~0.11, \teff~=~32~kK and
\logg~=~3.1 fits the observations fairly well, although there appears
to be some absorption in the red wing of most lines. \hi\ lines fit
poorly, although this could be due to wind contamination. This
star has strong H$\alpha$ emission, and \heii~$\lambda$4686\AA\
is too weak in the model spectrum.

%%%%%%%%%%%%%%%%%%%%%%%%%%%%%%%%%%%%%%%%%%%%%%%%%%%%%%%%
%%%%%%%%%%%%%%% SECTION = Microturbulence %%%%%%%%%%%
%%%%%%%%%%%%%%%%%%%%%%%%%%%%%%%%%%%%%%%%%%%%%%%%%%%%%%%%
\section{Microturbulence}
\label{sec:analyses:mturb}

In chapter \ref{ch:mod} we saw how the inclusion of microturbulence in
the line formation calculations increases the equivalent width of most
\hei\ lines, especially at low surface gravities. The effects of this on the
fit diagrams can clearly been seen in figure \ref{fig:mturb}. The
strengthening of the \hei\ lines mean that we can get consistent fits
at lower $y$ than we would for zero microturbulence. This has another
interesting effect. The \heii\ lines are relatively unaffected by
microturbulence in our grid (in more recent calculations by
\cite{sm:98}, this is not necessarily the case). Since we have to
lower the helium abundance to match equivalent widths for the \hei\
lines, we have to increase the temperature to get consistent fits for
the \heii\ lines. From figure \ref{fig:mturb} (and also appendix
\ref{app:osumm}) it appears that the temperatures are about 1~--~2~kK
higher%
\footnote{All temperatures quoted previously in this chapter \emph{do}
include microturbulence unless otherwise stated}
%
when including a microturbulence of 15~\kms.

\begin{sidewaysfigure} %%%%%%%%%% FIGURE
\epsfysize=13cm
\epsfbox{fig_analyses/hd218915_4ew.eps}
\caption[Fit diagrams showing dependence on microturbulence]
{\fcfont Series of fit diagrams for HD~218915 (O9.5 Iab) showing
dependence on microturbulence, $\xi$, and helium abundance,
$y$. Microturbulence increases from left to right (0, 5, 10, 15~\kms)
and helium abundance from top to bottom (0.08, 0.10, 0.12).}
\label{fig:mturb}
\end{sidewaysfigure} %%%%%%%%%% END FIGURE

The microturbulence is also relatively more important for the \hei\
triplet lines \lam\lam4471 and 4713\AA\ than the other \hei\
lines. These two lines are typically predicted too weak in standard
analyses not taking microturbulence into account
(cf. the \emph{generalized dilution effect}, see section \ref{sec:ews}).
Including a microturbulence of 15~\kms\ does increase the strength of
this line, but not enough to fit the observations. However,
\ybcite{sm:98} have shown that this line is
strengthened further by treating microturbulence in both the
statistical equilibrium and line formation calculations, and that this
could be enough to solve the discrepancy.

The results of the impact-parameter analyses described earlier
favour a microturbulent velocity, $\xi$, of 15~\kms\ for all stars
except HD~207198 (O9 Ib-II; $\xi = 0$~\kms) and HD~218195 (O9 III;
$\xi = 10$~\kms). However, in both of these cases this is probably caused by a
slight `defect' in the way we perform the impact-parameter
calculations. As mentioned previously in this chapter, outliers are
rejected before re-calculating the impact parameters. For both of these
stars, more lines were included in the $\xi = 15$~\kms\ fits, and
these probably do represent the most consistent fits. For these
reasons we have chosen to adopt a microturbulent velocity of 15~\kms\
for all the stars in the sample, including the main-sequence stars,
although microturbulence is relatively unimportant for these stars
(see also chapter \ref{ch:mod}).


%%%%%%%%%%%%%%%%%%%%%%%%%%%%%%%%%%%%%%%%%%%%%%%%%%%%%%%%
%%%%%%%%%%%%%%% SECTION = Summary of results %%%%%%%%%%%
%%%%%%%%%%%%%%%%%%%%%%%%%%%%%%%%%%%%%%%%%%%%%%%%%%%%%%%%
\section{Physical parameters of the O stars}

The adopted parameters for the WHT sample stars are summarised in
table \ref{ta:ostars}, and the observed and modelled equivalent widths
(and their ratios) are shown in table \ref{ta:ostars_ews}. There are
two trends of particular interest in this table:

\begin{itemize}
\item	The \hei\ line \lam4009\AA\ is consistently predicted too
	weak. The median ratio of $W_\mathrm{obs}/W_\mathrm{mod}$ for
	this line is over 1.4. This is probably due to some
	unidentified blending.

\item	The median of $W_\mathrm{obs}/W_\mathrm{mod}$ for the \hei\
	lines in each star is almost consistently larger than 1. This
	can probably be attributed to the impact-parameter analysis
	introducing a systematic bias. The effect seems to be
	particularly strong in stars where the fit-diagram
	intersection zone is close to the point where the \heii\ lines
	go from being temperature sensitive to gravity sensitive
	(e.g. HD~188209, HD~210809 and HD~218915). It is possible that
	this `pulls' the solution towards a slightly lower gravity
	causing an underestimation of the equivalent width of the
	\hei\ lines. However this effect is likely to be small, and
	within the uncertainty of the results.
\end{itemize}

For HD~12323 the ratio of the \hei\ lines to the models seem to be
particularly large; the median value is 1.153. This star has a helium
abundance which is very high, and the lines show a large scatter on
the fit-diagrams. It is possible that we have slightly underestimated
the helium abundance ($y = 0.17$), but increasing the helium abundance
results in a larger scatter of the lines on the fit-diagrams. The
solution for this star lies very close to the high-gravity limit of
our grid of models, and it is therefore possible that unwanted
`edge-effects' caused by interpolating the models causes a higher
uncertainty for all determined parameters for this star (see below).


\subsection{Uncertainties}
\label{sec:analyses:err}

It is possible to obtain an estimate of the uncertainties on the
results from the impact-parameter analyses. When finding the
point of closest approach to all the measured lines, the scatter of
the lines will be related to the internal error on our measurements
and models (not taking into account systematic effects caused by
missing physics in the models etc.). Typically the RMS scatter of the
lines is of the order 0.5~--~1.0~kK in effective temperature, and
0.05~--~0.09~dex in surface gravity. However, these values are only
valid if we only fit on 2-D grid, i.e. have fixed values of the helium
abundance and microturbulence. As discussed above, it appears that all
the stars show microturbulent velocities of the order 15~\kms, and
that neglecting microturbulence altogether will only lower the
temperatures by 1~--~2~kK. This probably means that the uncertainty on
the choice of microturbulence is likely to have only a negligible effect
on the temperatures and gravities. The helium abundance, however, will
probably be more important. The analyses vary the helium abundance
from subsolar $y = 0.08$ to about $y = 0.20$ in steps of 0.01. It
appears that the impact-parameter analysis is able to distinguish the
best-fit model from its `neighbours', so the precision of the
measured helium abundance will be about $\pm 0.01$. The implied uncertainty
on temperatures and gravities (from impact-parameter solution on the
neighbouring $y$ fit-diagrams) appear to be of the order 0.5~kK in
\teff\ and 0.05 dex for \logg. 

Within the context of the adopted models, good estimates of the
uncertainties on the four fitted parameters are then $\pm 1$~kK and
0.1~dex for \teff\ and \logg\ respectively, $\pm 0.01$ for the helium
number fraction, and $\pm 5$~\kms\ for the microturbulence. Of course,
the results may be wrong by much larger amounts. This will not be
clear until dynamical, spherically extended, line-blanketed nLTE
models are available for comparison (but see section \ref{sec:comp}).

\subsection{Masses and radii}

Using absolute visual magnitudes, $M_V$, from Garmany's unpublished
catalogue of O stars,%
\footnote{For HD~34078 Garmany gives $M_V = -5.0$ (probably from
\cite{hu:78}). However \acite{hu:78} based this number on the
assumption that HD~34078 belongs to the Ori OB1 association. According
to \ybcite{gi:87} this might not be the case. Like
\ybcite{he:92}, we therefore adopt a value of $M_V = -4.0$, which approximately
corresponds to a main-sequence star of this spectral type
(\cite{co:83}).}
%
(cf. \cite{ga:imf}) we can calculate radii and spectroscopic masses
for the sample. We can find the bolometric correction using:
%
\begin{equation}
\mathrm{BC} = K - 0.08(\teff/\mathrm{kK})
\end{equation}
%
where the constant $K$ is weakly dependent on surface gravity. We
adopt the values from \ybcite{ho:winds} with $K = -0.6$ for supergiants,
and $K = -0.5$ for all other luminosity classes. We can now calculate
luminosities and radii from:
%
\begin{equation}
\log(L_\star / L_\odot) = -0.4 (M_\mathrm{BOL} - 4.75)
\end{equation}
%
and
%
\begin{equation}
\log(R_\star / R_\odot) = 0.5 \log(L_\star / L_\odot) - 2 \log(\teff/T_\odot) 
\end{equation}
%
These derived parameters are listed in table \ref{ta:ostars:derived}
with internal error estimates using the uncertainties discussed in
section \ref{sec:analyses:err} and $\pm$~0.3 for $M_V$ (see \cite{he:92}).


%%%%%%%%%%%%%%%%%%%%%%%%%%%%%%%%%%%%%%%%%%%%%%%%%%%%%%%%
%%%%%%%%%%%%%%% SECTION = Other Sources  %%%%%%%%%%%
%%%%%%%%%%%%%%%%%%%%%%%%%%%%%%%%%%%%%%%%%%%%%%%%%%%%%%%%
\section{Comparisons with published work}
\label{sec:comp}

Plotting the effective temperatures against `typical' temperatures
from \ybcite{ho:winds} reveals a systematic difference (figure
\ref{fig:teffplot}). The temperatures are clearly seen to increase
with spectral type, and the slope of the relation appears to be the
similar for our adopted values and those listed in Howarth and Prinja,
however our temperatures come out systematically higher by about
2~--~5~kK for all luminosity classes.

The same plot for surface gravities (fig \ref{fig:loggplot}), reveals
a much closer correlation between the values in Howarth and Prinja and
those adopted here, although for the giants and supergiants there
might be a slight tendency for our adopted gravities to be slightly
lower than those in Howarth and Prinja.

Tables \ref{ta:comp:voels} -- \ref{ta:comp:herrero} compare our
results to those of other authors. The results compare well within the
uncertainties given above, except with the results of \ybcite{gr:92}
shown in table \ref{ta:comp:grigsby}. \acite{gr:92} consistently
predict effective temperatures 6~--~8~kK lower than our adopted
values. This discrepancy was also noted earlier for 10~Lac, and is
discussed in section \ref{sec:casestudy:discussion}.

There appears to be a trend for our results to systematically come out
slightly (about 2~kK) higher than those adopted by other authors. Most
likely this is due to the inclusion of microturbulence in our study
and/or the use of an objective impact parameter routine to find the
best-fit solution. As discussed in section \ref{sec:analyses:mturb},
using microturbulent velocities of about 15~\kms\ will increase the
adopted effective temperatures with 1~--~2~kK, which is comparable to
the systematic differences seen in tables \ref{ta:comp:voels} and
\ref{ta:comp:herrero}, and could also partly explain the disagreement
with \acite{ho:winds}. The surface gravities and helium abundances
generally agree well.

\begin{table} %%%%%%% TABLE %%%%%%%%%%%%%%%
\begin{center}
\caption{Observed parameters of the WHT O stars}
\vspace{\abovecaptionskip}
\begin{tabular}{lllrccccc}
\hline\hline
HD/BD & \multicolumn{1}{c}{Name} & \multicolumn{2}{c}{Spectral Type} &
\teff (kK) & \logg & $y$ & L. broadening & $v$ \\
\hline
10125 	&		& O9.7 II      	& W76	& 32.5	& 3.3	& 0.15	& mac & 132 \\
12323 	&		& ON9 V        	& W76 	& 36.0	& 4.1	& 0.17	& rot & 131 \\
13745 	&		& O9.7 II((n))	& W76	& 33.0	& 3.3	& 0.20	& rot & 176 \\
16429 	&		& O9.5 II((n))	& W76	& 35.0	& 3.4	& 0.12	& rot & 216 \\
30614 	& $\alpha$ Cam	& O9.5 Ia  	& W76	& 33.0	& 3.1	& 0.13	& mac & 90  \\
34078	& AE Aur	& O9.5 V	& W73	& 37.0	& 4.2	& 0.09	& rot & 30  \\
36486 	& $\delta$ Ori	& O9.5 II	& W76	& 34.0	& 3.4	& 0.10	& mac & 126 \\
37742	& $\zeta$ Ori	& O9.7 Ib	& W76 	& 34.0	& 3.3	& 0.10	& mac & 109 \\
188209	& 		& O9.5 Iab	& W76	& 33.0	& 3.1	& 0.14	& mac & 65 \\	
189957	& 		& O9.5 III	& W73	& 33.0	& 3.5	& 0.11	& mac & 85 \\
191781	& 		& ON9.7 Iab	& W76	& 31.0	& 3.1	& 0.16	& mac & 89 \\
194280	& 		& OC9.7 Iab	& W76 	& 32.0	& 3.2	& 0.09	& mac & 101 \\
195592	& 		& O9.7 Ia	& W76	& 31.0	& 3.0	& 0.12	& mac & 54 \\
201345	& 		& ON9 V		& W76	& 36.0	& 3.9	& 0.14	& rot & 109 \\
202124	& 		& O9.5 Iab	& W76	& 34.0	& 3.2	& 0.13	& mac & 93 \\
207198	& 		& O9 Ib-II	& W76	& 36.0	& 3.4	& 0.12	& mac & 67 \\
209975	& 19 Cep	& O9.5 Ib	& W76	& 35.0	& 3.4	& 0.09	& mac & 69 \\
210809	& 		& O9 Iab	& W76	& 36.0	& 3.3	& 0.13	& mac & 89 \\
214680	&10 Lac		& O9 V		& W73	& 38.0	& 4.2	& 0.09	& rot & 30 \\
218195	& 		& O9 III	& W76	& 37.0	& 3.6	& 0.12	& mac & 59 \\
218915 	& 		& O9.5 Iab	& W76	& 34.0	& 3.2	& 0.10	& mac & 68 \\
225160	& 		& O8 Ib(f)	& W71	& 38.0	& 3.4	& 0.15	& mac & 109 \\
+36 4063& 		& ON9.7 I	& M89	& 32.0	& 3.1	& 0.11	& mac & 84 \\
\hline\hline
\end{tabular}
\label{ta:ostars}
\end{center}
\end{table} %%%%%%%%%%%%%%%%%%%%%%%%%%%%%%%%%

%%%%%%%%%%%%%%%%%% SIDEWAYS LONG TABLE %%%%%%%%%%%%%%%
\newcommand{\tabhead}{
\multicolumn{1}{c}{HD/BD} & \multicolumn{1}{c}{} & 
\multicolumn{8}{c}{\hei\ equivalent widths (m\AA)} &
\multicolumn{4}{c}{\heii\ equivalent widths (m\AA)} &
\multicolumn{2}{c}{Median} \\
\multicolumn{2}{c}{} & \lam4009\AA & \lam4026\AA & \lam4143\AA & 
\lam4388\AA & \lam4471\AA & \lam4713\AA & \lam4922\AA &
\multicolumn{1}{c}{\lam5047\AA} &
\lam4200\AA & \lam4541\AA & \lam4686\AA &
\multicolumn{1}{c}{\lam5411\AA} & \hei & \heii \\
}
\begin{sidewaystable}
\scriptsize
\begin{center}
\caption{Measured and calculated equivalent widths}
\vspace{\abovecaptionskip}
\begin{tabular}{lc|cccccccc|cccc|cc}
\hline\hline
\tabhead
\hline
     10125 & obs & 268.0 & 736.0 & 294.0 & 488.0 & 965.0 & 325.0 & 598.0 & 179.0 & 395.0 & 351.0 & 484.0 & 480.0 \\
           & mod & 241.0 & 817.3 & 302.5 & 489.7 & 742.9 & 318.6 & 606.9 & 189.8 & 336.8 & 389.0 & 510.6 & 436.8 \\
           & obs/mod & 1.112 & 0.901 & 0.972 & 0.997 & 1.299 & 1.020 & 0.985 & 0.943 & 1.173 & 0.902 & 0.948 & 1.099 & 0.991 & 1.023 \\ 
\hline
     12323 & obs & 390.0 & 1062.0 & 451.0 & 626.0 & 1043.0 & 306.0 & 836.0 & 233.0 & 532.0 & 592.0 & 842.0 & 730.0 \\
           & mod & 301.1 & 1004.7 & 383.3 & 553.9 & 999.8 & 333.8 & 707.5 & 188.4 & 497.5 & 603.7 & 826.7 & 684.8 \\
           & obs/mod & 1.295 & 1.057 & 1.177 & 1.130 & 1.043 & 0.917 & 1.182 & 1.237 & 1.069 & 0.981 & 1.019 & 1.066 & 1.153 & 1.042 \\ 
\hline
     13745 & obs & 310.0 & 842.0 & 322.0 & 545.0 & 1073.0 & 377.0 & 584.0 & 205.0 & 399.0 & 398.0 & 409.0 & 527.0 \\
           & mod & 266.5 & 883.7 & 326.9 & 525.7 & 804.5 & 336.3 & 648.4 & 209.6 & 410.3 & 469.6 & 582.8 & 531.5 \\
           & obs/mod & 1.163 & 0.953 & 0.985 & 1.037 & 1.334 & 1.121 & 0.901 & 0.978 & 0.972 & 0.848 & 0.702 & 0.992 & 1.011 & 0.910 \\ 
\hline
     16429 & obs & 259.0 & 677.0 & 330.0 & 453.0 & 822.0 & 257.0 & 497.0 & 60.0 & 480.0 & 490.0 & 542.0 & 630.0 \\
           & mod & 164.9 & 765.7 & 228.2 & 383.0 & 664.1 & 298.5 & 493.5 & 141.1 & 411.1 & 491.6 & 542.7 & 601.6 \\
           & obs/mod & 1.571 & 0.884 & 1.446 & 1.183 & 1.238 & 0.861 & 1.007 & 0.425 & 1.168 & 0.997 & 0.999 & 1.047 & 1.095 & 1.023 \\ 
\hline
     30614 & obs & 187.0 & 709.0 & 181.0 & 333.0 & 920.0 & 335.0 & 469.0 & 124.0 & 333.0 & 367.0 & 199.0 & 485.0 \\
           & mod & 147.5 & 669.1 & 208.2 & 351.0 & 635.7 & 291.9 & 466.9 & 136.7 & 287.6 & 351.3 & 364.1 & 473.4 \\
           & obs/mod & 1.268 & 1.060 & 0.869 & 0.949 & 1.447 & 1.148 & 1.004 & 0.907 & 1.158 & 1.045 & 0.547 & 1.025 & 1.032 & 1.035 \\ 
\hline
     34078 & obs & 256.0 & 823.0 & 275.0 & 488.0 & 680.0 & 246.0 & 477.0 & 133.0 & 540.0 & 438.0 & 705.0 & 660.0 \\
           & mod & 179.1 & 809.8 & 254.5 & 388.9 & 804.5 & 267.6 & 505.7 & 121.3 & 389.7 & 502.8 & 755.3 & 606.1 \\
           & obs/mod & 1.429 & 1.016 & 1.081 & 1.255 & 0.845 & 0.919 & 0.943 & 1.096 & 1.386 & 0.871 & 0.933 & 1.089 & 1.048 & 1.011 \\ 
\hline
     36486 & obs & 317.0 & 642.0 & 201.0 & 396.0 & 737.0 & 231.0 & 490.0 & 167.0 & 432.0 & 403.0 & 425.0 & 458.0 \\
           & mod & 168.6 & 726.4 & 229.7 & 379.5 & 645.5 & 285.4 & 491.4 & 139.2 & 334.8 & 401.2 & 499.6 & 478.8 \\
           & obs/mod & 1.880 & 0.884 & 0.875 & 1.043 & 1.142 & 0.809 & 0.997 & 1.200 & 1.290 & 1.004 & 0.851 & 0.957 & 1.020 & 0.981 \\ 
\hline
     37742 & obs & 195.0 & 746.0 & 171.0 & 371.0 & 834.0 & 251.0 & 499.0 & 129.0 & 366.0 & 350.0 & 263.0 & 551.0 \\
           & mod & 146.9 & 691.3 & 206.3 & 348.7 & 619.0 & 280.0 & 460.5 & 128.5 & 322.0 & 390.0 & 451.6 & 488.2 \\
           & obs/mod & 1.327 & 1.079 & 0.829 & 1.064 & 1.347 & 0.896 & 1.084 & 1.004 & 1.137 & 0.897 & 0.582 & 1.129 & 1.072 & 1.013 \\ 
\hline

\end{tabular}
\label{ta:ostars_ews}
\end{center}
\end{sidewaystable} 

\begin{sidewaystable} 
\scriptsize
\begin{center}
Table \ref{ta:ostars_ews}: \emph{continued}\\
\vspace{\abovecaptionskip}
\begin{tabular}{lc|cccccccc|cccc|cc}
\hline\hline
\tabhead
\hline
    188209 & obs & 225.0 & 749.0 & 250.0 & 406.0 & 855.0 & 330.0 & 550.0 & 147.0 & 307.0 & 362.0 & 360.0 & 487.0 \\
           & mod & 156.7 & 687.8 & 218.2 & 366.2 & 647.8 & 298.9 & 482.2 & 144.5 & 299.6 & 364.0 & 382.0 & 487.1 \\
           & obs/mod & 1.436 & 1.089 & 1.146 & 1.109 & 1.320 & 1.104 & 1.141 & 1.017 & 1.025 & 0.995 & 0.942 & 1.000 & 1.125 & 0.997 \\ 
\hline
    189957 & obs & 295.0 & 768.0 & 313.0 & 465.0 & 814.0 & 260.0 & 576.0 & 141.0 & 301.0 & 277.0 & 560.0 & 440.0 \\
           & mod & 224.9 & 788.8 & 287.2 & 452.3 & 741.2 & 297.0 & 570.3 & 165.5 & 303.1 & 356.7 & 507.9 & 392.1 \\
           & obs/mod & 1.312 & 0.974 & 1.090 & 1.028 & 1.098 & 0.875 & 1.010 & 0.852 & 0.993 & 0.777 & 1.103 & 1.122 & 1.019 & 1.048 \\ 
\hline
    191781 & obs & 365.0 & 732.0 & 291.0 & 483.0 & 850.0 & 352.0 & 586.0 & 232.0 & 263.0 & 264.0 & 389.0 & 339.0 \\
           & mod & 234.7 & 763.7 & 293.3 & 477.9 & 707.8 & 311.4 & 595.2 & 191.8 & 269.2 & 309.8 & 427.4 & 350.4 \\
           & obs/mod & 1.555 & 0.958 & 0.992 & 1.011 & 1.201 & 1.130 & 0.985 & 1.210 & 0.977 & 0.852 & 0.910 & 0.967 & 1.071 & 0.939 \\ 
\hline
    194280 & obs & 318.0 & 675.0 & 231.0 & 430.0 & 810.0 & 270.0 & 550.0 & 135.0 & 234.0 & 284.0 & 384.0 & 350.0 \\
           & mod & 159.8 & 648.4 & 217.3 & 363.7 & 611.7 & 269.5 & 479.2 & 135.5 & 231.3 & 276.9 & 375.4 & 327.9 \\
           & obs/mod & 1.990 & 1.041 & 1.063 & 1.182 & 1.324 & 1.002 & 1.148 & 0.996 & 1.012 & 1.026 & 1.023 & 1.067 & 1.105 & 1.024 \\ 
\hline
    195592 & obs & 215.0 & 641.0 & 207.0 & 386.0 & 811.0 & 309.0 & 500.0 & 145.0 & 202.0 & 263.0 & \ldots & 364.0 \\
           & mod & 163.9 & 631.3 & 222.8 & 370.1 & 626.0 & 280.6 & 490.5 & 147.2 & 211.0 & 253.9 & 314.7 & 323.4 \\
           & obs/mod & 1.312 & 1.015 & 0.929 & 1.043 & 1.296 & 1.101 & 1.019 & 0.985 & 0.957 & 1.036 & \ldots & 1.126 & 1.031 & 1.036 \\ 
\hline
    201345 & obs & 313.0 & 925.0 & 355.0 & 546.0 & 910.0 & 300.0 & 629.0 & 170.0 & 419.0 & 481.0 & 768.0 & 624.0 \\
           & mod & 242.6 & 910.7 & 316.8 & 482.0 & 868.5 & 315.5 & 610.9 & 167.0 & 492.9 & 591.2 & 758.7 & 674.8 \\
           & obs/mod & 1.290 & 1.016 & 1.121 & 1.133 & 1.048 & 0.951 & 1.030 & 1.018 & 0.850 & 0.814 & 1.012 & 0.925 & 1.039 & 0.887 \\ 
\hline
    202124 & obs & 182.0 & 641.0 & 188.0 & 339.0 & 747.0 & 283.0 & 429.0 & 141.0 & 346.0 & 382.0 & 211.0 & 582.0 \\
           & mod & 150.7 & 710.8 & 212.3 & 358.8 & 634.7 & 297.3 & 471.7 & 137.5 & 346.1 & 419.4 & 430.9 & 548.2 \\
           & obs/mod & 1.208 & 0.902 & 0.886 & 0.945 & 1.177 & 0.952 & 0.909 & 1.025 & 1.000 & 0.911 & 0.490 & 1.062 & 0.948 & 0.955 \\ 
\hline
    207198 & obs & 186.0 & 745.0 & 188.0 & 369.0 & 817.0 & 281.0 & 457.0 & 122.0 & 440.0 & 479.0 & 488.0 & 658.0 \\
           & mod & 134.3 & 748.1 & 195.1 & 334.0 & 622.3 & 288.2 & 443.3 & 119.5 & 446.9 & 538.5 & 548.3 & 674.1 \\
           & obs/mod & 1.385 & 0.996 & 0.964 & 1.105 & 1.313 & 0.975 & 1.031 & 1.021 & 0.985 & 0.890 & 0.890 & 0.976 & 1.026 & 0.933 \\ 
\hline

\end{tabular}
\end{center}
\end{sidewaystable} 

\begin{sidewaystable} 
\scriptsize
\begin{center}
Table \ref{ta:ostars_ews}: \emph{continued}\\
\vspace{\abovecaptionskip}
\begin{tabular}{lc|cccccccc|cccc|cc}
\hline\hline
\tabhead
\hline
    209975 & obs & 210.0 & 700.0 & 179.0 & 362.0 & 814.0 & 273.0 & 441.0 & 114.0 & 330.0 & 377.0 & 364.0 & 512.0 \\
           & mod & 130.6 & 686.1 & 188.2 & 322.0 & 605.0 & 268.2 & 431.9 & 113.5 & 350.3 & 428.0 & 489.9 & 534.6 \\
           & obs/mod & 1.608 & 1.020 & 0.951 & 1.124 & 1.345 & 1.018 & 1.021 & 1.004 & 0.942 & 0.881 & 0.743 & 0.958 & 1.021 & 0.911 \\ 
\hline
    210809 & obs & 223.0 & 700.0 & 206.0 & 368.0 & 870.0 & 310.0 & 474.0 & 135.0 & 416.0 & 472.0 & 214.0 & 666.0 \\
           & mod & 111.7 & 715.5 & 169.0 & 296.5 & 602.1 & 281.6 & 408.6 & 104.0 & 428.9 & 520.9 & 477.8 & 679.0 \\
           & obs/mod & 1.996 & 0.978 & 1.219 & 1.241 & 1.445 & 1.101 & 1.160 & 1.298 & 0.970 & 0.906 & 0.448 & 0.981 & 1.230 & 0.938 \\ 
\hline
    214680 & obs & 224.0 & 791.0 & 247.0 & 399.0 & 646.0 & 265.0 & 466.0 & 117.0 & 632.0 & 562.0 & 797.0 & 764.0 \\
           & mod & 155.6 & 786.1 & 227.7 & 352.4 & 766.4 & 260.6 & 458.5 & 109.6 & 453.9 & 581.1 & 802.8 & 699.4 \\
           & obs/mod & 1.440 & 1.006 & 1.085 & 1.132 & 0.843 & 1.017 & 1.016 & 1.068 & 1.392 & 0.967 & 0.993 & 1.092 & 1.042 & 1.043 \\ 
\hline
    218195 & obs & 281.0 & 777.0 & 204.0 & 407.0 & 788.0 & 270.0 & 470.0 & 121.0 & 509.0 & 552.0 & 728.0 & 740.0 \\
           & mod & 146.9 & 792.1 & 210.5 & 348.5 & 655.1 & 289.7 & 454.5 & 123.1 & 522.5 & 624.0 & 662.4 & 746.0 \\
           & obs/mod & 1.913 & 0.981 & 0.969 & 1.168 & 1.203 & 0.932 & 1.034 & 0.983 & 0.974 & 0.885 & 1.099 & 0.992 & 1.009 & 0.983 \\ 
\hline
    218915 & obs & 168.0 & 653.0 & 201.0 & 346.0 & 794.0 & 287.0 & 467.0 & 115.0 & 346.0 & 288.0 & 317.0 & 456.0 \\
           & mod & 119.1 & 640.6 & 175.4 & 302.7 & 592.5 & 268.8 & 416.5 & 110.7 & 297.9 & 368.2 & 381.0 & 492.3 \\
           & obs/mod & 1.411 & 1.019 & 1.146 & 1.143 & 1.340 & 1.068 & 1.121 & 1.039 & 1.161 & 0.782 & 0.832 & 0.926 & 1.132 & 0.879 \\ 
\hline
    225160 & obs & 196.0 & 650.0 & 129.0 & 256.0 & 773.0 & 253.0 & 343.0 & 143.0 & 506.0 & 620.0 & \ldots & 798.0 \\
           & mod & 86.2 & 749.1 & 136.4 & 246.6 & 581.7 & 264.8 & 354.1 & 80.3 & 548.1 & 651.1 & 565.5 & 824.5 \\
           & obs/mod & 2.274 & 0.868 & 0.946 & 1.038 & 1.329 & 0.955 & 0.969 & 1.781 & 0.923 & 0.952 & \ldots & 0.968 & 1.003 & 0.952 \\ 
\hline
  +36 4063 & obs & 187.0 & 580.0 & 252.0 & 378.0 & 666.0 & 269.0 & 406.0 & 150.0 & 296.0 & 227.0 & 275.0 & 356.0 \\
           & mod & 158.4 & 651.9 & 217.5 & 363.6 & 621.5 & 281.8 & 481.0 & 141.4 & 244.4 & 294.5 & 360.7 & 373.0 \\
           & obs/mod & 1.181 & 0.890 & 1.159 & 1.040 & 1.072 & 0.955 & 0.844 & 1.061 & 1.211 & 0.771 & 0.762 & 0.954 & 1.050 & 0.863 \\ 
\hline
    Median & obs/mod & 1.411 & 0.996 & 0.992 & 1.105 & 1.296 & 0.975 & 1.016 & 1.018 & 1.012 & 0.902 & 0.933 & 1.000 \\ 

\hline\hline
\end{tabular}
\end{center}
\end{sidewaystable} 
%%%%%%%%%%%%%%%%%%%%%%%%%%%%%%%%%


\begin{table} %%%%%%% TABLE %%%%%%%%%%%%%%%
\begin{center}
\caption{Derived parameters for the WHT O stars}
\vspace{\abovecaptionskip}
\begin{tabular}{llccc r@{~$\pm$~}l r@{~$\pm$~}l r@{~$\pm$~}l}
\hline\hline
HD/BD & \multicolumn{1}{c}{Sp. Type} &
\teff (kK) & \logg & $M_V$ & 
\multicolumn{2}{c}{$\log_{10}L/L_\odot$} & 
\multicolumn{2}{c}{$R/R_\odot$} &
\multicolumn{2}{c}{ $M/M_\odot$}  \\
\hline
% Automatically generate tex table Mon Jan 18 16:00:18 GMT 1999
% cat spec teff logg y Mv logl err r/rsol err m/msol err
   10125 &       O9.7 II & 32.5 &  3.3 & $-5.8 $ & 5.5 & 0.2 & 17 &  3 & 21 & 12 \\
   12323 &         ON9 V & 36.0 &  4.1 & $-3.5 $ & 4.7 & 0.1 &  5 &  1 & 14 &  8 \\
   13745 &  O9.7 II((n)) & 33.0 &  3.3 & $-5.3 $ & 5.3 & 0.2 & 13 &  2 & 13 &  8 \\
   16429 &  O9.5 II((n)) & 35.0 &  3.4 & $-6.8 $ & 5.9 & 0.1 & 25 &  4 & 59 & 34 \\
   30614 &       O9.5 Ia & 33.0 &  3.1 & $-6.1 $ & 5.6 & 0.2 & 20 &  3 & 19 & 11 \\
   34078 &        O9.5 V & 37.0 &  4.2 & $-4.0^*$ & 4.9 & 0.2 &  7 &  1 & 26 & 15 \\
   36486 &       O9.5 II & 34.0 &  3.4 & $-6.6 $ & 5.8 & 0.2 & 24 &  4 & 51 & 30 \\
   37742 &       O9.7 Ib & 34.0 &  3.3 & $-7.0 $ & 6.0 & 0.1 & 30 &  5 & 64 & 37 \\
  188209 &      O9.5 Iab & 33.0 &  3.1 & $-6.0 $ & 5.6 & 0.2 & 19 &  3 & 17 & 10 \\
  189957 &      O9.5 III & 33.0 &  3.5 & $-5.0 $ & 5.2 & 0.2 & 12 &  2 & 15 &  9 \\
  191781 &     ON9.7 Iab & 31.0 &  3.1 & $-6.0 $ & 5.5 & 0.1 & 20 &  3 & 19 & 11 \\
  194280 &     OC9.7 Iab & 32.0 &  3.2 & $-6.1 $ & 5.6 & 0.2 & 21 &  3 & 25 & 14 \\
  195592 &       O9.7 Ia & 31.0 &  3.0 & $-6.0 $ & 5.5 & 0.1 & 20 &  3 & 15 &  9 \\
  201345 &         ON9 V & 36.0 &  3.9 & $-4.2 $ & 4.9 & 0.1 &  8 &  1 & 16 &  9 \\
  202124 &      O9.5 Iab & 34.0 &  3.2 & $-6.0 $ & 5.6 & 0.1 & 19 &  3 & 20 & 12 \\
  207198 &      O9 Ib-II & 36.0 &  3.4 & $-5.5 $ & 5.5 & 0.1 & 14 &  2 & 19 & 11 \\
  209975 &       O9.5 Ib & 35.0 &  3.4 & $-5.7 $ & 5.5 & 0.2 & 16 &  3 & 23 & 14 \\
  210809 &        O9 Iab & 36.0 &  3.3 & $-6.2 $ & 5.8 & 0.1 & 20 &  3 & 28 & 16 \\
  214680 &          O9 V & 38.0 &  4.2 & $-4.4 $ & 5.1 & 0.2 &  8 &  1 & 37 & 21 \\
  218195 &        O9 III & 37.0 &  3.6 & $-5.1 $ & 5.3 & 0.2 & 11 &  2 & 18 & 10 \\
  218915 &      O9.5 Iab & 34.0 &  3.2 & $-6.0 $ & 5.6 & 0.1 & 19 &  3 & 20 & 12 \\
  225160 &      O8 Ib(f) & 38.0 &  3.4 & $-6.6 $ & 6.0 & 0.2 & 23 &  4 & 48 & 27 \\
+36 4063 &       ON9.7 I & 32.0 &  3.1 & $-5.9 $ & 5.5 & 0.2 & 19 &  3 & 16 & 10 \\

\hline\hline
\end{tabular}
\label{ta:ostars:derived}
\end{center}
\tnote{* -- See discussion in text}
\end{table} %%%%%%%%%%%%%%%%%%%%%%%%%%%%%%%%%


\begin{table}[p] %%%%%%% TABLE %%%%%%%%%%%%%%%
\begin{center}
\caption{Physical parameters in Voels et al. (1989)}
\vspace{\abovecaptionskip}
\begin{tabular}{lcccccc}
\hline\hline
&
\multicolumn{3}{c}{This thesis} &
\multicolumn{3}{c}{Voels et al.} \\
\multicolumn{1}{c}{HD} &
 \teff (kK) & \logg & $y$ & \teff (kK) & \logg & $y$ \\
\hline
30614	& 33.0 & 3.1 & 0.13 & 30.0 & 2.90 & 0.18 \\
34078 	& 37.0 & 4.2 & 0.09 & 35.5 & 3.95 & 0.10 \\
36486	& 34.0 & 3.4 & 0.10 & 33.0 & 3.45 & 0.10 \\
37742	& 34.0 & 3.3 & 0.10 & 32.0 & 3.20 & 0.10 \\
\hline\hline
\end{tabular}
\label{ta:comp:voels}
\end{center}
\end{table} %%%%%%%%%%%%%%%%%%%%%%%%%%%%%%%%%

\begin{table} %%%%%%% TABLE %%%%%%%%%%%%%%%
\begin{center}
\caption{Physical parameters in Grigsby et al. (1992)}
\vspace{\abovecaptionskip}
\begin{tabular}{lccccc}
\hline\hline
&
\multicolumn{3}{c}{This thesis} &
\multicolumn{2}{c}{Grigsby et al.} \\
\multicolumn{1}{c}{HD} &
 \teff (kK) & \logg & $y$ & \teff (kK) & \logg \\
\hline
12323	& 36.0 & 4.1 & 0.17 & 29.0 & 4.00 \\
201345	& 36.0 & 3.9 & 0.14 & 29.0 & 4.00 \\
214680	& 38.0 & 4.2 & 0.09 & 30.0 & 4.00 \\
218195	& 37.0 & 3.6 & 0.12 & 31.5 & 3.60 \\
\hline\hline
\end{tabular}
\label{ta:comp:grigsby}
\end{center}
\end{table} %%%%%%%%%%%%%%%%%%%%%%%%%%%%%%%%%

\begin{table} %%%%%%% TABLE %%%%%%%%%%%%%%%
\begin{center}
\caption{Physical parameters in Herrero et al. (1992)}
\vspace{\abovecaptionskip}
\begin{tabular}{lcccccc}
\hline\hline
&
\multicolumn{3}{c}{This thesis} &
\multicolumn{3}{c}{Herrero et al.} \\
\multicolumn{1}{c}{HD} &
 \teff (kK) & \logg & $y$ & \teff (kK) & \logg & $y$ \\
\hline
34078 	& 37.0 & 4.2 & 0.09 & 36.5 & 4.05 & 0.09 \\
207198	& 36.0 & 3.4 & 0.12 & 34.0 & 3.30 & 0.12 \\
209975	& 35.0 & 3.4 & 0.09 & 32.5 & 3.20 & 0.09 \\
210809	& 36.0 & 3.3 & 0.13 & 33.0 & 3.10 & 0.12 \\
214680	& 38.0 & 4.2 & 0.09 & 37.5 & 4.00 & 0.10 \\
\hline\hline
\end{tabular}
\label{ta:comp:herrero}
\end{center}
\end{table} %%%%%%%%%%%%%%%%%%%%%%%%%%%%%%%%%

\begin{figure} %%%%%%%%%%%%%%%%%%% FIGURE
\epsfxsize=9cm
\setlength{\cen}{(\textwidth / 2) - (\epsfxsize / 2)}
\hspace{\cen}\epsfbox{fig_analyses/results/teffplot.eps}
\caption[\teff\ against spectral types for sample stars]
{\fcfont Effective temperatures against spectral types for sample
stars. \textbf{top}~--~luminosity class I,
\textbf{middle}~--~luminosity classes II and III, \textbf{bottom}~--~luminosity
classes IV and V. The size of the symbols refers to the helium number
fraction, $y$ (see top diagram for key). `Typical' parameters for O
stars from Howarth and Prinja (1989) are drawn as lines.}
\label{fig:teffplot}
\end{figure} %%%%%%%%%%%%%%%%%%% END FIGURE

\begin{figure} %%%%%%%%%%%%%%%%%%% FIGURE
\epsfxsize=9cm
\setlength{\cen}{(\textwidth / 2) - (\epsfxsize / 2)}
\hspace{\cen}\epsfbox{fig_analyses/results/loggplot.eps}
\caption[\logg\ against spectral types for sample stars]
{\fcfont Surface gravity against spectral types for sample
stars. \textbf{top}~--~luminosity class I,
\textbf{middle}~--~luminosity classes II and III, \textbf{bottom}~--~luminosity
classes IV and V. The size of the symbols refers to the helium number
fraction, $y$ (see top diagram for key). `Typical' parameters for O
stars from Howarth and Prinja (1989) are drawn as lines.}
\label{fig:loggplot}
\end{figure} %%%%%%%%%%%%%%%%%%% END FIGURE

%%%%%%%%%%%%%%%%%% END-OF-CHAPTER FIGURES %%%%%%%%%%%%%%%
\newpage


\begin{figure}[H]
\epsfxsize=14cm
\epsfbox{fig_analyses/ewds.1.eps}
\caption[Fit diagrams for all sample stars]
{\fcfont Fit diagrams for all sample stars. Catalogue number and
adopted value of $y$ shown in upper left-hand corner of each
diagram. Lines that appear horizontal at the bottom of the diagrams fall
outside the plotted area.}
\label{fig:allewd}
\end{figure}

\begin{figure}[H]
\epsfxsize=14cm
\epsfbox{fig_analyses/ewds.2.eps}
\begin{center}
\vspace{\abovecaptionskip}
Figure \ref{fig:allewd}: \fcfont -- continued
\end{center}
\end{figure}

\begin{figure}[H]
\epsfxsize=14cm
\epsfbox{fig_analyses/ewds.3.eps}
\begin{center}
\vspace{\abovecaptionskip}
Figure \ref{fig:allewd}: \fcfont -- continued
\end{center}
\end{figure}




