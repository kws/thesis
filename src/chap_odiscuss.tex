\chapter{Properties of the O stars}
\label{ch:odiscuss}

In chapter \ref{ch:analyses} we derived physical parameters for a
large sample of O stars.  There are obvious shortcomings to the
models, as is clearly seen in the case of \hei~$\lambda$4471\AA, for
which the models consistently fail to reproduce the observed
spectra. However, even with important physics left out or
approximated, we believe the stellar models of today are good enough
to give us a reasonable representation of the stellar atmosphere
and a unique insight into the processes that might take place in, on
and around the star. In this chapter we will look at the stellar
parameters derived in chapters \ref{ch:casestudy} and
\ref{ch:analyses}, and try to interpret some of the observed properties.

\subsubsection{Chapter conventions}
Several of the figures shown in this chapter will use a set of
`standard' symbols used throughout this chapter (and also in some
diagrams appearing in earlier chapters). Table \ref{ta:symbols}
summarises the different symbols. In addition to use of different
symbols for different subgroups of stars, the symbols are also
sometimes scaled according to some other parameter, e.g. the inferred
helium number fraction. In these cases the scaling will be explained,
either in the caption, or in the figure itself.

When the figures are there to investigate a relationship between two
parameters, rank correlation coefficients are calculated and shown in
the figure. Two such correlation coefficients are calculated; the Spearman
Rank-Order Correlation Coefficient, $r_\mathrm{s}$, and Kendall's~$\tau$.
Both of these are shown with their respective significance levels,
$p$. A small value of $p$ indicates a significant correlation. The
properties of these correlation coefficients are described in
\emph{Numerical Recipes} (\cite{nr}).

\begin{table} %%%%%%%%% TABLE (symbols)
\begin{center}
\caption{Symbols used in this chapter}
\vspace{\abovecaptionskip}
\begin{tabular}{r|cc}
\hline\hline
\multicolumn{1}{c}{} & \multicolumn{2}{c}{Luminosity Class}\\
\multicolumn{1}{c}{} & I, II, III & IV, V \\ \hline
Morphologically normal&
\epsfxsize=11pt\epsfbox{fig_odiscuss/symbols/oI.eps}&
\epsfxsize=11pt\epsfbox{fig_odiscuss/symbols/oV.eps}\\
OC classified&
\epsfxsize=11pt\epsfbox{fig_odiscuss/symbols/ocI.eps}\\
ON classified&
\epsfxsize=11pt\epsfbox{fig_odiscuss/symbols/onI.eps}&
\epsfxsize=11pt\epsfbox{fig_odiscuss/symbols/onV.eps}\\
\hline\hline
\end{tabular}
\label{ta:symbols}
\end{center}

\tnote{There are no known OC main-sequence stars.}
\end{table} %%%%%%%%%%%%% TABLE

\subsubsection{H--R diagrams and evolutionary tracks}
Figure \ref{fig:hrd:params} shows the sample stars on the H--R
diagram, with evolutionary tracks from \ybcite{schaller:92}. Also
indicated on the diagram is the Zero Age Main Sequence (ZAMS) and the
Terminal Age Main Sequence (TAMS). The latter is taken here to be the
most redward point the stars reach during their core hydrogen burning
lifetimes. One of the ON main sequence stars, the binary star
HD~12323, falls below the ZAMS (although the ZAMS is well within the
error box surrounding this star; see figure for typical uncertainties).

% NOTE ON POSITION OF HD~12323
%offset about 1.6~kK and log(L/L) between 4.65 and 4.8 although
%if we combine these the individual errors get a lot smaller.


\begin{figure} %%%%%%%%%FIGURE
\epsfxsize=\figwidth
\setlength{\cen}{(\textwidth / 2) - (\epsfxsize / 2)}
\hspace{\cen}\epsfbox{fig_odiscuss/hrd/hrd.eps}
\caption[H--R diagram with evolutionary tracks]
{\fcfont H--R diagram showing O star parameters and evolutionary
tracks. Main sequence stars are shown as squares, all other stars as
circles. The size of the symbols depends on the helium number fraction,
$y$ (see key in lower left-hand corner). The OC star is shown as a
filled circle, and the ON stars are shown cross-hatched. Evolutionary
tracks for 20, 25, 40, 60 and 85 $M_\odot$ from Schaller et al. (1992)
are shown, with the predicted ZAMS and TAMS (dashed line). Typical
error box (assuming an error of 0.3 on $M_V$; see Herrero et al. 1992)
is shown to the left of the ZAMS.}
\label{fig:hrd:params}
\end{figure}   %%%%%%%%%%%%%%%

\begin{figure} %%%%%%%%%FIGURE
\epsfxsize=11cm
\setlength{\cen}{(\textwidth / 2) - (\epsfxsize / 2)}
\hspace{\cen}\epsfbox{fig_odiscuss/schaller_fig5.eps}
\caption[H--R diagram including post-MS evolutionary tracks]
{\fcfont H--R diagram including post-MS evolutionary tracks from
Schaller et al. (1992).}
\label{fig:hrd:evol}
\end{figure}   %%%%%%%%%%%%%%%

After the core hydrogen is exhausted, the stars make a long trip
redwards in the H--R diagram before they return again (only for stars
with a ZAMS mass, $M_0 \gtrsim 40M_\odot$) to the O star
region, as shown in figure \ref{fig:hrd:evol}; this time the models
 predict nuclear processed material on the surface, and the
star are tens of per cent lighter than they were on the ZAMS. On the
return trip to the O-star region, the stars are very evolved and have
a high mass-loss. The atmospheres are predicted to show large amounts
of CNO processed materials, and much higher helium fractions than we observe.


%%%%%%%%%%%%%%%%%%%%%%%%%%%%%%%%%%%%%%%%%%%%%%%%%%%%
%%%%%%%%%%%%%% SECTION = helium abundance %%%%%%%%%%
%%%%%%%%%%%%%%%%%%%%%%%%%%%%%%%%%%%%%%%%%%%%%%%%%%%%
\section{The helium abundance}
\label{sec:helium}

From looking at the results listed in table \ref{ta:ostars} we see
than one of the most noticeable trends is that a large number of the
stars show a higher-than-solar helium abundance. This has also been
found by several other authors. \ybcite{ku:83} and \ybcite{bo:86} both
% Kudritzki et al. 1983 A&A 118 245
% Bohannan et al 1986 ApJ 308 728
found increased helium abundances for the O4~If star $\zeta$~Pup
($y$~=~0.14 and 0.17 respectively). \ybcite{sc:88} analysed a sample
of OBN class V stars and compared them to the normal OB stars 10~Lac
and $\tau$~Sco. All the nitrogen-rich stars also came out with high
helium abundances, whereas the normal stars had solar
composition. \ybcite{vo:89} performed a study of stars of spectral
type O9.5 and found helium enrichment only in the Ia star
$\alpha$~Cam. All other luminosity classes showed normal helium
abundances. Along the same lines, \ybcite{he:92} studied a sample of
OB stars and came to the conclusions that all main sequence stars (if
not fast rotators) have normal helium abundances; almost all stars
with a \vsini\ faster than 200~\kms\ or classified as Ia, Iab, f or
(f) show helium enhancement ($y \geq 0.12$; most of the sample $y \geq
0.15$); and that all other objects show normal helium fractions ($y
\leq 0.12$). From this they deduce that most O giants and supergiants
show CNO enrichment which is ``apparently the only explanation for
helium enrichment'', and further that the enrichment is reinforced
when the star has a high
\vsini\ or a low gravity and high luminosity. More recently,
\ybcite{sm:94} studied three supergiants of approximately the same spectral
type and luminosity class, one of which was an OC star, one was
classified as ON, and the third was a normal O star. They found that
both the ON and the O star showed enhanced helium abundances (0.17 and
0.13 respectively), and that the OC had a solar composition.


\subsection{Evolutionary effects}
All these effects could be explained as evolutionary effects. Hot
stars experience a large mass loss through stellar wind during their
lifetimes, and this will slowly strip off the outer layers of the
stars, exposing deeper layers. These layers might contain the products
of nuclear burning brought up from the core through some deep
convective process. The evolutionary models by \ybcite{schaller:92}
discussed above, taking into account mass loss and convective
overshooting, do reveal nuclear processed materials on the surface of
the stars. This happens, however, at a much later evolutionary stage
(i.e., post red supergiant) than that which these stars appear to be
in, and the helium abundances predicted by the evolutionary models are
far larger than observed. Only the very high-mass evolutionary tracks
predict an increase in surface helium abundance while the star is
still core hydrogen burning. Figure \ref{fig:3D} shows the predicted
surface helium number fraction of the evolutionary tracks, and figure
\ref{fig:hrd:mod_y} shows the same diagram as figure
\ref{fig:hrd:params}, with the inclusion of the model helium fractions. To
make the diagram easier to read, all stars are here shown as identical
symbols. It appears that the predicted increase in $y$ is roughly the
same as observed. However, only stars at a higher luminosity and lower
temperature than those analysed in this study, are predicted to show
this increased helium abundance.

\begin{figure} %%%%%%%% FIGURE

\setlength{\cen}{\figwidth/2}

\begin{tabular}{cc}
$40 M_\odot$ & $60 M_\odot$ \\
\epsfxsize=\cen\epsfbox{fig_odiscuss/3D/he_numfrac40.eps}&
\epsfxsize=\cen\epsfbox{fig_odiscuss/3D/he_numfrac60.eps}\\
$85 M_\odot$ & $120 M_\odot$ \\
\epsfxsize=\cen\epsfbox{fig_odiscuss/3D/he_numfrac85.eps}&
\epsfxsize=\cen\epsfbox{fig_odiscuss/3D/he_numfrac120.eps}\\
\end{tabular}

\caption[Helium number fractions, $y$, predicted by evolutionary tracks]
{\fcfont Helium number fractions, $y$, predicted by evolutionary tracks.}
\label{fig:3D}
\end{figure} %%%%%%%


\begin{figure} %%%%%%%%%FIGURE
\epsfxsize=\figwidth
\setlength{\cen}{(\textwidth / 2) - (\epsfxsize / 2)}
\hspace{\cen}\epsfbox{fig_odiscuss/mod_y/mod_y.eps}
\caption[H--R diagram with evolutionary helium fractions]
{\fcfont H--R diagram showing O star parameters and evolutionary
tracks (as in figure \ref{fig:hrd:params}). The predicted surface
helium number fractions are shown as `ghost' stars on the evolutionary
tracks. The size of the symbols depends on the helium number fraction,
$y$ (see key in lower left-hand corner). All stars are shown as
circles to simplify diagram.}
\label{fig:hrd:mod_y}
\end{figure}   %%%%%%%%%%%%%%%

\ybcite{wa:76} proposed several scenarios to explain the observed
carbon, nitrogen and sometimes oxygen anomalies in the OBN and OBC
stars, and it is probable that these are very closely related to the
observed variations in helium abundances (see section
\ref{sec:cno_anomalies}). Apart from the evolutionary effects
discussed above, it is also possible that the observed abundance
anomalies (C, N or He) could reflect initial abundance
differences. The main (if not only) evidence supporting this theory is
the existence of the three nitrogen deficient Orion Belt stars
(HD~36486, HD~37128, HD~37742). HD~37128 is a B0~Ia star, and as such
is not covered by this study. The other two are, and both are found to
have a normal helium abundance ($y~=~0.10$). In a recent study,
\ybcite{mcer:98} analysed HD~37128 using models very similar to the
ones used in this thesis, and they derived $y$~=~0.10.

To return to Herrero's conclusions; they notice that all their main
sequence stars, except the fast rotators ($\vsini \geq 200$~\kms),
have normal helium abundances. Figure \ref{fig:helium:logg} shows $y$
against \logg\ and luminosity class. Unlike Herrero et al., we have no
main sequence stars with a \vsini\ faster than 200~\kms, but we do
observe that two (out of four) of the MS stars do show enhanced
helium. Both of these stars were classified as ON stars by Walborn,
and both are possibly binaries\footnote{HD~12323 is a binary according to
\ybcite{ga:80}; HD~201345 was listed as binary according to
\ybcite{bolt_rog}, but single by \ybcite{gi:86} and
\ybcite{mason:98}}. Although 50~\% of the MS sample show helium
enhancement, there is a strong bias towards stars classified as either
ON or OC in our sample (there are no known OC main sequence stars) so
this is certainly not a representative percentage of helium enhanced O
main-sequence stars.

\begin{figure} %%%%%%%%%FIGURE
\epsfxsize=\figwidth
\setlength{\cen}{(\textwidth / 2) - (\epsfxsize / 2)}
\hspace{\cen}\epsfbox{fig_odiscuss/deps/loggplot.eps}
\caption[Helium fractions for different \logg\ and luminosity classes]
{\fcfont Helium number fractions for different \logg\ (\textbf{top
panel}) and luminosity classes (\textbf{bottom panel}). $r_s$ is the
Spearman rank-order correlation coefficient, and $\tau$ is Kendall's
$\tau$. The numbers in brackets indicate the probabilities that the
points belong to a random distribution, ie no correlation.}
\label{fig:helium:logg}
\end{figure}   %%%%%%%%%%%%%%%

Our supergiants also show a much larger spread in helium abundances
than those studied by Herrero et al. Roughly half the luminosity class
I objects have a helium abundance of 0.12 or below, but only two have
a $y$ of 0.15 or higher, whereas Herrero et al. found that all but one
of their Ia, Iab, f or (f) classified stars had abundances of 0.15 or
higher. In our sample, only five stars appear to have a helium
abundance this high, and two of these are classified as ON (one is the ON
main-sequence binary HD~12323).

Comparing the position in the H--R diagram to evolutionary models
allows us to estimate the mass of the star, and also its
age. Interpolating between the models by \ybcite{schaller:92} we can
find the evolutionary track that passes through a given point on the
H--R diagram (by supplying a \teff\ and $L$). Mapping this
evolutionary track onto a \teff\ -- $\log_{10} (L/L_\odot)$ grid we
can find the current mass (see section \ref{sec:mass_discrepancy}) and
age from the spectroscopic \teff. Figure \ref{fig:helium:age} shows
the helium abundance as a function of age and fractional age (age / MS
lifetime). Neither shows a clear relation, but the rank correlation
coefficients $r_s$ and $\tau$ shown in upper right-hand corner of
figures, suggest that there is possibly some correlation
between age and $y$. The probability that there is no correlation is
only 12~--~17\%, but there is no conclusive evidence to suggest that
the helium abundance increases with age, especially since there
appears to be very small probability of a correlation between the
fractional age and $y$.

\begin{figure} %%%%%%%%%FIGURE
\epsfxsize=\figwidth
\setlength{\cen}{(\textwidth / 2) - (\epsfxsize / 2)}
\hspace{\cen}\epsfbox{fig_odiscuss/ages/age.eps}
\caption[Helium fractions against stellar age]
{\fcfont Helium number fractions against the stellar age predicted by
evolutionary models (Schaller et al. 1992). \textbf{Top panel} shows
$y$ vs. the evolutionary age, and the \textbf{bottom panel} shows $y$
vs. fractional age (age / core hydrogen burning lifetime). HD~12323 is
not plotted due to its position below the main-sequence. $r_s$ and
$\tau$ are the correlation coefficients as for figure
\ref{fig:helium:logg}.}
\label{fig:helium:age}
\end{figure}   %%%%%%%%%%%%%%%


\subsection{Rotation}
\label{sec:odiscuss:rotation}
As mentioned above, \ybcite{he:92} found that their fast rotators,
arbitrarily defined as those stars with a \vsini~$\geq$~200~\kms,
showed high ($y \geq 0.15$) helium abundances. We have only one star
with a broadening velocity higher than 200~kms. Figure
\ref{fig:helium:vsini} shows $y$ as a function of broadening
velocity. Notice that the MS stars (marked as squares) and the two
fastest rotators belong to a separate distribution from the other
stars due to the use of different profiles to estimate the broadening
velocity. We have, however, treated them as one dataset when calculating
the correlation coefficient, $r$, shown in the upper right-hand
corner. Although the star with the broadest absorption lines shows only a
moderately high helium abundance, there appears to be a clear trend
between the width of the spectral lines and the helium abundance. From
the linear correlation coefficient we find a low probability
that these parameters are not correlated.

\begin{figure} %%%%%%%%%FIGURE
\epsfxsize=\figwidth
\setlength{\cen}{(\textwidth / 2) - (\epsfxsize / 2)}
\hspace{\cen}
\epsfbox{fig_odiscuss/deps/vsini.eps}
\caption[Helium fractions against broadening velocity]
{\fcfont Helium number fractions against the stellar broadening
velocity. Symbols are as shown in table \ref{ta:symbols}}
\label{fig:helium:vsini}
\end{figure}   %%%%%%%%%%%%%%%

If we assume that the main broadening mechanism in all O-stars is
rotation, then the implication of this would be that there is some
turbulent mixing of the stellar interior caused by the rotation that
brings processed material up to the surface. There is, however,
extensive evidence to suggest that this is not the case. The almost
total absence of sharp-lined O stars (as would be expected in the case
of rotational broadening, considering a number of the stars would be
seen pole-on) could imply that the broadening must, at least partly, be
caused by other mechanisms (see discussion in chapter
\ref{ch:xcorr}). That the dependence of $y$ on the line broadening
velocity is apparently the same for both main-sequence and giant stars
implies that the broadening mechanism must be the same. However, there
are only 4 main-sequence stars in the sample, and two of these have
very narrow lines, so the sample is too small to draw any firm
conclusions.

There is of course the possibility that the increases in helium
abundance are not real. It is easy to imagine that for stars with broad
lines, it is difficult to detect blends, and the measured equivalent
widths might be overestimated. Figure \ref{fig:helium:hd214680_250}
shows a simulation of this effect. The left hand column of the figures
shows the fit diagrams for the measured equivalent widths of 10~Lac
(HD~21460). This star has very narrow lines, and a `normal' helium
abundance. The right hand column shows equivalent widths measured
after artificially spinning this star up to 250~\kms\ by convolving the
spectrum with equation \ref{eq:mod:rotation} discussed in chapter
\ref{ch:mod}. The star now appears to have a helium abundance of roughly
0.12, compared to the `original' adopted value of 0.09. However, we
made no attempts at using multiple gaussians to fit irregular looking
lines, and it is quite likely that, in a careful study, these errors
could be reduced by careful fitting of the lines. It is therefore
possible that blending effects could cause a \emph{slight} increase in
the adopted helium fractions, but not enough to explain the enhanced
helium abundances observed in this study.

\begin{figure} %%%%%%%%%FIGURE
\epsfxsize=15cm
\epsfbox{fig_odiscuss/hd214680_250.eps}
\caption[Fit diagrams for `spun-up' 10~Lac]
{\fcfont Fit diagrams for artificially `spun-up' 10~Lac.  The
left-hand column shows the real fit diagrams for 10~Lac (HD~214680)
with an adopted helium fraction of 0.09. The right-hand column shows
fit diagrams from equivalent widths measured after artificially
`spinning-up' the spectrum of 10~Lac to 250~\kms. The star now appears
to have a helium fraction of 0.12.}
\label{fig:helium:hd214680_250}
\end{figure}   %%%%%%%%%%%%%%%

Another factor that might be contributing to the increase in helium
abundances observed by Herrero et al. for the very fastest rotators,
could be caused by rotational distortion. For stars that rotate very
fast, close to their break-up angular velocity, $\omega_\mathrm{c}$,
the equatorial radius will be significantly larger than the polar
radius, and similarly the gravity at the equator will be lower than at
the pole.  From the `law of gravity darkening' (\cite{zeipel}) we have
$\teff^4 \propto |\mathbf{g}|$. This will lead to differences in
atmospheric structure at the poles and equator, and differing
contributions from different elements (it will also lead to different
line profiles for lines that form predominantly near the equator and
those that form predominantly at the poles).

Ian Howarth (private communication) has generated synthetic spectra
for stars with $\omega/\omega_\mathrm{c}$ near 1. At these high
velocities the stars `bulge' around the equator, and consequently have
lower effective temperature and gravity at the equator. The synthetic
spectra are generated by dividing the stellar surface into sections,
and calculating an average \teff\ and \logg\ for each section. Spectra
from the grid of models are then co-added to generate a synthetic
spectrum of the star. We have fitted a random sample of these
synthetic spectra as if they were actual observed stars (the `true'
atmospheric parameters were not revealed until after the
analyses). Table \ref{ta:omega_crit} summarises the results of the
analysis. The models closest to their breakup-velocity appear to have
a slightly increased helium abundance. The effect is very small,
however, and probably too small to be observed in real stellar
spectra. For the low $\omega/\omega_\mathrm{c}$ models, we find values
of \teff\ and \logg\ very close to the input parameters. However, as
$\omega/\omega_\mathrm{c}$ increases, we find values that appear to
lie about half-way between the polar and equatorial values. Adopting
these values and computing synthetic spectra, we find that the \hei\
lines are predicted too strong and the \heii\ lines too broad and
weak. It would be very difficult to fit the observed line profiles
using just one value of \vsini\ since the \hei\ lines are mainly
formed closer to the hot poles (\heii\ is very temperature sensitive)
where the star rotates slower. From table \ref{ta:omega_crit} it
appears that the effects of rotational distortion are unable to
account for the observed enhanced helium abundances.

\begin{table}
\begin{center}
\caption{Results of high $\omega/\omega_\mathrm{c}$ analyses}
\vspace{\abovecaptionskip}
\begin{tabular}{ccccc|cccc}
\hline\hline
\multicolumn{5}{c|}{Input spectrum} &
\multicolumn{4}{c}{Fit parameters} \\
$\omega/\omega_\mathrm{c}$ 	&
$T_\mathrm{p}$			& $T_\mathrm{e}$ & 
$\log_{10}g_\mathrm{p}$ 	& $\log_{10}g_\mathrm{e}$ &
\teff				& \logg &
$y$				& \vsini \\
& (kK) & (kK) & & & (kK) & & & (\kms) \\
\hline
%Model 2
0.10 & 32.00 & 31.95 & 3.50 & 3.50 &
32.0 & 3.5 & 0.09 & 440 \\
%Model 3
0.10 & 40.00 & 39.94 & 3.90 & 3.90 &
39.0 & 3.9 & 0.09 & 440 \\
%Model 4
0.90 & 32.00 & 25.21 & 3.50 & 3.09 &
27.0 & 3.2 & 0.10 & 435 \\
%Model 5
0.90 & 40.00 & 31.51 & 3.90 & 3.49 & 
35.0 & 3.8 & 0.09 & 444 \\
%Model 8
0.95 & 32.00 & 23.00 & 3.50 & 2.93 &
27.0 & 3.3 & 0.09 & 435 \\
%Model 7
0.95 & 40.00 & 28.74 & 3.90 & 3.33 &
33.0 & 3.5 & 0.10 & 442 \\
%Model 6
0.98 & 32.00 & 20.37 & 3.50 & 2.72 &
27.0 & 3.1 & 0.10 & 428 \\
%Model 1
0.98 & 40.00 & 25.46 & 3.90 & 3.12 &
31.0 & 3.5 & 0.10 & 436 \\
\hline\hline
\end{tabular}
\label{ta:omega_crit}
\end{center}
\tnote{The velocity of all input spectra were 438~\kms. The velocities
listed are those found from the \hei\ lines. \heii\ line broadening
velocities are lower, at about 350~\kms, because the main contribution
is from closer to the hot poles. The subscripted \emph{p}s and
\emph{e}s denote polar and equatorial values respectively.}
\end{table}


%%%%%%%%%%%%%%%%%%%%%%%%%%%%%%%%%%%%%%%%%%%%%%%%%%%%
%%%%%%%%%%%%%% SECTION = mass discrepancy %%%%%%%%%%
%%%%%%%%%%%%%%%%%%%%%%%%%%%%%%%%%%%%%%%%%%%%%%%%%%%%
\section{The `mass discrepancy'}
\label{sec:mass_discrepancy}
Using exactly the same method as we described earlier to find the
stellar ages, we can also find their `evolutionary masses'. The
evolutionary mass is the mass predicted by the (model) evolutionary
track that passes through a star's position on the H--R diagram. Table
\ref{ta:ostars:mass_disc} lists the spectroscopic mass (derived in
chapter \ref{ch:analyses}), the ZAMS mass, and the `current'
evolutionary mass for each target. For the main sequence stars the two
`evolutionary' masses are similar, but it is apparent from the table
that the stars lose a significant amount of mass during their core
H-burning lifetime.

%Groenwegen, a&a 221
\ybcite{gr:89} and \ybcite{he:92} both noticed that there was a discrepancy
between the spectroscopic and evolutionary masses in O and B
stars. Herrero et al. suggested that this discrepancy could be made
smaller by assuming the stars were no longer core H-burning, and on
their way back to the blue after the red supergiant stage. However, at
this stage the stars are expected to be at a highly evolved stage with
very different properties than those observed (different surface
abundances, mass-loss etc.), so this is unlikely to be the case. Herrero
et al. also derive masses from radiation driven wind theory, and
find that these masses agree within the errors to the spectroscopic
masses. This probably rules out any large systematic errors in the
spectroscopic masses (however, see discussion below about the impact
of recent evolutionary models).

Table \ref{ta:ostars:mass_disc} clearly shows that the mass
discrepancy is also significant for our data, and that the
distribution of values is very similar to that found by Herrero et
al. Although we only have a very small sample of main-sequence stars,
we clearly see that the mass discrepancy is significantly smaller for
these objects. Figures \ref{fig:dm:logg} and \ref{fig:dm:age} show the
mass discrepancy against \logg\ and age. These two parameters are
clearly related, so it is no big surprise to see that the distribution
of values is similar in the two figures. What \emph{is} striking
however is a small clustering of three stars with near zero mass
discrepancy. These are clearly not main-sequence stars, as they both
show a higher age and lower surface gravity than the luminosity class
V objects. These three stars are the two Orion Belt stars (HD~36486
and HD~37742) and HD~16429 (which is the broad-lined star, with a
relatively low helium abundance). It is very interesting that these
three stars are the only stars in the sample which appear on Walborn's
list of normal O stars with moderate N deficiencies. The third Orion
Belt star, $\epsilon$~Ori (HD~37128; B0~Ia), which also appears on
Walborn's list, has recently been studied by \cite{mcer:98}. For their
adopted parameters, and $M_V = -7.0$ from \ybcite{gr:89}, we find that
this star has a $M_s = 48.5$ and $M_e = 49.9$; so it appears that this
star also belongs to the group with $\delta M \sim 0$.

Recent evolutionary models that also take into account the effects of
stellar rotation could partly help resolve the problem of the mass
discrepancy. Rotating stars evolve towards higher luminosity than
predicted by non-rotating models (\cite{ma:98}; \cite{me:98}). This
is mainly caused by two processes:
%
\begin{itemize}
\item	Rotational mixing `erodes' the molecular-weight gradient at
	the outer edge of the convective core, leading to a mass
	enhancement of the convective core.
\item	The mixing brings hydrogen-burning products (such as helium)
	to the surface. The surface helium enhancement decreases
	opacity.
\end{itemize}
%
Since the evolutionary masses are derived by comparing the
evolutionary tracks with observed luminosities and temperatures, the
effect of the overluminous models is to lower the evolutionary masses,
which would bring them more in line with the spectroscopic masses.

\begin{table} %%%%%%% TABLE %%%%%%%%%%%%%%%
\begin{center}
\caption{Spectroscopic and evolutionary masses for the O stars}
\vspace{\abovecaptionskip}
\begin{tabular}{llccccccc}
\hline\hline
HD/BD & \multicolumn{1}{c}{Sp. Type} &
\teff (kK) & \logg & $y$ & $M_s$ & $M_0$ & $M_e$ & $\delta M$\\
\hline
% Automatically generate tex table Thu Oct 15 11:23:38 BST 1998
% cat spec teff logg y M_s M_0 Me dM 
   10125 &       O9.7 II & 32.5 &  3.3 & 0.15 &20.9 & 34.2 & 32.8 & $11.9$\\
   12323 &         ON9 V & 36.0 &  4.1 & 0.17 &13.6 & 20.7 & 20.6 & $ 7.0$\\
   13745 &  O9.7 II((n)) & 33.0 &  3.3 & 0.20 &12.9 & 29.0 & 28.1 & $15.2$\\
   16429 &  O9.5 II((n)) & 35.0 &  3.4 & 0.12 &59.0 & 62.8 & 58.0 & $-1.0$\\
   30614 &       O9.5 Ia & 33.0 &  3.1 & 0.13 &18.6 & 42.0 & 39.6 & $21.0$\\
   34078 &        O9.5 V & 37.0 &  4.2 & 0.09 &26.2 & 24.1 & 24.1 & $-2.2$\\
   36486 &       O9.5 II & 34.0 &  3.4 & 0.10 &51.2 & 54.2 & 49.8 & $-1.4$\\
   37742 &       O9.7 Ib & 34.0 &  3.3 & 0.10 &64.5 & 69.7 & 64.3 & $-0.1$\\
  188209 &      O9.5 Iab & 33.0 &  3.1 & 0.14 &16.9 & 40.1 & 38.0 & $21.1$\\
  189957 &      O9.5 III & 33.0 &  3.5 & 0.11 &15.5 & 26.1 & 25.4 & $ 9.9$\\
  191781 &     ON9.7 Iab & 31.0 &  3.1 & 0.16 &18.8 & 36.3 & 34.4 & $15.6$\\
  194280 &     OC9.7 Iab & 32.0 &  3.2 & 0.09 &24.6 & 39.7 & 37.5 & $12.9$\\
  195592 &       O9.7 Ia & 31.0 &  3.0 & 0.12 &14.9 & 36.3 & 34.4 & $19.5$\\
  201345 &         ON9 V & 36.0 &  3.9 & 0.14 &16.4 & 23.9 & 23.7 & $ 7.4$\\
  202124 &      O9.5 Iab & 34.0 &  3.2 & 0.13 &20.4 & 41.9 & 39.7 & $19.4$\\
  207198 &      O9 Ib-II & 36.0 &  3.4 & 0.12 &18.8 & 36.9 & 35.8 & $17.0$\\
  209975 &       O9.5 Ib & 35.0 &  3.4 & 0.09 &23.5 & 38.4 & 36.9 & $13.4$\\
  210809 &        O9 Iab & 36.0 &  3.3 & 0.13 &28.4 & 50.8 & 47.5 & $19.1$\\
  214680 &          O9 V & 38.0 &  4.2 & 0.09 &36.6 & 27.9 & 27.7 & $-8.9$\\
  218195 &        O9 III & 37.0 &  3.6 & 0.12 &18.1 & 32.2 & 31.6 & $13.5$\\
  218915 &      O9.5 Iab & 34.0 &  3.2 & 0.10 &20.4 & 41.9 & 39.7 & $19.4$\\
  225160 &      O8 Ib(f) & 38.0 &  3.4 & 0.15 &48.3 & 68.5 & 65.5 & $17.2$\\
+36 4063 &       ON9.7 I & 32.0 &  3.1 & 0.11 &16.2 & 36.2 & 34.5 & $18.2$\\
\hline\hline
\end{tabular}
\label{ta:ostars:mass_disc}
\end{center}
\tnote{$M_s$ -- spectroscopic mass; $M_0$ -- evolutionary ZAMS mass;
       $M_e$ -- evolutionary mass; $\delta M = M_e - M_s$\\
       All masses are in units of solar mass ($M_\odot$).}
\end{table} %%%%%%%%%%%%%%%%%%%%%%%%%%%%%%%%%

\begin{figure} %%%%%%%%%FIGURE
\epsfxsize=\figwidth
\setlength{\cen}{(\textwidth / 2) - (\epsfxsize / 2)}
\hspace{\cen}\epsfbox{fig_odiscuss/massdisc/dm_g.eps}
\caption[Mass discrepancy against \logg]
{\fcfont Mass discrepancy against \logg.}
\label{fig:dm:logg}
\end{figure}   %%%%%%%%%%%%%%%

\begin{figure} %%%%%%%%%FIGURE
\epsfxsize=\figwidth
\setlength{\cen}{(\textwidth / 2) - (\epsfxsize / 2)}
\hspace{\cen}\epsfbox{fig_odiscuss/massdisc/dm_age.eps}
\caption[Mass discrepancy against age]
{\fcfont Mass discrepancy against age.}
\label{fig:dm:age}
\end{figure}   %%%%%%%%%%%%%%%


%%%%%%%%%%%%%%%%%%%%%%%%%%%%%%%%%%%%%%%%%%%%%%%%%%%%
%%%%%%%%%%%%%% SECTION = OBN & OBC %%%%%%%%%%%%%%%%%
%%%%%%%%%%%%%%%%%%%%%%%%%%%%%%%%%%%%%%%%%%%%%%%%%%%%
\section{Carbon and nitrogen abundance anomalies}
\label{sec:cno_anomalies}

Carbon and nitrogen, and sometimes oxygen, abundance anomalies in the
O and B stars have been known since 1967 when Jaschek and Jaschek
observed nitrogen deficiencies in two early-type supergiants. In the years
following, several more objects were discovered and there existed a
certain amount of confusion as to exactly which objects were anomalous
(e.g. $\zeta$~Ori or HD~188209; see \cite{jaja:74}). \ybcite{wa:76}
re-examined the classification criteria and reviewed the material and
we have mostly adopted classifications from his papers, with the
exception of BD+36~4063.

There are some problems in understanding the actual cause of the
anomalies, not so much in the case of the OBNs\footnote{OBN is a
collective term for BN and ON stars. Similarly, OBC includes both BC
and OC.}, where the strong nitrogen enhancement is probably caused by
mixing of CNO processed material into the atmosphere of the star, but
there is no obvious way of creating a nitrogen deficiency in a star at
early evolutionary stages. This led Walborn to propose that the
morphologically normal O and B stars, which are by definition the
largest groups, are not necessarily \emph{chemically} normal. The sequence
would then be that the nitrogen deficient OBC stars correspond to the
natal abundances, while the morphological normal OB stars are in fact
slightly chemically evolved, and the OBN stars are the extreme cases.

This picture is very pleasing and simple; it does, however, have some
major shortcomings. The existence of both main-sequence and supergiant
OBN stars, objects that should be highly evolved if the above theory is
correct, suggests that other mechanisms also exist for bringing
processed material onto the surface of the stars. There is also the
problem that all known OBC stars are supergiants (and are evolved objects --- see
e.g. figure \ref{fig:helium:age}) but show no chemical evolution. So
the chemical enrichment cannot solely be an evolutionary process.

Even considering all these problems, the evolutionary theory is still
probably one of the strongest candidates, although current models do
not predict any abundance enhancements until a later evolutionary
stage than that which is observed (see section
\ref{sec:helium}). These results are constantly being revised,
however, and recent calculations, including rotation (which we know to
be very important in O and B stars), clearly show a much earlier
mixing than previous models (cf. \cite{me:98}).
%Meynet, cumberland lodge
It is also clear that there are other factors that might influence the
stars' surface abundaces. There is some evidence to suggest that the
anomalies reflect the natal abundances. All three Orion Belt stars (of
which only two are analysed here) were found by Walborn to be nitrogen
deficient, and we also find that both stars have relatively low (ie. solar)
helium abundances. Interestingly enough, these two stars, and HD~16429,
seem to form a little subgroup of evolved stars with near-zero mass
discrepancy (see section \ref{sec:mass_discrepancy}).

A number (possibly all) of the OBN stars are also binaries, and it is possible that,
at least for some objects, the nature of the anomalies could be caused
by the binary nature of these stars. Mass transfer, either active or
past, by an evolved companion has been suggested as a method of
bringing CNO processed material onto the surface of the stars. Of the
two ON main-sequence stars in our sample, both have been proposed as
binaries by \ybcite{bolt_rog}. They also derived orbital elements for
HD~12323, but could only say that the line profiles for HD~201345
varied in a way as to suggest that it might be a binary. This star was
later regarded as single by \ybcite{gi:86}. 
% Gies(see HP89)
Of the other ON stars in our sample, \acite{bolt_rog} found variability
in the spectrum of HD~191781, but they did not have enough data to
confirm that it is a binary. Finally, BD+36~4063 has been confirmed as a
single-lined spectroscopic binary by I. D. Howarth (private communication).

\subsection{Carbon and nitrogen line strengths}
As we lack good model predictions for the strengths of carbon and
nitrogen lines, we cannot directly determine abundances for these
elements in our data. We can, however, try to identify anomalous
objects by looking at the ratio of equivalents widths for these two
elements. There is a lack of strong, unblended carbon and nitrogen
lines, and the only two unblended candidates are
N\scr{2}~$\lambda$4630\AA\ and C\scr{2}~$\lambda$4267\AA. These lines
are very weak and difficult to measure, so we can expect large errors
on these measurements. The equivalent widths were measured at least 4
times using the data analysis package \dipso\ (\cite{dipso}), and the
results are shown in table \ref{ta:cn_ew}. The uncertainties quoted
are based on the standard deviation of repeated measurements of the
equivalent widths.

\begin{table} %%%%%%% TABLE %%%%%%%%%%%%%%%
\begin{center}
\caption{C \& N equivalent widths}
\vspace{\abovecaptionskip}
\begin{tabular}{lccc}
\hline\hline
& N\scr{2}~$\lambda$4630\AA &
  C\scr{2}~$\lambda$4267\AA &
  N\scr{2} / C\scr{2} \\
\multicolumn{1}{c}{HD/BD} &
\multicolumn{1}{c}{EW (m\AA)} &
\multicolumn{1}{c}{EW (m\AA)} & ratio\\
\hline
% Automatically generate tex table Fri Oct 16 17:14:34 BST 1998
% Created by ./00maketable.csh (/home/kws/latex/thesis/fig_odiscuss/cno)
BD +36 4063 & $133.5 \pm  2.4$ & $  4.4\pm  2.2$ & $ 30.1 \pm 14.9$ \\
   HD 10125 & $159.4 \pm  9.8$ & $ 25.8\pm  3.9$ & $  6.2 \pm  1.0$ \\
   HD 12323 & $ 88.6 \pm  4.7$ & $  6.0\pm  2.1$ & $ 14.7 \pm  5.3$ \\
   HD 13745 & $166.0 \pm  7.0$ & $ 53.6\pm  2.7$ & $  3.1 \pm  0.2$ \\
   HD 16429 & $102.6 \pm 15.3$ & $ 21.0\pm  4.2$ & $  4.9 \pm  1.2$ \\
   HD 30614 & $ 91.7 \pm  6.2$ & $ 37.8\pm  3.3$ & $  2.4 \pm  0.3$ \\
   HD 34078 & $105.8 \pm  1.3$ & $ 65.2\pm  0.8$ & $  1.6 \pm  0.0$ \\
   HD 36486 & $ 95.2 \pm 13.1$ & $ 35.5\pm  4.5$ & $  2.7 \pm  0.5$ \\
   HD 37742 & $ 97.4 \pm  5.6$ & $ 44.2\pm  5.4$ & $  2.2 \pm  0.3$ \\
  HD 188209 & $123.5 \pm  4.0$ & $ 39.0\pm  3.5$ & $  3.2 \pm  0.3$ \\
  HD 189957 & $137.7 \pm  8.8$ & $ 73.2\pm  3.0$ & $  1.9 \pm  0.1$ \\
  HD 191781 & $211.2 \pm  9.1$ & $  9.5\pm  6.1$ & $ 22.3 \pm 14.5$ \\
  HD 194280 & $176.2 \pm  6.5$ & $ 81.2\pm 10.7$ & $  2.2 \pm  0.3$ \\
  HD 195592 & $198.0 \pm  7.0$ & $ 18.8\pm  2.6$ & $ 10.5 \pm  1.5$ \\
  HD 201345 & $103.8 \pm  2.7$ & $  6.1\pm  2.2$ & $ 16.9 \pm  6.2$ \\
  HD 202124 & $ 74.8 \pm  8.1$ & $ 13.7\pm  2.9$ & $  5.5 \pm  1.3$ \\
  HD 207198 & $ 55.8 \pm  4.6$ & $ 32.0\pm  5.9$ & $  1.7 \pm  0.4$ \\
  HD 209975 & $ 91.0 \pm  7.3$ & $ 42.2\pm  4.3$ & $  2.2 \pm  0.3$ \\
  HD 210809 & $ 73.2 \pm  3.8$ & $  8.7\pm  2.7$ & $  8.4 \pm  2.6$ \\
  HD 214680 & $ 66.2 \pm  1.6$ & $ 38.5\pm  2.4$ & $  1.7 \pm  0.1$ \\
  HD 218195 & $ 47.0 \pm  3.6$ & $ 31.3\pm  3.0$ & $  1.5 \pm  0.2$ \\
  HD 218915 & $114.8 \pm  2.5$ & $ 20.6\pm  4.7$ & $  5.6 \pm  1.3$ \\
  HD 225160 & $ 11.5 \pm  3.1$ & $  6.7\pm  3.1$ & $  1.7 \pm  0.9$ \\
\hline\hline
\end{tabular}
\end{center}
\label{ta:cn_ew}
\end{table} %%%%%%%%%%%%%%%%%%%%%%% END TABLE %%%%%%%%%%%%%%%%%%

Since we do not know the expected behaviour of this ratio, we have
plotted the N/C line ratio as functions of both \teff\ and \logg\ in figure
\ref{fig:cno:tg}. What we can clearly see from these figures are that
the ON stars all have far larger N/C line ratios than the other stars.
There also appears to be a subgroup of slightly higher ($\approx 5$)
ratios, but by comparing with figure \ref{fig:cno:behaviour} it
appears that this might be a luminosity effect. Most of these stars
seem to be at the low-\logg\ end of the distribution for their
temperature. The only exception to this is possibly HD~10125 at
32.5~kK, although these data have to be taken as they were intended;
simply as an illustration of the OBN/OBC anomaly.

\begin{figure} %%%%%%%%%FIGURE
\epsfxsize=\figwidth
\setlength{\cen}{(\textwidth / 2) - (\epsfxsize / 2)}
\hspace{\cen}\epsfbox{fig_odiscuss/cno/nc_t.eps}

\epsfxsize=\figwidth
\setlength{\cen}{(\textwidth / 2) - (\epsfxsize / 2)}
\hspace{\cen}\epsfbox{fig_odiscuss/cno/nc_logg.eps}

\caption[N\textsc{ii}~$\lambda$4630\AA\ /
C\textsc{ii}~$\lambda$4267\AA\ against \teff\ and \logg]
{\fcfont N\textsc{ii}~$\lambda$4630\AA\ /
C\textsc{ii}~$\lambda$4267\AA\ against \teff\ (\textbf{top}) and \logg\
(\textbf{bottom})}
\label{fig:cno:tg}
\end{figure}   %%%%%%%%%%%%%%%

\begin{figure} %%%%%%%%%FIGURE
\epsfxsize=\figwidth
\setlength{\cen}{(\textwidth / 2) - (\epsfxsize / 2)}
\hspace{\cen}\epsfbox{fig_odiscuss/cno/nc_behaviour.eps}
\caption[Dependence N\textsc{ii}~$\lambda$4630\AA\ /
C\textsc{ii}~$\lambda$4267\AA\ of stellar parameters]
{\fcfont Dependence of N\textsc{ii}~$\lambda$4630\AA\ /
C\textsc{ii}~$\lambda$4267\AA\ ratio on stellar parameters \teff\ and
\logg. Symbols are scaled to represent the line-strength ratio as shown
in lower right-hand corner. The N to C ratio seems to be higher at the
low-\logg\ end of the distribution for each temperature (except in the
case of the ON stars).}
\label{fig:cno:behaviour}
\end{figure}   %%%%%%%%%%%%%%%

\begin{figure}[t] %%%%%%%%%FIGURE
\setlength{\cen}{\figwidth / 2}
\epsfxsize=\cen
\setlength{\cen}{(\textwidth / 2) - (\epsfxsize / 2)}
\hspace{\cen}\epsfbox{fig_odiscuss/cno/nc_y.eps}
\caption[N\textsc{ii}~$\lambda$4630\AA\ / C\textsc{ii}~$\lambda$4267\AA\ against $y$]
{\fcfont N\textsc{ii}~$\lambda$4630\AA\ / C\textsc{ii}~$\lambda$4267\AA\ against $y$}
\label{fig:cno:y}
\end{figure}   %%%%%%%%%%%%%%%

\begin{figure} %%%%%%%%%FIGURE
\epsfxsize=\figwidth
\setlength{\cen}{(\textwidth / 2) - (\epsfxsize / 2)}
\hspace{\cen}\epsfbox{fig_odiscuss/cno/nc_age.eps}
\caption[N\textsc{ii}~$\lambda$4630\AA\ / C~\textsc{ii}~$\lambda$4267\AA\ against age]
{\fcfont N\textsc{ii}~$\lambda$4630\AA\ / C~\textsc{ii}~$\lambda$4267\AA\ 
against evolutionary age}
\label{fig:cno:age}
\end{figure}   %%%%%%%%%%%%%%%














