\chapter{Introduction}
\label{ch:intro}

This thesis discusses the spectroscopic properties of hot, luminous
stars.  The term `hot stars' is used to mean all normal stars of
spectral types O and B (OB for short); `luminous' means luminosity classes
I--V. The terms `hot stars', `early-type stars' and `OB
stars' will be used synonymously in this work. The thesis is
roughly divided into two parts; part one (chapters 
\ref{ch:casestudy} -- \ref{ch:odiscuss}) will deal with analyses
of high-resolution, high signal-to-noise optical
observations of 23 late-type O stars, and a discussion of the significance
of the results. Part two (chapter
\ref{ch:xcorr}) will investigate the line broadening parameter, \vsini,
by cross-correlating high-resolution, short-wavelength \iue\ data, and
expand the work of \ybcite{ho:xcorr} to include more stars, especially
B-stars.

The properties of early-type stars are of interest for a variety of
reasons. These stars live relatively short, but violent, lives. During
the main stages of their life cycles, they lose a significant amount
of mass through their stellar winds (e.g. \cite{pu:96}) and this mass
loss can affect their evolution substantially
(\cite{1987A&A...182..243M}).  The stellar winds can also have a great
effect on the interstellar medium (ISM) surrounding the star, both
kinetically heating it and enriching it with nuclear processed material
(e.g. \cite{ca:93}). At the end of their lives, they may inject vast
amounts of enriched material into the ISM as Wolf-Rayet stars and/or
supernovae.

Lately, thanks to the new generation of large telescopes and sensitive
detectors, observations of single early-type stars in other galaxies
have become more and more common. It has been possible to investigate,
among other things, metallicity gradients in nearby galaxies
(\cite{mo:97}; \cite{mo:98}). Also, very recently, many high-redshift
faint, blue galaxies were discovered in the Hubble Deep
Field (\cite{hdf}). Although it will, probably, never be possible to
observe individual stars in these galaxies, the early-type stars are
apparently a major contributor to the galaxies' overall
luminosity. If we could learn more about the early-type
stars in these galaxies, we should be able to tell a lot about the
conditions in the early universe, the early populations of stars, and
the star-forming history of the early universe (e.g. \cite{fr:98}).

%%%%%%%%%%%%%%%%%%%%%%%%%%%%%%%%%%%%%%%%%%
%%%%%%%% SECTION
%%%%%%%%%%%%%%%%%%%%%%%%%%%%%%%%%%%%%%%%%%
\section{Spectroscopic properties}

The optical spectra of O and B stars are dominated by absorption lines
of neutral hydrogen (\hi\ in spectroscopic notation) and neutral
helium (\hei). For the O stars, singly ionised helium (\heii)
also becomes increasingly important, increasing in strength towards
the earlier (hotter) types; whereas \hei\ (which is at its strongest
around B2~--~B3) slowly decreases in strength towards hotter types. The
\heii\ lines, having only one electron orbiting the nucleus, behave
spectroscopically very much like those of neutral hydrogen, and the
lines are often grouped into the Pickering series (analogous to the
Bracket, $n \rightarrow 3$, series of hydrogen) with lines at
\lam\lam5411, 4860, 4541, 4339, 4200, 4100, 4025 and 3968\AA\ etc.,
and the Fowler series (analogous to Paschen, $n \rightarrow 4$, series
of hydrogen) with lines at \lam\lam4686, 3204\AA\ etc.

Several other atomic species can also be observed in the optical; in
particular, carbon, oxygen, nitrogen and silicon. At later spectral
types, especially in the late-B stars, other metals become
increasingly more important. For the highest-luminosity stars, several
of these lines can sometimes be seen in emission. There are also many
`dynamic' features in the spectra. P Cygni profiles, signatures of
mass-loss and stellar winds, are often observed, especially in the
Balmer-$\alpha$ (H$\alpha$) line. Wind effects are even more
obvious in the UV, often seen in the resonance doublets, C\scr{4}
$\lambda\lambda$1548, 1551\AA, Si\scr{4} $\lambda\lambda$1394,
1402\AA, and Ni\scr{5} $\lambda\lambda$1249, 1242\AA. These lines are
extremely sensitive to the effects of stellar wind, and are often used
to `probe' these effects (e.g. \cite{ho:winds}).


%%%%%%%%%%%%%%%%%%%%%%%%%%%%%%%%%%%%%%%%%%
%%%%%%%% SECTION
%%%%%%%%%%%%%%%%%%%%%%%%%%%%%%%%%%%%%%%%%%
\section{Spectroscopic classifications}

With a huge number of observable stars in the Galaxy (a number which
is constantly growing thanks to technological advances for both detectors
and telescopes), it is clear that the stars somehow need to be grouped
together to make a more manageable dataset. As an example of the
number of observable stars, the original version of the Henry Draper
(HD; \cite{hdcat}) catalogue contained 225,300 stars. The Guide Star
Catalogue\footnote{\cite{1990AJ.....99.2019L}; \cite{1990AJ.....99.2059R};
\cite{1990AJ.....99.2082J}}, contains the positions of
over $1.9\times10^7$ stars between magnitudes six and fifteen. To make
interpretation of this number of stars more manageable, the stars are
grouped into spectral types.

\subsection{Harvard and Yerkes classifications}
The basis for most `modern' spectral types is rooted in the development
of the letter based classifications of the Harvard scheme. At first,
the stars were divided into spectral classes symbolized by capital
letters (introduced by Edward C. Pickering), arranged in alphabetical
%Edward Charles Pickering
order, starting with class A, which contained stars with only hydrogen
series visible (and interstellar Ca~K lines) at the low dispersion
used. In class B, the lines in class A were supplemented with
\hei~\lam4026\AA\ and \lam4471\AA. Class C contained stars of spectral
types A and B but with \hd\ and \hg\ observed as double lines. The order
of the classifications was later revised, and some types (like class
C above) were suppressed. The main groups were later complemented by
Annie J. Cannon with subdivisions denoted by numbers from zero to
nine, except for the O and M stars. For these groups, lower case
letters were used to denote the subtypes.
%Annie Jump Cannon
The final order of the classifications is:
\begin{quote}
O -- B -- A -- F -- G -- K -- M
\end{quote}
This order is related to temperature, with O being hottest, and M
coldest. Each spectral type was further divided into subtypes as
indicated in table \ref{ta:intro:spectypes}.

\begin{table} %%%%%%%%%% TABLE
\begin{center}
\caption{Spectral types used in the Harvard system}
\vspace{\abovecaptionskip}
\label{ta:intro:spectypes}
\begin{tabular}{cp{5cm}}
\hline\hline
O & a, b, c, d, e \\
B & 1, 2, 3, 5, 8, 9 \\
A & 0, 2, 3, 5 \\
F & 0, 2, 5 ,8 \\
G & 0, 5 \\
K & 0, 2, 5 \\
M & a, b, c, d \\
\hline\hline
\end{tabular}
\end{center}

\tref{Adopted from \ybcite{jaja:book}}
\end{table} %%%%%%%%%

The Harvard system was used to classify the stars listed in the HD
catalogue mentioned above. Roughly 99\% of the stars in the catalogue
(about $2.2\times10^5$) are classified into no more than 30 types.

As early as 1914, \acite{ad:14} had shown that certain lines in a
stellar spectrum were sensitive to the luminosity of the star. The
Yerkes system takes the Harvard classifications and introduces
luminosity classes from I--V, where I is the brightest (see table
\ref{ta:intro:lumclasses}). The Yerkes system is also often referred
to as the MK system, after W. W. Morgan and P. C. Keenan, two of the
principal scientists who worked on developing the scheme. In the
Yerkes scheme, stars are classified from slit spectrograms at a
typical dispersion of 115\AA/mm, in contrast to the Harvard scheme
which used objective prism plates with many low-dispersion stellar
spectra per plate. The Yerkes system also uses subtype numbers for the
O stars, from O4 to O9. O3, O9.5, O9.7 are later additions, as are
intermediate types for the B stars (like B0.5 and B9.5).

\begin{table} %%%%%%%%%% TABLE
\begin{center}
\caption{Luminosity classes in the Yerkes system}
\vspace{\abovecaptionskip}
\label{ta:intro:lumclasses}
\begin{tabular}{cp{4cm}l}
\hline\hline
Symbol & Name & Subdivisions \\
\hline
\scr{1}	& Supergiant (or hypergiant) & \scr{1}, \scr{1}a, \scr{1}ab, \scr{1}b \\
\scr{2} & Bright Giant	& \\
\scr{3} & Giant	& \scr{2} --\scr{3}, \scr{3}a, \scr{3}ab, \scr{3}b, \scr{3} --\scr{4} \\
\scr{4} & Subgiant \\
\scr{5} & Dwarf \\
\hline\hline
\end{tabular}
\end{center}

\tref{Adopted from \ybcite{jaja:book}}
\end{table} %%%%%%%%%

\subsection{`Modern' classifications}

Walborn (\cite*{wa:70}, \cite*{wa:71supp} and subsequent papers)
re-classified a large number of O and early-B type
stars. He used several lines of \hi, 
\hei, \heii, Si\scr{3} and Si\scr{4} to derive spectral types and
luminosity classes simultaneously. Walborn also extended the MK
subtypes, and introduced O9.7, B0.2 and B0.7. We have chosen to adopt
Walborn's spectral types, where these are available, throughout this
thesis. Walborn has also studied the UV spectral morphology of
early-type stars (\cite{wa:uv1}; \cite{wa:uv2};  \cite{wa:uv3};
\cite{wa:uv4}).

With the development of photoelectric plate-measuring devices and
spectral tracers, it became possible to classify stars from actual
measurements of lines, as opposed to the by-eye comparisons used in
the Harvard and Yerkes/MK schemes. Equivalent width classifications
of early-type stars have been performed by, among others,
\ybcite{co:71} (also \cite{co:v} and \cite{co:vi}). They chose to base
their classifications solely on the equivalent-width ratios of the
\hei\ lines \lam4471\AA\ to \lam4541\AA\ (for spectral type) and
Si\scr{4}~\lam4089\AA\ to
\hei~\lam4143\AA\ (for luminosity class). Although these
classifications are generally of good quality, they do not reflect
chemical peculiarities as well as Walborn's classifications.

\ybcite{bi:82} performed an in-depth analysis of equivalent widths of
the N\scr{3}~\lam4514\AA\ line in a large number of O-type stars. They
found a good agreement between the ON phenomenon and the strength of
the N\scr{3} line. They also concluded that there is a continuous
distribution of N abundances, and not a marked discontinuity between
normal O-type stars and the ON stars (see section \ref{sec:intro_obn}
for more about the ON stars). \ybcite{1988A&AS...76..427M} and
\ybcite{ma:89} used equivalent width measurements to derive spectral
types that were consistent with both the results of Conti et al. and
of \acite{bi:82}, and therefore also consistently treat the OBN/OBC
phenomenon. The spectral type of
\acite{ma:89} was used for the star BD+36~4063 (ON9.7~I) --- the only
star in the WHT sample not classified by Walborn (see chapter \ref{ch:obs}).

\subsection{OBN/OBC stars}
\label{sec:intro_obn}
Early-type stars with unusually weak nitrogen spectra were first observed by
\ybcite{jaja:67}, and the opposite phenomenon, stars with unusually strong
N spectra, was first reported by \ybcite{wa:70}. The classifications
OBN/OBC were first suggested by \ybcite{wa:76} to describe stars with
strong nitrogen and/or weak carbon and weak nitrogen and/or strong
carbon respectively. These stars are also sometimes referred to as CNO
stars (e.g. \cite{jaja:74}).

Several scenarios have been suggested to explain these anomalies;

\begin{enumerate}
\item	\ybcite{wa:76} proposed that the spectroscopic sequence OC
	$\rightarrow$ O $\rightarrow$ ON, corresponding to a gradual
	increase in nitrogen abundance and a depletion of carbon, is
	an evolutionary sequence. Mixing of CNO-cycled material into
	the photosphere could explain the observed abundances if (i)
	the OBC supergiants show their `natal' abundances, (ii)
	morphologically normal OB supergiants all show some mixing, and
	(iii) OBN stars are at an advanced state of mixing.
	\label{item:cno:evol}

\item	Walborn also suggested that the abundance abnormalities could
	merely reflect the abundances of the material the stars were
	born from. There is some evidence that this could be the
	case. The strongest (and possibly only) case is with the Orion
	Belt stars, which all show a similar degree of nitrogen
	deficiency.

\item	\ybcite{bolt_rog} found that a significant fraction (possibly all) 
	of the OBN stars were in fact binaries, whereas they could
	find no indication of binarity in any of the OBC stars (this
	has later been confirmed in a study by
	\cite{1988ApJS...68..319L}). It is possible that at least some
	of the OBN stars could come from mass-transfer, tidal mixing,
	or some other effect caused by a binary companion.
\end{enumerate}

Further to scenario \ref{item:cno:evol} above, `classical'
evolutionary models for massive stars (e.g. \cite{schaller:92}) do not
predict the exposure of CNO processed material until the stars are at
very advanced stages of stellar evolution, and not at the evolutionary
stage where the OB stars appear to be. However, recent models taking
into account the effects of rotation on the stellar evolution
(\cite{ma:98}; \cite{me:98}) do predict a possible increase in the
surface N/C abundance ratio at an evolutionary stage that is
compatible with this scenario, at least for the OBN supergiants.

If the nitrogen and carbon anomalies arise from the exposure of CNO
processed material in the stars, then other by-products of the CNO
cycle should also be apparent. In chapters \ref{ch:casestudy} --
\ref{ch:analyses} we will derive helium abundances for several normal
and ON/OC stars, and if CNO material is exposed we would expect to
see increased (above solar) helium abundance for the ON stars, with
possibly a helium abundance slightly above solar in the morphologically
normal O stars. 


\subsection{Be stars}
\label{sec:intro:be}

A different group of `peculiar' hot stars is the Be stars. These stars
are commonly defined as ``all non-supergiant B-stars whose spectra
have, or have had, one or more Balmer lines in emission''
(\cite{sl:88}). There is also a subgroup of the Be stars, called the
shell stars, which show narrow absorption lines of several ionised
metals superimposed on the Be star spectrum. Be star spectra often
show very broad lines, implying projected equatorial rotational
velocities of 200 -- 300~\kms\ or higher (\cite{sl:82}). The spectral
features in Be stars can show sometimes dramatic variability, on
periods from tens of years for shell appearance and disappearance, to
as little as a matter of days, or even hours for line-profile
variations (e.g. \cite{1997ApJ...481..479S}). Many Be stars are part
of binary pairs; however, it appears that there are not enough
eclipsing binary Be stars to attribute the Be phenomenon exclusively
to binarity (\cite{1976IAUS...70..439P}). Recent observations in IR,
UV and X-ray have led to the following possible scenarios to explain
the observed properties of Be stars (\cite{sl:88} and references
therein):

\begin{enumerate}
\item	\emph{Rotationally-enhanced stellar-wind
	model.} Rotation can cause radiation-driven winds to be
	denser around the equator than at the poles, creating a
	disk-like structure around the equator (e.g. \cite{1993ApJ...409..429B})

\item	\emph{Spheroidal/ellipsoidal, variable mass-loss, decelerated
	Be-star model.} Variable mass outflow, caused by ``nonthermal
	subatmospheric modes'' passes through several atmospheric
	layers surrounding the photosphere. The outflow cools and is
	deaccelerated through interactions with mass flows of other
	velocities or the local stellar environment, and form a cool,
	\ha\ emitting region surrounding the star. A coupling of the
	expanding flow with rotation will cause an ellipsoidal shape
	for the emitting region (\cite{do:87}).

\item	\emph{Nonradial pulsation model.} Most of the rapidly rotating
	Be stars show rapid line profile variations, which could be
	explained by nonradial pulsations. Occasional mode-switching of
	the pulsations could provide the energy to drive the
	mass-loss events (e.g. \cite{1994IAUS..162..311B}).

\item	\emph{Interacting binary model.} The formation of the Be
	envelope is a result of evolutionary processes and
	interactions in a binary system (\cite{ha:87}).

\item	\emph{Magnetic-loop model.} A low-density wind forms a disk
	from interactions with magnetically supported plumes around a
	rapidly rotating star (\cite{un:87}).
\end{enumerate}

There is observational evidence both to support and to refute most of
these models. In chapter \ref{ch:xcorr} we will concentrate on the
rotational properties of a sample of Be stars.


%%%%%%%%%%%%%%%%%%%%%%%%%%%%%%%%%%%%%%%%%%
%%%%%%%% SECTION
%%%%%%%%%%%%%%%%%%%%%%%%%%%%%%%%%%%%%%%%%%
\section{Physical properties of hot stars}

The point has already been made that OB stars are among the hottest,
most luminous and most massive stars known (all these parameters are
of course related); however, the O stars sample considered in this
thesis is fairly limited in range%
\footnote{By choosing stars of a limited range in
\teff\ and \logg\ it should be possible to measure differential
chemical abundances relatively accurately. This will be discussed
further in chapter \ref{ch:obs}}.
%
The earliest-type star investigated
is an O8 supergiant, on the other hand, the latest-type star is the
cross-correlation template, \alplyr\, which is an A0 main-sequence
star.  Table \ref{ta:intro:params} lists some `typical' parameters for
OB stars.
\begin{table} %%%%%%%%%%% TABLE
\begin{center}
\caption{Typical parameters for OB stars}
\vspace{\abovecaptionskip}
\label{ta:intro:params}
\begin{tabular}{lrrrrrrr}
\hline\hline
& \multicolumn{2}{c}{\teff (kK)} & \multicolumn{2}{c}{$\log(L/L_\odot)$} &
\multicolumn{2}{c}{$R/R_\odot$} & $M/M_\odot$ \\
Sp. Type & \multicolumn{1}{c}{I} & \multicolumn{1}{c}{V} &
\multicolumn{1}{c}{I} & \multicolumn{1}{c}{V} &
\multicolumn{1}{c}{I} & \multicolumn{1}{c}{V} & ZAMS \\
\hline
O3	& 45.0 & 48.8 & 6.21 & 6.00 & 21 & 14 & 88 \\
O5	& 41.0 & 44.0 & 6.20 & 5.69 & 25 & 12 & 58 \\
O7	& 36.0 & 39.3 & 5.94 & 5.33 & 24 & 10 & 36 \\
O9	& 31.0 & 34.3 & 5.57 & 4.90 & 21 &  8 & 24 \\
B0	& 25.0 & 30.1 & 5.34 & 4.56 & 25 &  7 & 17 \\
B3	& 15.0 & 17.7 & 4.49 & 3.35 & 26 &  5 &  9 \\
B5 	& 13.5 & 15.3 & 4.27 & 3.09 & 25 &  5 &  6 \\
\hline
\end{tabular}
\end{center}

\tnote{The radii for supergiants are representative of luminosity
class Ib objects.}
\tref{\ybcite{ahnr} and references therein.}
\end{table} %%%%%%%%%%%%%%%%

There are several different approaches to measuring (or estimating)
the physical properties of these stars. If the stars are part of a
binary system, and the orbits of the components can be found, then the
masses of the components can be calculated. If the system also happens
to be inclined so that eclipses occur, then it is also possible to
calculate radii and luminosity ratios of the
components. Unfortunately, such systems are rare.

For stars that are hot enough to ionise the ISM surrounding them,
forming a compact H\scr{2} region, one can apply the \emph{Zanstra
method}. It is assumed that all the Lyman-continuum photons emitted by
the star are converted into Balmer series photons in the H\scr{2}
region. A measurement of the radio continuum or a Balmer line can give
an estimate of the Lyman flux emitted by the star. Comparison with
theoretical stellar models yields the effective temperature, \teff\
(e.g. \cite{hj:68} and \cite{mo:69}). This method is very sensitive
to the model predictions of the EUV flux.
%See andy-dog's thesis

A different approach is to integrate the observed flux from all the
available wavelength regions. The measured flux has to be corrected
for atmospheric and interstellar absorption, and missing wavelength
regions have to be `filled in' by predicted values from stellar
models. If the angular diameter of the star is known, say from
interferometry, then \teff\ can be calculated (e.g. \cite{code:76};
\cite{un:79}; \cite{un:82}). This technique has been shown to be
unreliable by \ybcite{hu:88} for stars earlier than O9.5.
%See andy-dog's thesis

The determination of stellar parameters through comparison of
observations to theoretical models is now usually the `standard
approach' for hot stars (see \cite{ku:90} for several examples). The
development of detectors capable of capturing high resolution, high
signal-to-noise spectra of even quite faint stars, and steady progress
in the field of theoretical modelling (helped by advances in computer
technology), have made it possible to get good agreement between
models and observations. This is also the method used in this
thesis to find effective temperatures and surface gravities of 23
O-type stars. 

%\subsection{Stellar winds and mass-loss}
%Mass loss from early-type stars was first detected in optical spectra
%(e.g. \cite{be:29}; \cite{wi:57}), but it wasn't until 

\subsection{Line broadening}
One of the most notable characteristics of the hot stars, and
especially of the O stars, is the absence of stars with narrow
absorption lines. Equivalent widths computed by non-LTE models (see chapter
\ref{ch:mod}) compare reasonably well with those measured from observations.
However, the model line profiles are consistently too narrow, and have
to be convolved with a line-broadening function before they reproduce the
observations satisfactorily. 

For any large sample of stars, even if they are all rotating, there
should at least be some stars that are observed nearly pole-on, and
thus have narrow lines. The narrowest-lined O stars observed have a
velocity in the order of 15~--~30~\kms\ (e.g. \cite{co:vii};
\cite{pe:xcorr};
\cite{ho:xcorr}). For the supergiants the case is even worse; the
minimum line broadening appears to be in the order of 50~\kms\ for an
O9 supergiant (\acite{ho:xcorr}). \acite{co:vii} and \acite{ho:xcorr}
suggested that, apart from large rotational velocities, there also
appears to be another form of line broadening taking place in the O
stars, perhaps attributed to some turbulent velocity field in the
stars' atmospheres. This turbulence also appears to be more important
in the supergiants than in the main-sequence stars (see also chapter
\ref{ch:xcorr}).

%%%%%%%%%%%%%%%%%%%%%%%%%%%%%%%%%%%%%%%%%%
%%%%%%%% SECTION
%%%%%%%%%%%%%%%%%%%%%%%%%%%%%%%%%%%%%%%%%%
\section{Main-sequence evolution}

It is difficult to know where to start when discussing massive star
evolution. \ybcite{ha:98} defines a ZAMS star as follows: ``A zero-age
main-sequence (ZAMS) star has its minimum radius, its maximum mass
(for single-star evolution), its bluest colour (or hottest effective
temperature), and its central core possesses its peak H/He''. However,
she then points out that \ybcite{pa:93} have found that as soon as a
hydrostatic, mass-accreting protostar has a mass of over $10M_\odot$,
it will start burning hydrogen, no matter what its final mass will
eventually be. Thus, when the term ZAMS is used in this thesis, it
will be in the sense of the \emph{theoretical} ZAMS --- the starting
point for stellar evolutionary models.

Early-type main-sequence stars `burn' hydrogen to helium through the CNO
bi-cycle. Although they have a large quantity of hydrogen in their
cores, the high energy-production rate quickly converts hydrogen into
helium, and the stars have a relatively short lifetime compared to low
mass stars. The evolutionary models published by \ybcite{schaller:92}
predict a core H-burning (i.e. main-sequence) lifetime of about $10^8$
years for a $5M_\odot$ star, $8\times10^6$ for $20M_\odot$ and only
$3\times10^6$ for a $60M_\odot$ star.

While burning hydrogen, the massive stars evolve slowly towards colder
temperatures in the H--R diagram, and their luminosity also increases
slightly. This means that, unlike for colder stars, even giant and
supergiant O stars are core H-burning. These stars also lose an
appreciable amount of mass during their main-sequence lifetimes. A
$40M_\odot$ star, is only predicted to contain $36M_\odot$ of material
by the time it has exhausted the hydrogen in the core, and a
$60M_\odot$ star will have lost over $12M_\odot$ through its stellar
wind. It is also possible, as mentioned earlier, that effects caused
by stellar rotation can mix the atmosphere sufficiently to bring CNO
processed material up to the surface of massive stars while they are
still on the main-sequence. In chapter \ref{ch:odiscuss} we will try
to place the O stars in the sample on the H-R diagram, and determine
their evolutionary status.


