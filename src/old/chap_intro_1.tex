\chapter{Introduction}

\note{
Start out by putting things into context. Briefly: 1) What are OB stars,
2) Where are OB stars, 3) How many of the stars are OB stars, 4) What do
they do for a living. Importance of OB stars (you know, all that stuff
about them being massive and with outflows and then they go kaboom and
spew all this stuff into space - the standard stuff).
}

\paragraph{}
In this work we will look at some of the properties of O and B stars 
(OB stars). These
stars are amongst the most massive stars in the Universe. Although they are
outnumbered by the far more numerous low mass stars, they constitute a
significant proportion of the luminous mass in our Galaxy. 

\paragraph{}
The OB stars are characterised by their high masses, temperatures and
luminosities. To maintain their high luminosity, they burn up their fuel
which consist primarily of hydrogen, at a very high pace. This leads to
a relatively short lifetime compared to the low mass stars. When the hydrogen
is finished they will start burning helium and possibly even heavier elements.
During their entire lifetime they ``blow off'' material into space through 
their stellar winds. And at the end of their lifetime, the very high mass 
stars will end
their lives in brilliant supernova explosions, expelling huge amounts of
nuclear processed material into space.
It thus follows
that a good understanding of the processes that take place in and around
these massive stars is needed if we want to try to understand the 
dynamical properties of the Galaxy around us.

\paragraph{}
Their high luminosities mean that we can observe the OB stars to
large distances. Individual O star spectra have been obtained from O
stars in  globular clusters \smnote{ref}, the Magellanic Clouds\smnote{ref}
 and other galaxies\smnote{ref}. A spectroscopic analysis
of these objects have given us a lot more knowledge about the
distances in the Univserse\smnote{ref}, and about metallicity
gradients both in our and other galaxies\smnote{ref}. However, all
these results depend on the theories and models used, and there are still many
aspects of these stars that are poorly understood. A more detail study
of nearby OB stars might help to answer some of the many questions that
still remain unanswered. This is what we will concentrate on in this work.

%%%%%%%%%%%

\section{Distribution or `\emph{Where do we see these things?}'}

\note{Numbers and number densities. Distribution in space. OB associations.}

\paragraph{}
We have already mentioned that the OB stars are very bright
(see\smnote{table in next section}) 
and can be seen to large distances. \ybcite{ga:imf} have compiled
a catalogue of O stars with known spectral types and luminosity classes and 
with $UBV$ photometric data. They counted the stars in shells
centered on the
Sun and found that the number of stars increased as the distance to
the power 2.3, out to 2.8 kpc. This suggests that the sample is fairly
complete out to this distance, although they comment that there are
uncertainties due to non-uniformities in the  distribution of stars in
the Galaxy.

\paragraph{}
The distribution of OB stars is in fact far from uniform. \\ \smnote{
Historical - who discovered that the Galaxy was spiral?} Early this
century \ybcite{ka:parallax} and \ybcite{ed:movements} noted on the
distribution of the `helium stars' as they were called then. 
In 1947, Ambartzumian \nocite{am:evolution} described the
observed groups of OB stars as associations and hypothised that they were
gravitationally unbound. This was confirmed when \ybcite{bl:perOB}
discovered that the OB association Perseus OB2 is rapidly expanding. It is
important to emphasize the size of the OB associations. They are much
larger than open star clusters, often spanning many tens of parsecs in
diameter. They often consist of several stellar subgroups, all
with different ages and kinematics (\cite{ga:OBreview}).
\smnote{Maybe I should discuss/explain a bit more?}

\begin{figure}
  \label{fig:OBdist}
  \epsfxsize=\figwidth
  \epsfbox{fig_intro/fig_OBdist.gs}
  \caption{The distribution over a section of the galactic plane 
	of O and early B-type for which both
	photometry and spectral types have been published
	(\protect\cite{ga:OBreview}).}
\end{figure}

\paragraph{}
In figure \ref{fig:OBdist} it is clear that although there are cleary
concentrations of stars, there are still many stars that do not appear
to belong to any association. \ybcite{co:viii} discuss the radial velocities
of a large sample of over 200 O-type stars. They find that even after
accounting for stars with variable radial velocities and radial velocity
effects caused by mass outflow from the stars, there are still some stars
with a peculiar radial velocity. Of their sample, they identify
six stars with a peculiar velocity larger than 40\kms. These probably
belong to the ``runaway'' stars.

\paragraph{}
The runaway OB stars are generally defined as 
those stars with peculiar velocities
of more than 3 sigma from the mean, or about 30 to 40\kms 
(\cite{bl:runaway,un:monograph}).
They were first recognised by \ybcite{bl:runaway}, who suggested that they
were ejected from their birthplace when one of the stars in a binary pair goes
supernova. A different scenario was suggested by \ybcite{po:67}. IN this
scenario, stars get ejected from the parent clusters by gravitational
interactions with other stars in the cluster. There exists observational
evidence for and against both scenarios (for example 
\cite{le:runawayII,st:runaway,mo:hipparcosI}),
and it is possible that both mechanisms take place. There is also a much
higher number of runaway O-type stars than there are runaway B stars, 
although the actual number is also a matter of some discussion 
(\cite{st:runaway}).


\section{Physical properties}
Parameters - masses, temperatures, surface gravities + rotation,
	mass loss.

\section{Evolution}
Maybe this sould be before the parameters bit?

\section{Spectra of hot stars}
Classification, with emphasis on how parameters have been
determined until now. Start with line ratios and then go on
to cover single line modelling++. \\
Spectral features, both in optical and UV.

\section{Spectral Peculiarities}
Subgroups and peculiar stars.
