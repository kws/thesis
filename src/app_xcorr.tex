\chapter{Line broadening parameters}
\label{app:xcorr}

The results from the cross-correlation analyses discussed in chapter
\ref{ch:xcorr} are summarised in the table on the following
pages. Column one gives the HD catalogue number, columns two and three
give spectral type and reference respectively, column four gives the
Bright Star Catalogue (BSC) spectral type. Column five shows stars
flagged as spectroscopic binaries (SB) in the BSC. A question mark
denotes a possible spectroscopic binary, 1 or 2 are for single or
double lined spectroscopic binaries respectively, and an O marks
systems where orbital data exist (all according to the BSC). Column six
gives the observation catalogue number (SWP -- Short Wavelength Prime
camera), and columns seven through ten give the corrected \vsini\
found using each of the four template spectra (see section
\ref{sec:xcorr:cal}). Velocities set in \textbf{bold} indicate the
`strongest' cross-correlation function (found from the integrated
area under the c.c.f.). Column 11 gives the adopted velocity, which,
when there are multiple spectra, is the mean of the corrected
velocites from the strongest CCFs. The spectral type references are:

\vspace{5mm}
\begin{tabular}{l@{ -- }l}
GG	& \ybcite{gg} \\
HS	& \ybcite{hs} \\
L68 	& \ybcite{l68} \\
W71	& \ybcite{wa:71supp} \\
W76	& \ybcite{wa:76} \\
\end{tabular}
\vspace{5mm}

An asterix (*) in any column indicates that a more extensive entry for
the star is to be found at the end of the table. ``\ldots'' indicates
observations that failed to give a significant cross-correlation peak,
or where the automatic routines could not complete (this would
sometimes occur if the cross-correlation peak was very narrow or
extremely wide). The hypertext table on the CD-ROM can, in some cases,
reveal why the analysis failed for a given star.

\newpage
\scriptsize
\begin{longtable}{r lcl c r rrrr r}
\multicolumn{11}{c}{Results of line broadening analysis}\\
\hline\hline
& & & \multicolumn{2}{c}{B.S.C.} & & \multicolumn{5}{c}{\vsini\ (\kms)} \\
\multicolumn{1}{c}{HD} &
\multicolumn{1}{c}{Spectral Type} & 
\multicolumn{1}{c}{Ref} & 
\multicolumn{1}{c}{Spectral Type} &
\multicolumn{1}{c}{SB} & 
\multicolumn{1}{c}{SWP} & 
$\tau$Sco & $\iota$Her & $\pi$Cet & $\alpha$Lyr & Ad.\\
\hline
\endfirsthead
\multicolumn{11}{c}{Results of line broadening analysis --- \emph{continued}}\\
\hline\hline
& & & \multicolumn{2}{c}{B.S.C.} & & \multicolumn{5}{c}{\vsini\ (\kms)} \\
\multicolumn{1}{c}{HD} &
\multicolumn{1}{c}{Spectral Type} & 
\multicolumn{1}{c}{Ref} & 
\multicolumn{1}{c}{Spectral Type} & 
\multicolumn{1}{c}{SB} & 
\multicolumn{1}{c}{SWP} & 
$\tau$Sco & $\iota$Her & $\pi$Cet & $\alpha$Lyr & Ad.\\
\hline
\endhead
\multicolumn{10}{r}{\emph{continues next page\ldots}}
\endfoot
\hline\hline
\endlastfoot
   144 &            &     & B9III      &  &  15936 &\ldots          &{100}           &\textbf{89}     &{100}           & 94\\
       &            &     &            &  &  29402 &\ldots          &{127}           &\textbf{98}     &{106}           &\\
   315 &            &     & B8IIIpSi   &  &  51670 &\ldots          &{138}           &\textbf{77}     &{56}            & 77\\
   358 &  kB9hB\ldots * &  GG & B8IVpMnHg  & O&  14954 &\ldots          &{54}            &\textbf{44}     &{34}            & 37\\
       &            &     &            &  &  40448 &\ldots          &{53}            &\textbf{30}     &{28}            &\\
   886 &  B2 IV     &  HS & B2IV       & O&   5260 &\textbf{$\leq$ 20}&\ldots          &{46}            &{50}            &$\leq$ 20\\
       &            &     &            &  &  52777 &\textbf{24}     &\ldots          &{43}            &{41}            &\\
       &            &     &            &  &  52783 &\textbf{$\leq$ 20}&\ldots          &{45}            &{47}            &\\
  1279 &  B7 II+    &  GG & B7III      &  &   6756 &\ldots          &{25}            &\textbf{$\leq$ 20}&{$\leq$ 20}     &$\leq$ 20\\
  1909 &            &     & B9IVMn     & 1&  23106 &\ldots          &{23}            &\textbf{$\leq$ 20}&{$\leq$ 20}     &$\leq$ 20\\
  1976 &  B5 IV     & L68 & B5IV       & O&   6250 &{107}           &\textbf{115}    &{118}           &{121}           &115\\
  2626 &            &     & B9IIIn     &  &  46894 &\ldots          &{263}           &\textbf{231}    &{245}           &231\\
  2884 &  B9 IV     &  GG & B9V        &SB&  53043 &\ldots          &{201}           &{135}           &\textbf{129}    &129\\
  2905 &  BC0.7 Ia var & W76 & B1Ia       &SB&  14939 &\textbf{95}     &{79}            &{61}            &\ldots          & 90\\
       &            &     &            &  &  40359 &\textbf{89}     &{82}            &{68}            &\ldots          &\\
       &            &     &            &  &  54038 &\textbf{87}     &{83}            &{58}            &\ldots          &\\
  3240 &  B7 III    & L68 & B7III      &  &  53240 &{78}            &{68}            &\textbf{53}     &{51}            & 53\\
  3360 &  B2 IV     & L68 & B2IV       & ?&  10057 &\textbf{30}     &\ldots          &{46}            &{49}            & 27\\
       &            &     &            &  &  53874 &{26}            &\textbf{24}     &{50}            &{42}            &\\
  3369 &  B5 V      & L68 & B5V        &1O&   9141 &{80}            &\textbf{39}     &{60}            &{49}            & 39\\
  3379 &  B2.5 IV   & L68 & B2.5IV     &SB&  20584 &{67}            &\textbf{52}     &{66}            &{52}            & 52\\
  3901 &  B2.5 V    & L68 & B2V        & 1&  52659 &{143}           &\textbf{151}    &{133}           &{138}           &151\\
  4142 &  B4 V      & L68 & B5V        &SB&  14612 &{167}           &{148}           &\textbf{157}    &{147}           &157\\
  4180 &  B5 III    & L68 & B5IIIe     &1O&  20436 &{117}           &\textbf{160}    &{178}           &{171}           &160\\
  4727 &  B5 V      & L68 & B5V+F8V    &2O&  51960 &{45}            &\textbf{25}     &{24}            &{$\leq$ 20}     & 25\\
  5394 &  B0.5 IVe  & L68 & B0IVe      &SB&  17093 &\textbf{308}    &{321}           &\ldots          &\ldots          &276\\
       &            &     &            &  &  53910 &\textbf{268}    &{242}           &{220}           &\ldots          &\\
       &            &     &            &  &  56621 &\textbf{251}    &{234}           &\ldots          &\ldots          &\\
  5737 &  B7 II+    &  GG & B7IIIp     & O&  14984 &\ldots          &{24}            &\textbf{21}     &{$\leq$ 20}     & 22\\
       &            &     &            &  &  32137 &{30}            &{24}            &\textbf{22}     &{$\leq$ 20}     &\\
  6811 &            &     & B7Ve       &  &  20437 &{74}            &{67}            &\textbf{69}     &{56}            & 69\\
  6882 &            &     & B6V+B9V    &2O&  16499 &\ldots          &\ldots          &\ldots          &\ldots          &\ldots\\
       &            &     &            &  &  16512 &\ldots          &\ldots          &\ldots          &\ldots          &\\
  7374 &  B8 IV     &  GG & B8III      &  &  18972 &\ldots          &{24}            &\textbf{$\leq$ 20}&{$\leq$ 20}     &$\leq$ 20\\
  9996 &            &     & B9pCrEu    &1O&  18437 &\ldots          &\ldots          &\textbf{22}     &{$\leq$ 20}     & 22\\
 10390 &            &     & B9IV-V     &  &  18023 &\ldots          &{48}            &{44}            &\textbf{48}     & 48\\
 10516 &  B2 Vpe    & L68 & B2Vep      &2O&  14690 &{255}           &\textbf{376}    &{319}           &\ldots          &323\\
       &            &     &            &  &  18104 &{280}           &\textbf{270}    &{276}           &\ldots          &\\
 10982 &            &     & B9.5V      &SB&   6814 &\ldots          &{39}            &{29}            &\textbf{30}     & 30\\
 11241 &  B1.5 V    & L68 & B1.5V      & O&  49469 &{141}           &\textbf{161}    &{176}           &\ldots          &161\\
 11415 &  B3 Vp     & L68 & B3III      &  &  32615 &{50}            &\textbf{39}     &{45}            &{30}            & 39\\
 12301 &  B8 Ib     & L68 & B8Ib       &  &   4265 &{50}            &{36}            &\textbf{27}     &{$\leq$ 20}     & 27\\
 12767 &  kB8hB\ldots * &  GG & B9.5pSi    &  &  51130 &\ldots          &{50}            &\textbf{37}     &{$\leq$ 20}     & 37\\
 13267 &  B5 Ia     & L68 & B5Ia       &  &  51892 &{76}            &\textbf{44}     &{40}            &{34}            & 44\\
 13294 &  B9 IVn    &  GG & B9V        &  &  40511 &\ldots          &{175}           &{123}           &\textbf{128}    &128\\
 13709 &            &     & B9V        &  &  55256 &\ldots          &{300}           &{220}           &\textbf{210}    &210\\
 14228 &  B8 IV     &  GG & B8V-IV     &  &  51597 &\ldots          &{244}           &\textbf{214}    &{202}           &214\\
 14818 &  B2 Ia     & L68 & B2Ia       &  &   9416 &{83}            &\textbf{63}     &{45}            &\ldots          & 63\\
 14951 &  B7 IV     & L68 & B7IV       &SB&   5605 &\ldots          &{132}           &\textbf{117}    &{121}           &117\\
 15130 &            &     & B9.5Vn     &  &  18007 &\ldots          &{237}           &{185}           &\textbf{180}    &180\\
 15318 &            &     & B9III      & ?&  48525 &\ldots          &{48}            &{44}            &\textbf{47}     & 47\\
 15371 &  B5 IV     &  HS & B5IV       & ?&  21046 &{28}            &{22}            &\textbf{$\leq$ 20}&{$\leq$ 20}     &$\leq$ 20\\
 16582 &  B2 IV     & L68 & B2IV       &SB&   4492 &\textbf{$\leq$ 20}&\ldots          &{55}            &{31}            &$\leq$ 20\\
 16978 &  B9 Va     &  GG & B9V        &  &  24455 &\ldots          &{101}           &{82}            &\textbf{83}     & 83\\
 17036 &  B8 IV     &  GG & B9Vn       &  &  46700 &\ldots          &{195}           &\textbf{180}    &{162}           &180\\
       &            &     &            &  &  46760 &\ldots          &\ldots          &\ldots          &\ldots          &\\
 17081 &  B7 V      & L68 & B7V        &SB&  16256 &{$\leq$ 20}     &{25}            &\textbf{$\leq$ 20}&{$\leq$ 20}     &$\leq$ 20\\
       &            &     &            &  &  32303 &{21}            &\ldots          &\textbf{$\leq$ 20}&{$\leq$ 20}     &\\
       &            &     &            &  &  32304 &{25}            &{26}            &\textbf{$\leq$ 20}&{$\leq$ 20}     &\\
 17543 &            &     & B6V        & O&  42210 &{82}            &{64}            &\textbf{56}     &{47}            & 56\\
 17573 &  B8 Vn     &  GG & B8Vn       &SB&  40510 &\ldots          &{193}           &\textbf{175}    &{157}           &175\\
 17769 &  B7 V      & L68 & B7V        &  &  18008 &{122}           &{143}           &\textbf{125}    &{137}           &125\\
 18296 &            &     & B9pSi      &SB&   3154 &\ldots          &{31}            &\textbf{25}     &{$\leq$ 20}     & 25\\
 18537 &  B7 III    & L68 & B7V        &SB&  48527 &\ldots          &{83}            &\textbf{78}     &{77}            & 78\\
 18552 &            &     & B8Vne      &SB&  55906 &{264}           &\textbf{277}    &\ldots          &\ldots          &277\\
 19356 &  B8 Vs     &  GG & B8V        & O&   2643 &\ldots          &\ldots          &\ldots          &\ldots          & 45\\
       &            &     &            &  &  37021 &\ldots          &{48}            &\textbf{45}     &{40}            &\\
 19374 &  B1.5 V    & L68 & B1.5V      &SB&  20594 &\textbf{32}     &\ldots          &{57}            &{95}            & 32\\
 19400 &            &     & B3V+A0IV   &  &   4415 &{$\leq$ 20}     &{29}            &\textbf{$\leq$ 20}&{$\leq$ 20}     &$\leq$ 20\\
 19832 &            &     & B9pSi      &  &  39685 &\ldots          &{220}           &\textbf{167}    &{152}           &167\\
 20315 &  B8 IV     & L68 & B8V        &SB&  18024 &\ldots          &{233}           &\textbf{211}    &{207}           &211\\
 20319 &            &     & B9V        &  &   7840 &\ldots          &{134}           &\textbf{121}    &{114}           &121\\
 20336 &  B2.5 Vne  & L68 & B2.5Ven    &SB&  29401 &{225}           &\textbf{249}    &{257}           &{260}           &249\\
 20809 &  B4 V      & L68 & B5V        & ?&  49124 &{127}           &\textbf{155}    &{159}           &{181}           &155\\
 21278 &  B4 V      & L68 & B5V        &1O&   2838 &\ldots          &\ldots          &\ldots          &\ldots          &\ldots\\
 21291 &  B9 Ia     &  GG & B9Ia       &  &  10612 &{49}            &{39}            &\textbf{27}     &{$\leq$ 20}     & 27\\
 21362 &  B7 Vn     & L68 & B6Vn       &SB&  18027 &\ldots          &\textbf{268}    &{219}           &{251}           &268\\
 21428 &  B5 V      & L68 & B3V        &  &  49532 &{119}           &\textbf{123}    &{128}           &{111}           &123\\
 21551 &  B8 III-IVn &  GG & B8V        &SB&  26895 &\ldots          &{286}           &\textbf{286}    &{277}           &286\\
 21699 &  kB8hB\ldots * &  GG & B8IIIpMn   &  &   4377 &{33}            &\textbf{41}     &{39}            &{30}            & 38\\
       &            &     &            &  &  32136 &{50}            &\textbf{36}     &{27}            &{$\leq$ 20}     &\\
 21790 &  B9 III    &  GG & B9V s      &  &  24553 &\ldots          &{80}            &{71}            &\textbf{64}     & 64\\
 22192 &  B5 Ve     & L68 & B5Ve       &  &   4558 &\ldots          &{202}           &\textbf{202}    &{167}           &205\\
       &            &     &            &  &  42673 &\ldots          &\textbf{208}    &{184}           &{167}           &\\
 22203 &  B8 IV     &  GG & B8V+B8V    &2O&   1946 &\ldots          &\ldots          &\ldots          &\ldots          &\ldots\\
 22316 &            &     & B9p        &  &  49775 &\ldots          &{21}            &\textbf{$\leq$ 20}&{$\leq$ 20}     &$\leq$ 20\\
       &            &     &            &  &  49777 &\ldots          &{23}            &\textbf{$\leq$ 20}&{$\leq$ 20}     &\\
 22470 &            &     & B9.5p      &  &   4542 &\ldots          &{60}            &\textbf{52}     &{42}            & 52\\
       &            &     &            &  &  21943 &\ldots          &\ldots          &\ldots          &\ldots          &\\
 22780 &  B7 Vn     & L68 & B7Vne      &  &  29392 &\ldots          &{251}           &\textbf{261}    &{248}           &261\\
 22920 &            &     & B9IIIpSi4200 &  &  14506 &{45}            &{38}            &\textbf{33}     &{$\leq$ 20}     & 33\\
 22928 &  B5 III    & L68 & B5IIIe     &SB&  47041 &{164}           &{179}           &\textbf{168}    &{178}           &168\\
 22951 &  B1 IV     & L68 & B0.5V      &SB&   6248 &\textbf{44}     &{40}            &{99}            &\ldots          & 48\\
       &            &     &            &  &   8023 &\textbf{52}     &{54}            &{139}           &\ldots          &\\
 23016 &            &     & B9Vne      &SB&  40509 &\ldots          &{248}           &\textbf{239}    &{228}           &239\\
 23180 &  B1 III    & L68 & B1III      &2O&  50031 &{86}            &\textbf{73}     &{236}           &\ldots          & 73\\
 23227 &            &     & B5IV       &SB&  40256 &{138}           &\textbf{164}    &{172}           &{200}           &164\\
 23288 &  B7 IV     & L68 & B7IV       &  &   5947 &\ldots          &\ldots          &\ldots          &\ldots          &\ldots\\
 23302 &  B6 III    & L68 & B6III      &1O&   7920 &{126}           &{142}           &\textbf{132}    &{133}           &140\\
       &            &     &            &  &  27514 &{123}           &{141}           &\textbf{147}    &{146}           &\\
 23324 &  B8 Vn     &  GG & B8V        & 2&   7923 &\ldots          &{221}           &\textbf{178}    &{161}           &178\\
 23338 &  B6 IV     & L68 & B6IV       & 1&  40508 &{100}           &{113}           &\textbf{93}     &{85}            & 93\\
 23383 &            &     & B9Vnn      &  &  19342 &\ldots          &{269}           &\textbf{296}    &{233}           &296\\
 23408 &  B8 III    & L68 & B8III      &SB&   5985 &{31}            &{29}            &\textbf{25}     &{$\leq$ 20}     & 25\\
 23432 &  B8 Vn     &  GG & B8V        &SB&   5974 &\ldots          &\ldots          &\ldots          &\ldots          &\ldots\\
 23466 &  B3 V      & L68 & B3V        &2O&  20582 &{95}            &\textbf{91}     &{105}           &{97}            & 91\\
 23480 &  B6 IV     & L68 & B6IVe      &  &   5944 &\ldots          &{196}           &\textbf{189}    &{207}           &210\\
       &            &     &            &  &  27666 &\ldots          &{215}           &\textbf{231}    &{204}           &\\
 23552 &            &     & B8Vne      &  &  10046 &\ldots          &{204}           &\textbf{183}    &{177}           &183\\
 23630 &  B7 III    & L68 & B7IIIe     &  &   8021 &\ldots          &{173}           &\textbf{130}    &{121}           &130\\
 23793 &  B3 V      & L68 & B3V+F5V    &  &  20583 &{52}            &\textbf{41}     &{48}            &{30}            & 41\\
 23850 &  B8 III    &  GG & B8III      &1O&   5977 &\ldots          &{184}           &\textbf{155}    &{154}           &155\\
 23862 &            &     & B8Vpe      &SB&  53233 &\ldots          &{281}           &\textbf{254}    &{224}           &246\\
       &            &     &            &  &  54080 &\ldots          &{266}           &\textbf{239}    &{226}           &\\
 24072 &            &     & B9V        &  &  40455 &\ldots          &{278}           &{208}           &\textbf{167}    &111\\
       &            &     &            &  &  48345 &\ldots          &{48}            &{36}            &\textbf{55}     &\\
 24131 &  B0.5 V    & L68 & B1V        &  &  20845 &\textbf{82}     &{73}            &{117}           &\ldots          & 82\\
 24388 &  B7.5 IV   &  GG & B8V        &  &  40456 &{103}           &{110}           &\textbf{107}    &{111}           &107\\
 24398 &  B1 Ib     & W76 & B1Ib       &SB&   6454 &{64}            &\textbf{55}     &{72}            &\ldots          & 55\\
 24479 &            &     & B9.5V      &  &  20434 &\ldots          &{65}            &{64}            &\textbf{61}     & 61\\
 24504 &  B6 V      & L68 & B6V        &SB&  18025 &{157}           &{190}           &\textbf{178}    &{184}           &178\\
 24534 &  O9.5 pe   & L68 &            &  &   2082 &\textbf{225}    &{194}           &\ldots          &\ldots          &231\\
       &            &     &            &  &  10789 &\textbf{237}    &{289}           &\ldots          &\ldots          &\\
 24626 &            &     & B6V        &  &   4788 &{39}            &{42}            &\textbf{41}     &{28}            & 41\\
 24640 &  B1.5 V    & L68 & B1.5V      &SB&  53315 &{102}           &\textbf{102}    &{132}           &\ldots          &102\\
 24760 &  B0.7 III  & W71 & B0.5V+A2V  & 2&   6711 &\textbf{149}    &{118}           &{158}           &\ldots          &140\\
       &            &     &            &  &  56579 &\textbf{131}    &{113}           &{146}           &\ldots          &\\
 25204 &  B3 IV     & L68 & B3V+A4IV   &2O&   3297 &{78}            &\textbf{81}     &{98}            &{83}            & 81\\
 25267 &            &     & B6V+B9.5V  & O&   7540 &\ldots          &{40}            &\textbf{26}     &{$\leq$ 20}     & 26\\
 25340 &  B5 V      & L68 & B5V        &  &  18020 &{86}            &{76}            &\textbf{75}     &{66}            & 75\\
 25558 &  B3 V      & L68 & B3V        &  &  20595 &{70}            &\textbf{48}     &{53}            &{42}            & 48\\
 25823 &            &     & B9pSi      &1O&  30567 &\ldots          &\ldots          &\ldots          &\ldots          &\ldots\\
 25940 &  B3 Ve     & L68 & B3Ve       &  &  54081 &{100}           &\textbf{138}    &{167}           &{166}           &138\\
 26326 &  B4 V      & L68 & B4V        &  &  20586 &{48}            &\textbf{25}     &{24}            &{$\leq$ 20}     & 25\\
 26356 &  B5 V      & L68 & B5V        &  &  29400 &{212}           &\textbf{220}    &{219}           &{249}           &220\\
 26571 &            &     & B9IIIp:Si: &SB&  19628 &\ldots          &\textbf{21}     &{23}            &{47}            & 21\\
 26670 &            &     & B5Vn       &  &  20435 &\ldots          &{257}           &\textbf{257}    &{244}           &257\\
 26676 &            &     & B8Vn       &  &  34223 &\ldots          &{230}           &\textbf{207}    &{200}           &207\\
 26912 &  B3 IV     & L68 & B3IV       & ?&  22280 &{67}            &\textbf{59}     &{61}            &{53}            & 59\\
 27192 &  B1.5 IV   & L68 & B1.5IV     &SB&  49849 &\textbf{131}    &{123}           &{137}           &\ldots          &131\\
 27295 &            &     & B9IV       & O&  49761 &\ldots          &{23}            &\textbf{22}     &{$\leq$ 20}     & 22\\
 27376 &            &     & B9V        &2O&  53040 &{33}            &{30}            &\textbf{$\leq$ 20}&{$\leq$ 20}     &$\leq$ 20\\
 27638 &            &     & B9V        &  &  30557 &\ldots          &\ldots          &\ldots          &\ldots          &\ldots\\
 27742 &            &     & B8IV-V     &  &  46672 &\ldots          &{201}           &\textbf{173}    &{152}           &173\\
 27778 &            &     & B3V        &  &  21298 &{112}           &\textbf{114}    &{101}           &{104}           &114\\
 28149 &  B7 V      & L68 & B7V        &  &  45496 &{142}           &{125}           &\textbf{117}    &{115}           &117\\
 28446 &  B0 IIIn   & L68 & B0III      &SB&  22429 &\textbf{289}    &{241}           &\ldots          &\ldots          &289\\
 28459 &            &     & B9.5Vn     &  &  18022 &\ldots          &{269}           &{342}           &\textbf{277}    &277\\
 28497 &  B1.5 Ve   & L68 & B1Vne      &  &   8594 &{298}           &\textbf{289}    &{315}           &\ldots          &289\\
 28843 &            &     & B9III      &  &   7522 &{110}           &{113}           &\textbf{86}     &{71}            & 94\\
       &            &     &            &  &  21098 &\ldots          &{114}           &\textbf{101}    &{86}            &\\
 28867 &            &     & B9IVn      &  &  49847 &\ldots          &{278}           &\textbf{229}    &{270}           &229\\
 28873 &            &     & B2IV-V     &  &  53193 &{25}            &\ldots          &\textbf{35}     &{30}            & 35\\
 28929 &            &     & B9pHg      &  &  30558 &\ldots          &\ldots          &\ldots          &\ldots          &\ldots\\
 29248 &  B2 III    & L68 & B2III      &SB&   4359 &{29}            &\textbf{29}     &{56}            &\ldots          & 29\\
 29335 &  B7 V      & L68 & B7V        &  &  39767 &{88}            &{81}            &\textbf{75}     &{58}            & 75\\
 29763 &            &     & B3V        &2O&   2637 &{109}           &\textbf{90}     &{100}           &{90}            & 87\\
       &            &     &            &  &  45931 &{95}            &\textbf{84}     &{87}            &{79}            &\\
 29866 &  B8 IVn    & L68 & B8IVne     &  &  10047 &\ldots          &{259}           &\textbf{239}    &{237}           &239\\
 30836 &  B2 III    & L68 & B2III+B2IV & O&   2358 &{43}            &\textbf{39}     &{61}            &{51}            & 39\\
 31237 &  B2 III    & L68 & B3III+B0V  & O&  21998 &{77}            &\textbf{78}     &{99}            &{90}            & 78\\
 31327 &  B2.5 Ib   & L68 & B2Ib       &  &  21280 &{75}            &\textbf{51}     &{49}            &{34}            & 51\\
 31331 &  B5 V      & L68 & B5V        &SB&  40502 &{132}           &{148}           &\textbf{153}    &{137}           &153\\
 31726 &  B1 V      & L68 & B1V        &  &  21681 &\textbf{21}     &\ldots          &{63}            &\ldots          & 21\\
 32249 &            &     & B3V        &  &  48993 &{52}            &\textbf{49}     &{74}            &{50}            & 49\\
 32309 &            &     & B9.5Vn     &  &  40454 &\ldots          &{339}           &{324}           &\textbf{279}    &279\\
 32343 &  B2.5 Ve   & L68 & B2.5Ve     &  &   6932 &{60}            &\textbf{67}     &{79}            &{72}            & 70\\
       &            &     &            &  &  32558 &{76}            &\textbf{74}     &{78}            &{68}            &\\
 32612 &  B2.5 IV   & L68 & B2.5IV     &  &  20588 &{54}            &\textbf{47}     &{71}            &{63}            & 47\\
 32630 &  B3 V      & W71 & B3V        &  &  44004 &{83}            &\textbf{85}     &{97}            &{89}            & 85\\
 32990 &  B2 V      & L68 & B2V        & O&  14842 &{64}            &\textbf{54}     &{83}            &{86}            & 54\\
       &            &     &            &  &  21489 &{71}            &\textbf{55}     &{82}            &{77}            &\\
 32991 &  B2 Ve     & L68 & B2Ve       &  &  14840 &{138}           &\textbf{169}    &{191}           &{268}           &169\\
 33328 &  B2 IVn    & L68 & B2IVne     &  &  55944 &{181}           &\textbf{218}    &{235}           &{235}           &218\\
 33904 &            &     & B9IIIpHgMn &  &  40118 &{31}            &{24}            &\textbf{$\leq$ 20}&{$\leq$ 20}     &$\leq$ 20\\
 34085 &            &     & B8Ia:      &SB&  15001 &{55}            &{42}            &\textbf{31}     &{28}            & 32\\
       &            &     &            &  &  41077 &{52}            &{40}            &\textbf{33}     &{$\leq$ 20}     &\\
 34310 &            &     & B9V        &  &  40255 &\ldots          &{176}           &{127}           &\textbf{123}    &123\\
 34759 &  B5 V      & L68 & B3V        & O&  10389 &{49}            &\textbf{47}     &{59}            &{52}            & 54\\
       &            &     &            &  &  38308 &{62}            &\textbf{61}     &{73}            &{62}            &\\
 34797 &            &     & B8/9III/IV &  &  24995 &\ldots          &{131}           &\textbf{50}     &{48}            & 50\\
 34798 &            &     & B3V        &  &  21787 &{56}            &\textbf{45}     &{55}            &{53}            & 45\\
 34816 &  B0.5 IV   & L68 & B0.5IV     &  &  56888 &\textbf{36}     &{37}            &{70}            &\ldots          & 36\\
 34863 &  B7 IVnn   & L68 & B7IVnn     &  &  40254 &\ldots          &{285}           &\textbf{282}    &{272}           &282\\
 34959 &  B5 Vp     & L68 & B5Vp       &  &  35918 &\ldots          &{245}           &\textbf{211}    &{167}           &211\\
 35149 &            &     & B1V        &  &  30052 &{270}           &\textbf{286}    &{340}           &\ldots          &286\\
 35337 &  B2 IV     & L68 & B2IV       &  &  49749 &\textbf{$\leq$ 20}&\ldots          &{50}            &{78}            &$\leq$ 20\\
 35407 &  B4 IVn    & L68 & B4IVn      &SB&  22663 &\ldots          &\textbf{226}    &{283}           &{301}           &226\\
 35411 &  B0.5 Vnn  & L68 & B1V+B2e    &2O&  13401 &\textbf{47}     &{102}           &{153}           &\ldots          & 52\\
       &            &     &            &  &  52811 &\textbf{56}     &{168}           &{174}           &\ldots          &\\
 35439 &  B1 Vn     & L68 & B1Vpe      &  &   7716 &{294}           &\textbf{308}    &{331}           &\ldots          &289\\
       &            &     &            &  &  50413 &{282}           &\textbf{270}    &{342}           &\ldots          &\\
 35468 &  B2 III    & W71 & B2III      & ?&  44092 &{52}            &\textbf{45}     &{72}            &{83}            & 45\\
 35497 &  B7 III    &  HS & B7III      &  &  31878 &{80}            &{53}            &\textbf{45}     &{31}            & 45\\
 35532 &  B2 Vn     & L68 & B2Vn       &  &  35853 &{203}           &\textbf{216}    &{279}           &{359}           &216\\
 35588 &  B2.5 V    & L68 & B2.5V      & O&  35893 &{131}           &\textbf{155}    &{158}           &{181}           &155\\
 35600 &            &     & B9Ib       &  &  52723 &\ldots          &{32}            &\textbf{22}     &{$\leq$ 20}     & 22\\
 35640 &            &     & B9.5Vn     &  &  40253 &\ldots          &{231}           &{203}           &\textbf{196}    &196\\
 35715 &  B1 V      & L68 & B2IV       &2O&   3264 &\textbf{95}     &{140}           &{166}           &\ldots          & 96\\
       &            &     &            &  &  35892 &\textbf{98}     &{97}            &{265}           &\ldots          &\\
 36267 &  B5 V      & L68 & B5V        &  &  35943 &{97}            &{139}           &\textbf{137}    &{137}           &137\\
 36285 &  B2 IV-V   & L68 & B2IV-V     &  &  50056 &\textbf{$\leq$ 20}&\ldots          &{50}            &{$\leq$ 20}     &$\leq$ 20\\
 36351 &  B1.5 V    & L68 & B1IV+B1.5V &  &  49065 &{38}            &\textbf{30}     &{88}            &{106}           & 30\\
 36371 &  B5 Iab    & L68 & B5Iab      & O&   2945 &{65}            &\textbf{49}     &{43}            &{47}            & 49\\
 36408 &            &     & B7IIIe     &SB&  29394 &{77}            &{67}            &\textbf{59}     &{53}            & 59\\
 36512 &  B0 V      & W71 & B0V        &  &  40415 &\textbf{29}     &{27}            &{52}            &\ldots          & 29\\
 36576 &  B2 IV-Ve  & L68 & B2IV-Ve    &  &  47042 &{196}           &\textbf{248}    &{301}           &\ldots          &248\\
 36591 &  B1 IV     & L68 & B1IV       &  &   4695 &\textbf{$\leq$ 20}&\ldots          &{102}           &\ldots          &$\leq$ 20\\
 36653 &  B3 V      & L68 & B3V        &  &  35897 &{119}           &\textbf{134}    &{166}           &{181}           &134\\
 36695 &            &     & B1V        &1O&  15639 &\textbf{145}    &{149}           &{231}           &\ldots          &145\\
 36779 &  B2.5 V    & L68 & B2.5V      &  &  32291 &{130}           &\textbf{132}    &{191}           &\ldots          &132\\
 36819 &  B2.5 IV   & L68 & B2.5IV     &SB&  46138 &{77}            &\textbf{74}     &{94}            &{82}            & 74\\
 36822 &  B0.5 IV-V & L68 & B0III      & O&   8595 &\textbf{36}     &{40}            &{81}            &\ldots          & 36\\
       &            &     &            &  &  35898 &\textbf{37}     &{40}            &{98}            &\ldots          &\\
 36960 &  B0.5 V    & W76 & B0.5V      &  &  56185 &\textbf{41}     &{34}            &{64}            &\ldots          & 41\\
 37017 &  B1.5 V    & L68 & B1.5V      &1O&  15816 &{99}            &\textbf{128}    &{140}           &{145}           &128\\
 37018 &  B1 V      & L68 & B1V        &SB&  20782 &\textbf{64}     &{67}            &{132}           &\ldots          & 64\\
 37020 &            &     & B0.5V      & O&  50204 &\ldots          &\ldots          &\ldots          &\ldots          &\ldots\\
 37021 &            &     & B0V        & O&  53178 &\textbf{112}    &{148}           &{185}           &{264}           &112\\
 37023 &            &     & B0.5Vp     &SB&  50205 &\ldots          &\ldots          &\ldots          &\ldots          &\ldots\\
 37055 &  B3 IV     & L68 & B3IV       &  &  32216 &{88}            &\textbf{89}     &{106}           &{103}           & 89\\
 37098 &            &     & B9IV-V     &  &  46843 &{51}            &{103}           &\textbf{60}     &{52}            & 60\\
 37128 &  B0 Ia     & W76 & B0Ia       &SB&   8110 &\textbf{99}     &{94}            &\ldots          &\ldots          & 97\\
       &            &     &            &  &   8130 &\textbf{95}     &{92}            &\ldots          &\ldots          &\\
 37202 &  B4 IIIp   & L68 & B4IIIpe    &1O&   3138 &{171}           &\textbf{145}    &{92}            &{64}            &168\\
       &            &     &            &  &  54067 &{255}           &\textbf{191}    &{144}           &{72}            &\\
 37232 &  B2 IV-V   & L68 & B2IV-V     &  &  35945 &{84}            &\textbf{86}     &{137}           &\ldots          & 86\\
 37303 &  B1.5 V    & L68 & B1V        &  &  36897 &{238}           &\textbf{304}    &{301}           &\ldots          &304\\
 37356 &  B2 IV-V   & L68 & B2IV-V     &  &  22114 &\textbf{26}     &\ldots          &{69}            &{82}            & 26\\
 37367 &  B2 IV-V   & L68 & B2IV-V     &SB&  21013 &{48}            &\textbf{30}     &{44}            &{36}            & 30\\
 37438 &  B3 IV     & L68 & B3IV       & O&  53869 &{53}            &\textbf{44}     &{53}            &{33}            & 44\\
 37479 &  B2 Vp     & L68 & B2Vp       &  &   2231 &{98}            &\textbf{105}    &{131}           &\ldots          &116\\
       &            &     &            &  &  15817 &{114}           &\textbf{127}    &{151}           &\ldots          &\\
 37481 &  B1.5 IV   & L68 & B1.5IV     &  &  32215 &{75}            &\textbf{66}     &{118}           &\ldots          & 66\\
 37490 &  B2 IIIe   & L68 & B3IIIe     &  &  15982 &{136}           &\textbf{148}    &{164}           &\ldots          &144\\
       &            &     &            &  &  56758 &{125}           &\textbf{141}    &{164}           &\ldots          &\\
 37519 &            &     & B9.5III-IVp: &  &  46851 &\ldots          &{231}           &\textbf{224}    &{185}           &224\\
 37744 &  B1.5 V    & L68 & B1.5V      &  &   8059 &{36}            &\textbf{31}     &{90}            &\ldots          & 31\\
 37752 &            &     & B8p        &  &  14987 &{38}            &{36}            &\textbf{27}     &{$\leq$ 20}     & 27\\
 37756 &  B2 IV-V   & L68 & B2IV-V     & O&  32293 &{59}            &\textbf{55}     &{96}            &{93}            & 55\\
 37795 &            &     & B7IVe      &  &  55941 &{157}           &\textbf{180}    &{203}           &{216}           &180\\
 37967 &  B2.5 Ve   & L68 & B2.5Ve     &  &  21491 &{133}           &\textbf{171}    &{163}           &{171}           &171\\
 38771 &  B0.5 Ia   & W76 & B0.5Ia     &  &   6736 &\textbf{88}     &{84}            &{96}            &\ldots          & 82\\
       &            &     &            &  &   8259 &\textbf{78}     &{78}            &{83}            &\ldots          &\\
       &            &     &            &  &  30267 &\textbf{80}     &{80}            &{95}            &\ldots          &\\
 38899 &            &     & B9IV       &  &  16641 &\ldots          &{28}            &\textbf{25}     &{$\leq$ 20}     & 25\\
 39698 &  B2 V      & L68 & B2V        & O&   3437 &{106}           &\textbf{109}    &{144}           &\ldots          &109\\
 39777 &  B1.5 V    & L68 & B1.5V      &  &  16881 &{25}            &\textbf{24}     &{51}            &{60}            & 24\\
 39844 &            &     & B6V        &  &   4298 &{$\leq$ 20}     &{24}            &\textbf{$\leq$ 20}&{$\leq$ 20}     &$\leq$ 20\\
 40111 &  B1 Ib     & L68 & B0.5II     &SB&   4235 &\textbf{129}    &{107}           &{110}           &\ldots          &129\\
 40494 &            &     & B2.5IV     &  &  53044 &{59}            &\textbf{55}     &{63}            &{52}            & 55\\
 40626 &            &     & B9.5IV     &  &  50181 &\ldots          &\ldots          &\ldots          &\ldots          &\ldots\\
 40967 &            &     & B5III      &SB&  50278 &{32}            &\textbf{33}     &{65}            &{82}            & 33\\
 41040 &            &     & B8III      & 2&   4709 &\ldots          &{69}            &\textbf{61}     &{58}            & 62\\
       &            &     &            &  &   4710 &\ldots          &{67}            &\textbf{62}     &{56}            &\\
 41117 &  B2 Ia     & W76 & B2Ia       &  &   6471 &{80}            &\textbf{60}     &{42}            &\ldots          & 62\\
       &            &     &            &  &  40669 &{75}            &\textbf{64}     &{58}            &\ldots          &\\
 41335 &  B2 Vne+   & L68 & B2Ven      &SB&  16526 &{227}           &\textbf{263}    &{327}           &\ldots          &248\\
       &            &     &            &  &  55943 &{240}           &\textbf{233}    &{227}           &\ldots          &\\
 41534 &            &     & B2.5V      &  &  10021 &{74}            &\textbf{64}     &{97}            &{94}            & 64\\
 41753 &  B3 IV     & L68 & B3V        &1O&   9934 &{44}            &\textbf{44}     &{57}            &{49}            & 44\\
 42054 &  B5 V      &  HS & B4IVe      &  &  29370 &\ldots          &{151}           &\textbf{172}    &{194}           &172\\
 42087 &  B2.5 Ib   & W71 & B2.5Ib     &  &   8646 &{76}            &\textbf{68}     &{62}            &\ldots          & 68\\
 42545 &  B5 Vn     & L68 & B5Vn       &SB&   9896 &\ldots          &\textbf{215}    &{218}           &{304}           &215\\
 42560 &  B3 IV     & L68 & B3IV       &1O&   9932 &{130}           &\textbf{121}    &{150}           &{321}           &121\\
 42933 &  B0.5 IV   &  HS & B3III+O9V  &2O&   7663 &\textbf{178}    &{175}           &{191}           &\ldots          &178\\
 43112 &  B1 V      & L68 & B1V        &  &   9935 &\textbf{$\leq$ 20}&\ldots          &{76}            &\ldots          &$\leq$ 20\\
 43317 &  B3 IV     & L68 & B3IV       &SB&   9949 &{85}            &\textbf{88}     &{122}           &{117}           & 88\\
 43384 &  B3 Iab    & L68 & B3Ib       &  &   4656 &{56}            &\textbf{42}     &{52}            &{55}            & 42\\
 43445 &            &     & B9Vn       &  &  40504 &\ldots          &{266}           &\textbf{226}    &{215}           &226\\
 43544 &  B2.5 Vne  & L68 & B2.5Vn     &  &  21911 &{180}           &\textbf{257}    &{300}           &{309}           &257\\
 44112 &  B2.5 V    & L68 & B2.5V      &SB&  49490 &{85}            &\textbf{94}     &{151}           &{217}           & 94\\
 44173 &  B5 III    & L68 & B5III      &  &   6803 &\ldots          &{142}           &\textbf{181}    &{261}           &181\\
 44402 &  B2.5 IV   &  HS & B2.5V      & O&  22111 &{56}            &\textbf{60}     &{92}            &{61}            & 60\\
 44458 &  B1 Ve     & L68 & B1Vpe      &SB&  54066 &{224}           &\textbf{230}    &{230}           &\ldots          &230\\
 44506 &            &     & B1.5Ve     &  &  36893 &{222}           &\textbf{217}    &{253}           &\ldots          &217\\
 44700 &  B3 V      & L68 & B3V        &SB&   9948 &{29}            &\textbf{25}     &{27}            &{$\leq$ 20}     & 25\\
 44743 &  B1 II-III & L68 & B1II-III   &SB&   4537 &{31}            &\textbf{28}     &{59}            &\ldots          & 28\\
 44953 &            &     & B8IIIHe wk &  &  14979 &\ldots          &\textbf{27}     &{27}            &{$\leq$ 20}     & 27\\
 45542 &  B6 III    & L68 & B6IIIe     &SB&  27663 &\ldots          &{179}           &\textbf{165}    &{149}           &165\\
 45546 &  B2 V      & L68 & B2V        &  &  22112 &{59}            &\textbf{56}     &{94}            &{113}           & 56\\
 45725 &  B3 Ve     & L68 & B3Ve       & O&  30191 &\ldots          &\textbf{283}    &{288}           &\ldots          &283\\
 45727 &            &     & B3e        &  &  30702 &\ldots          &{244}           &\textbf{267}    &\ldots          &267\\
 45789 &  B2.5 IV-V & W71 &            &  &   9884 &{100}           &\textbf{93}     &{111}           &{135}           & 93\\
 45813 &  B4 V      &  HS & B4V        &  &  20783 &{103}           &\textbf{108}    &{123}           &{107}           &108\\
 45911 &  B2 IV-V   & W71 &            &  &   9885 &{22}            &\textbf{23}     &{45}            &{$\leq$ 20}     & 23\\
 45995 &  B2.5 Ve   & L68 & B2V:nne    &  &   9936 &{200}           &\textbf{200}    &{229}           &\ldots          &200\\
 46328 &            &     & B0.5IV     &SB&   5114 &\textbf{$\leq$ 20}&\ldots          &{42}            &\ldots          &$\leq$ 20\\
 46485 &  O7 V:n(e) & W71 &            &  &   9913 &\textbf{316}    &\ldots          &\ldots          &\ldots          &316\\
 46487 &  B5 Vn     & L68 & B5Vn       &  &  18037 &\ldots          &{223}           &\textbf{249}    &{256}           &249\\
 46769 &  B8 Ib     & L68 & B8Ib       &  &   6341 &{56}            &{53}            &\textbf{58}     &{46}            & 58\\
 46885 &            &     & B9III      &  &  30298 &{36}            &{41}            &{34}            &\textbf{30}     & 30\\
 47054 &  B8 IV     &  HS & B8V        &  &  21909 &\ldots          &{202}           &\textbf{198}    &{203}           &198\\
 47152 &            &     & B9npEu     &  &  14957 &\ldots          &{178}           &{165}           &\textbf{206}    &206\\
 47240 &  B1 II     & L68 & B1Ib       &  &   4704 &\textbf{127}    &{113}           &{108}           &\ldots          &118\\
       &            &     &            &  &  48909 &\textbf{108}    &{107}           &{87}            &\ldots          &\\
 47382 &  B0 III-IV & W71 &            &  &   9882 &\textbf{65}     &{64}            &{89}            &\ldots          & 65\\
 47431 &            &     & B8IIIn     &  &  11011 &\ldots          &{115}           &\textbf{98}     &{96}            & 98\\
 47670 &  B8 III    &  HS & B8III      &SB&  33338 &\ldots          &{191}           &\textbf{172}    &{136}           &172\\
 47964 &            &     & B8III      &  &   9887 &\ldots          &{48}            &\textbf{44}     &{31}            & 44\\
 48279 &  O8 V      & W76 &            &  &  11299 &\textbf{150}    &\ldots          &\ldots          &\ldots          &150\\
 48434 &  B0 III    & W76 & B0III      &  &   6447 &\textbf{74}     &{71}            &{86}            &\ldots          & 74\\
 48879 &  B4 IV     & L68 & B4IV       &SB&  50256 &{84}            &\textbf{101}    &{101}           &{94}            &101\\
 48917 &  B2 III    &  HS & B2IIIe     &  &  53908 &{194}           &\textbf{183}    &{192}           &\ldots          &183\\
 48977 &  B2.5 V    & L68 & B2.5V      &  &   9893 &{24}            &\textbf{25}     &{44}            &{30}            & 25\\
 49319 &            &     & B6Vnne     &  &  52923 &\ldots          &{297}           &\textbf{225}    &{213}           &225\\
 49333 &            &     & B7IIIn     &  &   4613 &{53}            &{60}            &\textbf{66}     &{47}            & 66\\
 49567 &  B3 II-III & L68 & B3II-III   &  &   9888 &{56}            &\textbf{62}     &{78}            &{70}            & 62\\
 49591 &            &     & B9IV       &  &  40505 &\ldots          &{134}           &{107}           &\textbf{117}    &117\\
 49606 &            &     & B7III      &  &  14966 &{$\leq$ 20}     &{24}            &\textbf{23}     &{$\leq$ 20}     & 23\\
 50013 &  B2 IV     &  HS & B1.5IVne   &  &  15981 &{183}           &\textbf{233}    &{269}           &\ldots          &233\\
 50123 &            &     & B6Vnpe     &  &  56825 &\ldots          &{156}           &\textbf{162}    &{134}           &162\\
 50707 &            &     & B1IV       &  &  27758 &\textbf{45}     &{40}            &{73}            &\ldots          & 45\\
 50820 &  B3 IVE+F  & L68 & B3IVe+K2II &SB&  21027 &{64}            &\textbf{54}     &{73}            &{55}            & 54\\
 51283 &            &     & B3II-III   &  &  19710 &{152}           &\textbf{162}    &{173}           &\ldots          &162\\
 51309 &  B3 II     & L68 & B3II       &  &  13937 &{54}            &\textbf{42}     &{40}            &{30}            & 42\\
 52089 &  B2 II     &  HS & B2II       &  &  54337 &{47}            &\textbf{40}     &{62}            &\ldots          & 40\\
 52382 &  B1 Ib     & L68 & B1Ib       &  &  33293 &\textbf{85}     &{78}            &{69}            &\ldots          & 85\\
 52559 &  B2 IV-V   & L68 & B2IV-V     &  &  16535 &\ldots          &\ldots          &\ldots          &\ldots          &\ldots\\
 52918 &  B1 IV     & L68 & B1V        &  &  52939 &{261}           &\textbf{257}    &{296}           &\ldots          &257\\
 53138 &  B3 Ia     & W76 & B3Iab      &SB&  30169 &{63}            &\textbf{50}     &{41}            &{30}            & 50\\
 53755 &  B0.5 IVn  & L68 & B0.5V+F5III &SB&   8851 &\textbf{294}    &{207}           &{227}           &\ldots          &288\\
       &            &     &            &  &  27974 &\textbf{281}    &{249}           &{271}           &\ldots          &\\
 53929 &            &     & B9.5III    &  &  18961 &{25}            &{26}            &\textbf{$\leq$ 20}&{$\leq$ 20}     &$\leq$ 20\\
 53974 &  B0.5 III  & L68 & B0.5IV     &  &  10580 &\textbf{129}    &{108}           &{107}           &\ldots          &129\\
 54031 &            &     & B3V        &SB&  41438 &{29}            &\ldots          &\textbf{27}     &{$\leq$ 20}     & 27\\
 54309 &  B2 IV     &  HS & B2IVe      &  &  29373 &{160}           &\textbf{155}    &{174}           &\ldots          &155\\
 54893 &            &     & B2IV-V     &  &  53220 &{32}            &\textbf{27}     &{41}            &{$\leq$ 20}     & 27\\
 55857 &            &     & B0.5V      &  &  46933 &\textbf{157}    &{150}           &{191}           &\ldots          &157\\
 55879 &  B0 III    & L68 & B0III      &  &   8852 &\textbf{51}     &{70}            &{103}           &\ldots          & 51\\
 56014 &            &     & B3IIIe     & 2&  38150 &\textbf{223}    &{129}           &{157}           &\ldots          &268\\
       &            &     &            &  &  41294 &\textbf{312}    &{213}           &{221}           &\ldots          &\\
 56139 &            &     & B2IV-Ve    & ?&  15980 &{83}            &\textbf{88}     &{99}            &{105}           & 88\\
 56456 &            &     & B8-9V      &  &  34188 &\ldots          &{251}           &{284}           &\textbf{250}    &250\\
 58050 &  B2 Ve     & L68 & B2Ve       &  &  16536 &{97}            &\textbf{119}    &{141}           &\ldots          &119\\
 58343 &  B2.5 IVe  & L68 & B2.5IVe    &  &  16264 &{51}            &\textbf{43}     &{55}            &{35}            & 43\\
 58350 &            &     & B5Ia       &  &   7704 &{62}            &\textbf{47}     &{44}            &{37}            & 46\\
       &            &     &            &  &  30198 &{62}            &\textbf{46}     &{41}            &{30}            &\\
 58661 &            &     & B9pHgMn    &  &  18985 &{$\leq$ 20}     &{34}            &\textbf{26}     &{$\leq$ 20}     & 26\\
 58715 &  B8 V      &  HS & B8Ve       &SB&  31890 &\ldots          &{193}           &\textbf{197}    &{198}           &197\\
 58978 &            &     & B0IV:pe    &  &  54238 &\textbf{304}    &{252}           &{250}           &\ldots          &304\\
 59635 &            &     & B5Vp       &  &  52272 &{43}            &\textbf{27}     &{41}            &{$\leq$ 20}     & 27\\
 60606 &  B3 V      &  HS & B3Vne      &SB&  29375 &{162}           &\textbf{184}    &{196}           &{215}           &184\\
 60855 &  B2 Ve     & L68 & B2Ve       &  &  53906 &{190}           &\textbf{219}    &{241}           &{325}           &219\\
 61429 &            &     & B8IV       &  &   4611 &\ldots          &{97}            &\textbf{92}     &{89}            & 92\\
 62226 &            &     & B5V        & 2&  39770 &{38}            &\textbf{34}     &{105}           &{33}            & 34\\
 63462 &            &     & B0V:pe:    &  &  41304 &\textbf{359}    &{317}           &{329}           &\ldots          &359\\
 63578 &            &     & B1.5IV     &SB&   5532 &\textbf{79}     &{112}           &{162}           &\ldots          & 79\\
 63922 &  B0 III    &  HS & B0III      &  &  40146 &\textbf{45}     &{44}            &{84}            &\ldots          & 45\\
 64503 &  B2.5 V    &  HS & B2.5V      &SB&  21950 &{112}           &\textbf{121}    &{155}           &{300}           &121\\
 64740 &            &     & B1.5Vp     &  &  19153 &{125}           &\textbf{133}    &{175}           &\ldots          &133\\
 64760 &            &     & B0.5Ib     &  &  53781 &\textbf{231}    &{175}           &\ldots          &\ldots          &231\\
 64802 &            &     & B2V        &  &  46848 &{56}            &\textbf{49}     &{64}            &{53}            & 49\\
 65575 &  B3 IV     &  HS & B3IVp      &  &  21952 &{57}            &\textbf{50}     &{57}            &{45}            & 50\\
 65818 &            &     & B1Vp+B3IV: &2O&  54424 &\textbf{266}    &{311}           &{310}           &\ldots          &266\\
 65904 &            &     & B4V        &  &  46849 &{127}           &\textbf{142}    &{155}           &{157}           &142\\
 66194 &            &     & B2IVpne    &  &  31232 &{178}           &\textbf{214}    &{218}           &\ldots          &214\\
 67536 &            &     & B2.5Vn     &  &  53912 &{211}           &\textbf{220}    &{259}           &\ldots          &220\\
 67880 &  B2.5 V    & L68 & B2.5V      &  &  53222 &{37}            &\ldots          &\textbf{47}     &{40}            & 47\\
 67888 &            &     & B4V        &  &  31233 &{45}            &\textbf{36}     &{41}            &{31}            & 36\\
 68217 &            &     & B2IV-V     &  &  24913 &{98}            &\textbf{113}    &{134}           &{158}           &113\\
 68243 &  B1 IV     &  HS & B1IV       & O&  32437 &\textbf{92}     &{99}            &{133}           &\ldots          & 92\\
 68351 &            &     & B9pSiCr    & O&  18969 &\ldots          &{44}            &\textbf{30}     &{$\leq$ 20}     & 30\\
 68520 &  B6 IV     &  HS & B6IV       & O&  24456 &{42}            &{23}            &\textbf{$\leq$ 20}&{$\leq$ 20}     &$\leq$ 20\\
 68980 &            &     & B1.5IIIe   & 2&  36214 &{130}           &\textbf{144}    &{185}           &\ldots          &144\\
 69081 &            &     & B1.5IV     &  &  10159 &{160}           &\textbf{153}    &{174}           &\ldots          &153\\
 69082 &            &     & B2IV-V     & 2&   7926 &{29}            &\textbf{24}     &{41}            &{51}            & 24\\
 70556 &            &     & B2IV-V     &SB&  51420 &{38}            &\textbf{33}     &{73}            &{64}            & 33\\
 70930 &            &     & B1V        &  &   5531 &{103}           &\textbf{100}    &{137}           &\ldots          &100\\
 72067 &            &     & B2Vn       &  &  29376 &{117}           &\textbf{154}    &{173}           &{180}           &154\\
 72232 &            &     & B5III      &  &  14105 &\ldots          &\ldots          &\ldots          &\ldots          &\ldots\\
 72350 &            &     & B4IV       &  &  44774 &{111}           &\textbf{141}    &{171}           &{173}           &141\\
 73340 &            &     & B8Si       &  &   9032 &\ldots          &\ldots          &\ldots          &\ldots          &\ldots\\
 74067 &            &     & B9V        &  &  24821 &\ldots          &\ldots          &\ldots          &\ldots          &\ldots\\
 74195 &            &     & B3IV       &SB&  21415 &{31}            &\textbf{23}     &{26}            &{$\leq$ 20}     & 23\\
 74196 &            &     & B7Vn       &SB&   7908 &\ldots          &{270}           &\textbf{290}    &{273}           &290\\
 74273 &            &     & B1.5V      &  &  53919 &\textbf{127}    &{135}           &{167}           &\ldots          &127\\
 74280 &  B4 V      & L68 & B3V        & ?&  54052 &{82}            &\textbf{93}     &{106}           &{89}            & 93\\
 74371 &            &     & B6Iae      &  &  37494 &{60}            &\textbf{46}     &{45}            &{42}            & 46\\
 74375 &  B1.5 III  &  HS & B1.5III    &SB&  53228 &{57}            &\textbf{47}     &{70}            &\ldots          & 47\\
 74455 &            &     & B1.5Vn     &  &   5558 &\textbf{169}    &{170}           &{213}           &\ldots          &209\\
       &            &     &            &  &  52931 &\textbf{249}    &{241}           &{280}           &\ldots          &\\
 74535 &            &     & B8Si       &  &  48065 &{42}            &{40}            &\textbf{36}     &{$\leq$ 20}     & 36\\
 74560 &            &     & B3IV       &SB&  53219 &{39}            &\textbf{27}     &{27}            &{$\leq$ 20}     & 27\\
 74575 &  B1.5 III  &  HS & B1.5III    &  &   5145 &{28}            &\textbf{28}     &{49}            &\ldots          & 28\\
 74753 &            &     & B0IIIn     &  &   5557 &\textbf{284}    &{198}           &{224}           &\ldots          &284\\
 75311 &  B3 V      &  HS & B3Vne      &  &  21953 &{182}           &\textbf{237}    &{273}           &{312}           &237\\
 75387 &            &     & B2IV-V     &  &   7911 &\textbf{28}     &\ldots          &{45}            &{28}            & 28\\
 75759 &            &     & B1-2III    &2O&  54009 &\textbf{120}    &\ldots          &\ldots          &\ldots          &120\\
 76161 &            &     & B3Vn       &  &  14122 &\ldots          &{230}           &\textbf{265}    &{275}           &265\\
 76566 &            &     & B3IV       &  &  14121 &{$\leq$ 20}     &\ldots          &\textbf{27}     &{$\leq$ 20}     & 27\\
 76805 &            &     & B5V        &1O&  32421 &{68}            &{54}            &\textbf{63}     &{56}            & 63\\
 78764 &  B2 IV     &  HS & B2IVe      &  &  26389 &{122}           &\textbf{126}    &{143}           &\ldots          &126\\
 79158 &            &     & B8IIIpMn   &  &  32947 &\ldots          &{48}            &\textbf{40}     &{34}            & 40\\
 79186 &            &     & B5Ia       &  &  37499 &{64}            &\textbf{52}     &{45}            &{38}            & 52\\
 79447 &  B3 III    &  HS & B3III      &  &  53231 &{24}            &\textbf{$\leq$ 20}&{$\leq$ 20}     &{$\leq$ 20}     &$\leq$ 20\\
 79469 &            &     & B9.5V      &SB&  21401 &\ldots          &{185}           &\textbf{79}     &{72}            & 79\\
 79694 &            &     & B5/6IV/V   &  &  46931 &{162}           &{173}           &\textbf{168}    &{167}           &168\\
 79931 &            &     & B9III      &  &   4630 &\ldots          &{57}            &\textbf{51}     &{52}            & 51\\
 81188 &            &     & B2IV-V     & O&   2081 &{35}            &\textbf{41}     &{97}            &{59}            & 41\\
       &            &     &            &  &  36789 &\ldots          &\ldots          &\ldots          &\ldots          &\\
 83058 &            &     & B1.5IV     &SB&  53221 &\textbf{64}     &{71}            &{141}           &\ldots          & 64\\
 83754 &  B5 V      & L68 & B5V        &  &  54740 &{110}           &{133}           &\textbf{136}    &{121}           &136\\
 83953 &  B5 V      &  HS & B6Ve       &  &  56848 &{180}           &\textbf{252}    &{234}           &{268}           &252\\
 85871 &            &     & B1IV       &SB&  15201 &{229}           &\textbf{226}    &{238}           &\ldots          &226\\
 86360 &            &     & B9IV       &SB&   9038 &\ldots          &{99}            &{87}            &\textbf{95}     & 95\\
 86440 &            &     & B5Ib       &  &   6323 &{31}            &\textbf{30}     &{25}            &{$\leq$ 20}     & 30\\
 86606 &  B1 Ib     & W76 & B9Ib       &  &  38784 &\textbf{59}     &{47}            &{74}            &\ldots          & 59\\
 86612 &            &     & B4Ve       &  &  54628 &{114}           &\textbf{149}    &{165}           &{158}           &149\\
 87015 &  B2.5 IV   & L68 & B2.5IV     &  &  21988 &{152}           &\textbf{179}    &{198}           &{288}           &179\\
 87504 &            &     & B9III-IV   & 2&  50698 &\ldots          &{53}            &{47}            &\textbf{46}     & 46\\
 87543 &            &     & B7IVne     &  &  33318 &\ldots          &{231}           &\textbf{221}    &{204}           &221\\
 87901 &  B7 V      &  HS & B7V        &SB&   8648 &\ldots          &{264}           &\textbf{264}    &{247}           &247\\
       &            &     &            &  &   8649 &\ldots          &{236}           &\textbf{238}    &{309}           &\\
       &            &     &            &  &  10379 &\ldots          &{216}           &\textbf{213}    &{255}           &\\
       &            &     &            &  &  54284 &\ldots          &{254}           &\textbf{273}    &{276}           &\\
 88661 &  B2 IV     &  HS & B2IVpne    &  &  53905 &{206}           &\textbf{220}    &{258}           &\ldots          &220\\
 88825 &            &     & B4Ve       &  &  31231 &{108}           &{128}           &\textbf{116}    &{107}           &116\\
 89080 &            &     & B8III      &  &  32560 &\ldots          &{219}           &\textbf{184}    &{171}           &184\\
 89688 &  B2.5 IV   & L68 & B2.5IV     &  &  10472 &{212}           &\textbf{190}    &{179}           &{245}           &168\\
       &            &     &            &  &  21987 &{145}           &\textbf{146}    &{176}           &{238}           &\\
 89890 &            &     & B3IIIe     &  &  33319 &{70}            &\textbf{60}     &{74}            &{61}            & 60\\
 90264 &            &     & B8V        & 2&   7362 &\ldots          &{65}            &\textbf{48}     &{46}            & 48\\
 90994 &  B6 V      & L68 & B6V        &  &   9220 &{92}            &{93}            &\textbf{81}     &{67}            & 81\\
 91120 &            &     & B9Vn       &SB&  30911 &\ldots          &{243}           &\textbf{223}    &{209}           &223\\
 91316 &  B1 Iab    & W76 & B1Ib       &SB&   4553 &{76}            &\textbf{67}     &{87}            &\ldots          & 72\\
       &            &     &            &  &   8651 &\textbf{77}     &{68}            &{82}            &\ldots          &\\
       &            &     &            &  &  11312 &{76}            &\textbf{69}     &{80}            &\ldots          &\\
       &            &     &            &  &  53059 &\textbf{73}     &{62}            &{81}            &\ldots          &\\
 91465 &            &     & B4Vne      &  &  54082 &{169}           &\textbf{216}    &{226}           &{254}           &216\\
 91619 &            &     & B7Iae      &  &   6324 &{59}            &\textbf{46}     &{42}            &{29}            & 46\\
 92664 &            &     & B9pSi      &  &   7905 &{79}            &{53}            &\textbf{48}     &{30}            & 48\\
 93030 &            &     & B0Vp       &SB&  54100 &\textbf{100}    &{100}           &{132}           &\ldots          &100\\
 93563 &            &     & B8-9IIIe   &  &  33317 &\ldots          &{226}           &\textbf{202}    &{171}           &202\\
 93845 &  B2.5 IV   &  HS & B2.5IV     &  &  53883 &{50}            &\textbf{41}     &{47}            &{33}            & 41\\
 96919 &            &     & B9Ia       &  &  48010 &{80}            &{43}            &\textbf{28}     &{26}            & 28\\
 98664 &            &     & B9.5V s    &SB&  24818 &\ldots          &\ldots          &\ldots          &\ldots          &\ldots\\
 98718 &            &     & B5Vn       &  &  20793 &\ldots          &\textbf{269}    &{309}           &{269}           &269\\
100600 &  B4 V      & L68 & B4V        &SB&   9209 &{122}           &\textbf{141}    &{146}           &{144}           &141\\
100673 &            &     & B9Ve       &  &  36404 &\ldots          &{200}           &{124}           &\textbf{128}    &128\\
100889 &            &     & B9.5Vn     &SB&  46757 &\ldots          &{231}           &{144}           &\textbf{160}    &160\\
101431 &            &     & B9V        &  &  54190 &\ldots          &{179}           &{103}           &\textbf{108}    &108\\
102232 &            &     & B6III      &  &   4628 &{33}            &{37}            &\textbf{28}     &{$\leq$ 20}     & 28\\
103192 &            &     & B9IIIpSi   &  &  54240 &\ldots          &{191}           &\textbf{61}     &{52}            & 61\\
104174 &            &     & B9Vn       &  &  54669 &\ldots          &{272}           &{229}           &\textbf{233}    &233\\
104337 &  B1 V      & L68 & B1.5V      & O&  36492 &{96}            &\textbf{89}     &{215}           &\ldots          & 89\\
105382 &  B4 III    &  HS & B6IIIe     &  &  36403 &{52}            &\textbf{61}     &{74}            &{55}            & 61\\
105937 &            &     & B3V        &  &  44211 &{84}            &\textbf{95}     &{103}           &{95}            & 95\\
106231 &            &     & B4IV       &  &   5658 &{126}           &\textbf{108}    &{129}           &{156}           &108\\
106343 &            &     & B1.5Ia     &  &   6320 &{85}            &\textbf{67}     &{57}            &\ldots          & 67\\
106490 &  B2 IV     &  HS & B2IV       &  &  21954 &\textbf{127}    &{137}           &{170}           &\ldots          &127\\
106625 &            &     & B8IIIpHgMn &SB&  19063 &\ldots          &{32}            &\textbf{26}     &{$\leq$ 20}     & 26\\
106911 &  B5 V      &  HS & B5Vn       &  &  21957 &\ldots          &{208}           &\textbf{234}    &{238}           &234\\
106983 &  B2.5 V    &  HS & B2.5V      &  &  46634 &{77}            &\textbf{60}     &{78}            &{55}            & 60\\
107348 &  B8 V      &  HS & B8Vne      &SB&  53833 &\ldots          &{214}           &\textbf{193}    &{200}           &193\\
108248 &  B0.5 IV   &  HS & B0.5IV     & O&  33950 &\textbf{100}    &{95}            &{131}           &\ldots          &104\\
       &            &     &            &  &  44493 &\textbf{109}    &{115}           &{124}           &\ldots          &\\
108257 &            &     & B3Vn       &  &  19345 &\ldots          &\textbf{257}    &{290}           &{210}           &257\\
108483 &  B2 V      &  HS & B2V        &  &  20792 &{141}           &\textbf{162}    &{189}           &{237}           &162\\
109026 &            &     & B5V        &  &  51268 &{143}           &{148}           &\textbf{151}    &{137}           &151\\
109387 &  B6 IIIp   & L68 & B6IIIpe    & O&  15056 &\ldots          &{175}           &\textbf{167}    &{187}           &160\\
       &            &     &            &  &  39973 &\ldots          &{170}           &\textbf{153}    &{134}           &\\
109668 &  B2 IV-V   &  HS & B2IV-V     &  &  20873 &{86}            &\textbf{100}    &{139}           &{130}           &100\\
109857 &            &     & B8Vne      &  &  36405 &\ldots          &{351}           &{251}           &\textbf{204}    &204\\
       &            &     &            &  &  39434 &\ldots          &\ldots          &\ldots          &\ldots          &\\
109867 &  B0.7 Ib   & W76 & B1Ia       &  &  44494 &\textbf{83}     &{74}            &{81}            &\ldots          & 83\\
110073 &            &     & B8II/III   & 1&  54182 &{41}            &\ldots          &\textbf{$\leq$ 20}&{$\leq$ 20}     &$\leq$ 20\\
110335 &            &     & B6IVe      &SB&  44487 &\ldots          &{213}           &\textbf{179}    &{170}           &179\\
110432 &            &     & B1IIIe     &SB&  40695 &\textbf{274}    &{235}           &{235}           &\ldots          &274\\
110879 &            &     & B2.5V      &  &  45089 &{109}           &\textbf{121}    &{156}           &{207}           &121\\
111123 &  B0.5 III  &  HS & B0.5III    &SB&  14931 &\textbf{48}     &{43}            &{102}           &\ldots          & 48\\
111226 &            &     & B8III      &  &  20277 &\ldots          &{69}            &\textbf{72}     &{75}            & 72\\
111774 &            &     & B8V        &  &  30559 &\ldots          &\ldots          &\ldots          &\ldots          &\ldots\\
111904 &            &     & B9Ia       &  &  46883 &{75}            &{39}            &\textbf{26}     &{$\leq$ 20}     & 26\\
111973 &            &     & B5Ia       &  &  46847 &{78}            &\textbf{58}     &{45}            &{49}            & 58\\
112078 &  B4 V      &  HS & B4Vne      &  &  21965 &\ldots          &\textbf{325}    &{320}           &{319}           &325\\
112091 &  B5 V      &  HS & B5Vne      &  &  40675 &\ldots          &{182}           &\textbf{174}    &{174}           &174\\
112092 &  B2 IV-V   &  HS & B2IV-V     &  &  21955 &{51}            &\textbf{33}     &{50}            &{47}            & 33\\
112413 &            &     &            &  &   4813 &\ldots          &\ldots          &\ldots          &\ldots          &\ldots\\
113120 &  B2 III    &  HS & B1.5IIIne  &  &  30910 &{345}           &\textbf{322}    &{367}           &\ldots          &322\\
113791 &            &     & B1.5V      &1O&  44645 &\textbf{21}     &\ldots          &{44}            &{30}            & 21\\
113797 &            &     & B9V        &  &  22295 &\ldots          &{201}           &{125}           &\textbf{127}    &127\\
113904 &            &     & B0Ia+WC5:  & O&  39551 &\textbf{133}    &\ldots          &\ldots          &\ldots          &133\\
116072 &            &     & B2.5Vn     &  &  20362 &{150}           &\textbf{211}    &{212}           &{212}           &211\\
116084 &            &     & B2.5Ib     &  &  44795 &{83}            &\textbf{62}     &{46}            &{34}            & 62\\
116087 &            &     & B3V        &  &  20791 &{165}           &\textbf{182}    &{198}           &{196}           &182\\
118991 &            &     & B8Vn       &  &  30072 &\ldots          &{334}           &{283}           &\textbf{271}    &271\\
119159 &            &     & B0.5III    &  &  10807 &\textbf{131}    &{112}           &{125}           &\ldots          &131\\
119361 &            &     & B8III      &  &  30561 &\ldots          &\ldots          &\ldots          &\ldots          &\ldots\\
120307 &  B2 IV     &  HS & B2IV       &1O&  36482 &{62}            &\textbf{54}     &{91}            &\ldots          & 54\\
120315 &  B3 V      & W71 & B3V        & ?&  10894 &{150}           &\textbf{121}    &{129}           &{130}           &118\\
       &            &     &            &  &  55240 &{111}           &\textbf{114}    &{125}           &{111}           &\\
120324 &  B2 IV-V   &  HS & B2IV-Ve    &SB&  51095 &{111}           &\textbf{139}    &{144}           &{160}           &139\\
120640 &            &     & B2Vp       &  &  14685 &{30}            &\ldots          &\textbf{40}     &{30}            & 40\\
120709 &            &     & B5IIIp     &  &  30546 &{29}            &\ldots          &\textbf{$\leq$ 20}&{$\leq$ 20}     &$\leq$ 20\\
120955 &            &     & B4IV       & O&  30545 &\ldots          &\ldots          &\ldots          &\ldots          &\ldots\\
120991 &  B2 III    &  HS & B2IIIe     &  &  27498 &{66}            &\textbf{55}     &{89}            &\ldots          & 55\\
121263 &            &     & B2.5IV     &2O&  54855 &\textbf{155}    &{184}           &{205}           &\ldots          &155\\
121743 &  B2 IV     &  HS & B2IV       &  &  21956 &{70}            &\textbf{60}     &{91}            &{80}            & 60\\
121790 &  B2 IV-V   &  HS & B2IV-V     &  &  45209 &{108}           &\textbf{112}    &{131}           &{118}           &112\\
121847 &            &     & B8VpShell  &SB&  30530 &\ldots          &\ldots          &\ldots          &\ldots          &\ldots\\
122879 &  B0 Ia     & W76 & B0Ia       &  &  40897 &\textbf{100}    &{90}            &\ldots          &\ldots          &100\\
122980 &  B2 V      &  HS & B2V        &  &  42106 &\textbf{$\leq$ 20}&\ldots          &{34}            &{27}            &$\leq$ 20\\
125238 &  B2.5 IV   &  HS & B2.5IV     &  &  20891 &{154}           &\textbf{192}    &{274}           &{360}           &192\\
125288 &            &     & B6Ib       &  &   6321 &{40}            &{36}            &\textbf{26}     &{$\leq$ 20}     & 26\\
125745 &            &     & B8V        &  &  30543 &\ldots          &\ldots          &\ldots          &\ldots          &\ldots\\
125823 &            &     & B7IIIp     &  &  14088 &{22}            &\ldots          &\textbf{37}     &{28}            & 37\\
126341 &            &     & B2IV       &  &   5208 &\textbf{$\leq$ 20}&\ldots          &{43}            &{51}            &$\leq$ 20\\
127381 &  B2 III    &  HS & B2III      &  &  48225 &{57}            &\textbf{54}     &{102}           &{94}            & 54\\
127971 &            &     & B7V        &  &  50485 &{168}           &{133}           &\textbf{126}    &{129}           &126\\
127972 &  B1.5 V    &  HS & B1.5Vne    &SB&  41260 &{230}           &\textbf{260}    &{264}           &\ldots          &260\\
128345 &  B5 V      &  HS & B5V        &  &  20786 &{118}           &{155}           &\textbf{168}    &{161}           &168\\
129116 &  B3 V      &  HS & B3V        &  &  45090 &{112}           &\textbf{130}    &{146}           &{145}           &130\\
132058 &  B2 III    &  HS & B2III/IV   &SB&  17458 &{85}            &\textbf{84}     &{123}           &\ldots          & 84\\
132200 &  B2 IV     &  HS & B2IV       &SB&  17459 &{32}            &\textbf{27}     &{45}            &{29}            & 27\\
132742 &            &     & B9.5V      & O&  45452 &\ldots          &{74}            &{61}            &\textbf{68}     & 68\\
133029 &            &     & B9pSiSrCr  &  &   6102 &\ldots          &{28}            &\textbf{$\leq$ 20}&{$\leq$ 20}     &$\leq$ 20\\
133242 &            &     & B5V        &SB&  20364 &\ldots          &{152}           &\textbf{159}    &{163}           &159\\
134481 &  B9 V      &  HS & B9.5Vne    &  &  36776 &\ldots          &{206}           &{137}           &\textbf{136}    &136\\
134687 &            &     & B3IV       &1O&  20219 &{29}            &\ldots          &\textbf{27}     &{$\leq$ 20}     & 27\\
135160 &            &     & B0.5Ve     &  &  16726 &\textbf{164}    &{194}           &{176}           &\ldots          &164\\
135348 &            &     & B3IV       &  &   7685 &{90}            &\textbf{86}     &{84}            &{79}            & 86\\
136298 &  B1.5 IV   &  HS & B1.5IV     &  &  16778 &{185}           &\textbf{162}    &{183}           &\ldots          &162\\
136504 &            &     & B2IV-V     &2O&  33285 &{85}            &\textbf{87}     &{109}           &{117}           & 87\\
136664 &  B4 V      &  HS & B4V        &  &  20789 &{107}           &\textbf{157}    &{168}           &{180}           &157\\
137387 &            &     & B3IVe      &  &  41240 &{199}           &\textbf{221}    &{225}           &\ldots          &221\\
137432 &            &     & B4Vp       &  &   7689 &{75}            &{60}            &\textbf{63}     &{57}            & 63\\
138485 &  B2 Vn     & L68 & B2Vn       &SB&  20597 &{166}           &\textbf{201}    &{254}           &{286}           &201\\
138690 &            &     & B2IV       &  &  20890 &{189}           &\textbf{254}    &{288}           &\ldots          &254\\
138749 &  B6 Vnn    & L68 & B6Vnne     &  &  13588 &\ldots          &\textbf{265}    &{293}           &{272}           &284\\
       &            &     &            &  &  17096 &\ldots          &\textbf{305}    &{306}           &{272}           &\\
       &            &     &            &  &  50410 &\ldots          &\textbf{283}    &{298}           &{316}           &\\
138769 &  B3 IV     &  HS & B3IVp      &SB&  50593 &{72}            &\textbf{64}     &{70}            &{51}            & 64\\
139365 &            &     & B2.5V      &SB&  40902 &{145}           &\textbf{180}    &{182}           &{211}           &180\\
139892 &            &     & B7V        &2O&  20197 &\ldots          &\ldots          &\ldots          &\ldots          &\ldots\\
140008 &            &     & B5V        &2O&  22093 &{115}           &{106}           &\textbf{112}    &{111}           &112\\
140436 &            &     & B9IV+A3V   &  &  50595 &\ldots          &{221}           &{107}           &\textbf{110}    &110\\
141318 &            &     & B2II       &SB&  39022 &{51}            &\textbf{41}     &{54}            &{47}            & 41\\
141556 &            &     & B9IV       &2O&  48509 &\ldots          &{27}            &\textbf{$\leq$ 20}&{$\leq$ 20}     &$\leq$ 20\\
141637 &            &     & B3V        &  &  22092 &{196}           &\textbf{238}    &{243}           &\ldots          &238\\
142184 &  B3 V      &  HS & B2.5Vne    &  &  34217 &\ldots          &\textbf{275}    &{301}           &{254}           &275\\
142301 &            &     & B8IIIp     &  &   7899 &{106}           &{75}            &\textbf{77}     &{57}            & 76\\
       &            &     &            &  &  51414 &{78}            &{78}            &\textbf{75}     &{52}            &\\
142514 &            &     & B7III      &  &  23434 &{129}           &{76}            &\textbf{68}     &{58}            & 68\\
142669 &            &     & B2IV-V     &  &  55200 &{79}            &\textbf{88}     &{108}           &{84}            & 88\\
142926 &            &     & B9pe       & O&  44981 &\ldots          &{264}           &\textbf{249}    &{221}           &249\\
142983 &  B5 IIIp   & L68 & B5IIIp     & ?&   8810 &\ldots          &{65}            &{42}            &\textbf{23}     & 21\\
       &            &     &            &  &  54091 &\ldots          &{45}            &{27}            &\textbf{$\leq$ 20}&\\
142990 &            &     & B5IV       &  &  51412 &{97}            &\textbf{100}    &{105}           &{94}            &100\\
143018 &            &     & B1V+B2V    &2O&  54422 &\textbf{112}    &{237}           &{252}           &\ldots          &112\\
143118 &            &     & B2.5IV     &  &  42117 &{172}           &\textbf{180}    &{210}           &\ldots          &180\\
143275 &  B0.3 IV   &  HS & B0.3IV     &SB&  17395 &\textbf{137}    &{160}           &{149}           &\ldots          &137\\
143699 &            &     & B6IV       &  &  50532 &\ldots          &\ldots          &\ldots          &\ldots          & 81\\
       &            &     &            &  &  51385 &{94}            &{87}            &\textbf{81}     &{72}            &\\
144206 &            &     & B9III      &  &  40560 &\ldots          &\ldots          &\textbf{$\leq$ 20}&{$\leq$ 20}     &$\leq$ 20\\
144217 &  B0.5 V    & L68 & B1V        & O&   6233 &\textbf{96}     &{158}           &{246}           &\ldots          & 96\\
144294 &  B2.5 V    &  HS & B2.5Vn     &  &  20787 &{215}           &\textbf{265}    &{315}           &\ldots          &259\\
       &            &     &            &  &  22090 &{192}           &\textbf{253}    &{265}           &{304}           &\\
144334 &            &     & B8p        &  &   9224 &{64}            &{59}            &\textbf{50}     &{41}            & 51\\
       &            &     &            &  &  51411 &{53}            &{55}            &\textbf{52}     &{31}            &\\
144470 &  B1 V      & W71 & B1V        &  &  13906 &\textbf{102}    &{101}           &{144}           &\ldots          &102\\
144661 &            &     & B7IIIp:    &  &  13953 &{52}            &{42}            &\textbf{29}     &{22}            & 29\\
144844 &            &     & B9IVp:     & 2&  51322 &{73}            &{25}            &\textbf{$\leq$ 20}&{$\leq$ 20}     &$\leq$ 20\\
145389 &            &     & B9p:Mn:    &1O&   6886 &\ldots          &\ldots          &\ldots          &\ldots          &\ldots\\
145482 &            &     & B2V        &  &  19351 &{166}           &\textbf{165}    &{166}           &{224}           &165\\
145483 &            &     & B9V        &  &  16306 &\ldots          &{214}           &{174}           &\textbf{172}    &172\\
145502 &  B2 IVp    & L68 & B3V        & O&  29140 &{114}           &\textbf{119}    &{140}           &{177}           &119\\
145842 &            &     & B8V        &  &  50530 &\ldots          &{112}           &\textbf{97}     &{95}            & 97\\
146001 &            &     & B7IV       &  &  51399 &\ldots          &{106}           &\textbf{79}     &{66}            & 79\\
147152 &            &     & B6IV       &  &  19346 &{107}           &{96}            &\textbf{100}    &{89}            &100\\
147165 &  B1 III    & W71 & B1III      & O&  39508 &\textbf{61}     &{56}            &{94}            &\ldots          & 61\\
147628 &            &     & B8IV       &  &  50531 &\ldots          &{128}           &\textbf{115}    &{117}           &115\\
147971 &            &     & B4V        &2O&  50529 &\ldots          &\ldots          &\ldots          &\ldots          &\ldots\\
148112 &            &     & B9pCr      &  &   2726 &\ldots          &{61}            &\textbf{34}     &{$\leq$ 20}     & 34\\
       &            &     &            &  &   7548 &\ldots          &\ldots          &\ldots          &\ldots          &\\
148184 &  B1.5 Ve   & L68 & B2IV:pe    &1O&  15059 &\textbf{108}    &{121}           &{140}           &\ldots          &110\\
       &            &     &            &  &  54090 &\textbf{113}    &{109}           &{144}           &\ldots          &\\
148199 &            &     &            &  &   4561 &\ldots          &{27}            &\textbf{$\leq$ 20}&{$\leq$ 20}     &$\leq$ 20\\
148379 &            &     & B1.5Iape   &  &   4349 &{77}            &\textbf{60}     &{45}            &\ldots          & 60\\
148605 &  B2 V      &  HS & B2V        &  &  39506 &{149}           &\textbf{160}    &{172}           &{212}           &160\\
148688 &  B1 Ia     & W76 & B1Iae      &  &   1871 &{106}           &\textbf{78}     &{46}            &\ldots          & 78\\
148703 &  B2 III    &  HS & B2III-IV   &SB&  39707 &{56}            &\textbf{53}     &{98}            &{151}           & 53\\
149038 &  O9.7 Iab  & W76 & B0Ia       &  &  36166 &\textbf{100}    &{79}            &\ldots          &\ldots          &100\\
149121 &            &     & B9.5III    &  &  19757 &\ldots          &{21}            &\textbf{$\leq$ 20}&{$\leq$ 20}     &$\leq$ 20\\
149438 &  B0.2 V    & W71 & B0V        &  &   9809 &\textbf{$\leq$ 20}&{26}            &{126}           &\ldots          &$\leq$ 20\\
       &            &     &            &  &  39686 &\textbf{$\leq$ 20}&{23}            &{132}           &\ldots          &\\
       &            &     &            &  &  52043 &\textbf{$\leq$ 20}&\ldots          &\ldots          &\ldots          &\\
149485 &            &     & B7Vn       &  &  22216 &\ldots          &{264}           &\textbf{278}    &{286}           &278\\
149630 &            &     & B9V        &  &  29408 &\ldots          &{315}           &{216}           &\textbf{215}    &215\\
149757 &  O9.5 Vnn  & L68 &            &  &  15137 &\textbf{405}    &\ldots          &\ldots          &\ldots          &405\\
149822 &            &     & B9pSiCrSr: &  &  18973 &\ldots          &{86}            &\textbf{41}     &{33}            & 41\\
150168 &            &     & B1Iab-Ib   &SB&  47256 &\textbf{126}    &{107}           &{102}           &\ldots          &126\\
150745 &            &     & B2IV-V     &  &  22218 &{106}           &\textbf{108}    &{125}           &{138}           &108\\
150898 &            &     & B0.5Ia     &  &  50552 &\textbf{108}    &{75}            &{80}            &\ldots          &108\\
151890 &  B1.5 IV   &  HS & B1.5V+B6.5V &2O&  39367 &{217}           &\textbf{223}    &{276}           &\ldots          &223\\
152234 &  B0.5 Ia   & W76 & B0.5Ia     &SB&  16206 &\textbf{100}    &{76}            &{69}            &\ldots          &100\\
152235 &  B0.7 Ia   & W76 & B1Ia       &  &  33294 &\textbf{86}     &{74}            &{63}            &\ldots          & 86\\
152236 &  B1.5 Ia+  & W76 & B1Iape     &  &  54152 &{123}           &\textbf{86}     &{66}            &\ldots          & 86\\
152478 &            &     & B3Vnep     &  &  36197 &{259}           &\textbf{321}    &{326}           &{305}           &321\\
152614 &            &     & B8V        & 2&   6987 &\ldots          &{100}           &\textbf{93}     &{92}            & 93\\
153261 &            &     & B2IVne     &  &  31221 &{215}           &\textbf{241}    &{255}           &\ldots          &241\\
153613 &            &     & B8V        &  &  51847 &\ldots          &{130}           &\textbf{129}    &{129}           &129\\
154090 &  B0.7 Ia   & W76 & B1Ia       &  &  14828 &\textbf{88}     &{77}            &{59}            &\ldots          & 88\\
155763 &  B6 III    & L68 & B6III      &  &   4577 &{33}            &{37}            &\textbf{34}     &{27}            & 34\\
156247 &            &     & B5Vnn+B5V  &2O&  14864 &\ldots          &\textbf{77}     &{453}           &{104}           & 77\\
156633 &  B1.5 Vp   & L68 & B1.5Vp+B5III &2O&  20596 &{110}           &\textbf{128}    &{175}           &{434}           &128\\
157038 &            &     & B4Ia       &  &  30759 &{92}            &\textbf{60}     &{44}            &{97}            & 60\\
157042 &            &     & B2IIIne    &  &  29404 &{237}           &\textbf{273}    &{329}           &\ldots          &273\\
157056 &            &     & B2IV       &SB&   4430 &{29}            &\textbf{24}     &{47}            &{52}            & 24\\
157246 &  B1 Ib     &  HS & B1Ib       &  &  54241 &\textbf{260}    &{198}           &\ldots          &\ldots          &260\\
157741 &            &     & B9V        &  &  17999 &\ldots          &{268}           &{270}           &\textbf{286}    &286\\
158408 &  B2 IV     &  HS & B2IV       &SB&  31685 &{54}            &\textbf{54}     &{104}           &{94}            & 54\\
158427 &  B3 V      &  HS & B2Vne      &SB&  50432 &{192}           &\textbf{240}    &{235}           &{233}           &240\\
158643 &            &     & B9.5Ve     &  &  54144 &\ldots          &{284}           &{151}           &\textbf{167}    &167\\
158926 &            &     & B2IV+B     & 2&  20627 &\textbf{125}    &{136}           &{166}           &\ldots          &125\\
159975 &            &     & B8II-IIIp:Mn &  &   2724 &\ldots          &{100}           &\textbf{86}     &{84}            & 86\\
160578 &  B1.5 III  &  HS & B1.5III    &SB&   4348 &{101}           &\textbf{105}    &{151}           &\ldots          &105\\
160762 &  B3 IV     & L68 & B3IV       &1O&   5720 &{$\leq$ 20}     &\textbf{$\leq$ 20}&{23}            &{$\leq$ 20}     &$\leq$ 20\\
       &            &     &            &  &  29958 &{36}            &\ldots          &\textbf{25}     &{$\leq$ 20}     &\\
162374 &            &     & B6V        &  &  14091 &{55}            &\textbf{44}     &{49}            &{31}            & 44\\
162732 &            &     & BepShell   & O&  44980 &\ldots          &{63}            &{44}            &\textbf{29}     & 29\\
163955 &            &     & B9V        &  &  48332 &\ldots          &{216}           &{120}           &\textbf{121}    &121\\
164284 &  B2 Ve     & L68 & B2Ve       &SB&  15482 &{208}           &\textbf{238}    &{258}           &\ldots          &256\\
       &            &     &            &  &  55946 &{310}           &\textbf{274}    &\ldots          &\ldots          &\\
164353 &  B5 Ib     & L68 & B5Ib       &  &   4267 &{56}            &\textbf{40}     &{34}            &{28}            & 40\\
164402 &  B0 Ib     & W76 & B0Ib       &  &  54242 &\textbf{91}     &{89}            &{80}            &\ldots          & 91\\
164637 &            &     & B0.5III    &SB&  42280 &\textbf{62}     &{52}            &{73}            &\ldots          & 62\\
164852 &  B3 IV     & L68 & B3IV       &2O&  29534 &{122}           &\textbf{156}    &{149}           &{148}           &156\\
165024 &            &     & B2Ib       &  &   6020 &{97}            &\textbf{87}     &{113}           &\ldots          & 87\\
165174 &  B0 IIIn   & L68 & B0IIIn     & ?&   5228 &\textbf{320}    &\ldots          &\ldots          &\ldots          &320\\
165516 &            &     & B0.5Ib     &  &  48276 &\textbf{72}     &{64}            &{85}            &\ldots          & 72\\
165793 &            &     & B1II       &  &  50429 &\textbf{61}     &{51}            &{63}            &\ldots          & 61\\
166014 &            &     & B9.5V      &SB&  20852 &\ldots          &{282}           &{129}           &\textbf{128}    &128\\
166182 &  B2 IV     & L68 & B2IV       &  &  41861 &{51}            &\textbf{37}     &{57}            &{40}            & 37\\
166469 &            &     & B9IVpSrEuCr &  &  48396 &\ldots          &{33}            &\textbf{$\leq$ 20}&{$\leq$ 20}     &$\leq$ 20\\
166596 &            &     & B2.5III    &  &  18143 &{173}           &\textbf{167}    &{181}           &\ldots          &167\\
166937 &            &     & B8Iap      & O&   7080 &{153}           &{49}            &\textbf{43}     &{40}            & 40\\
       &            &     &            &  &  26610 &{59}            &{45}            &\textbf{36}     &{31}            &\\
167128 &  B3 III    &  HS & B3IIIep    &SB&  31220 &{74}            &\textbf{70}     &{82}            &{74}            & 70\\
167264 &            &     & B0Ia       &SB&  24190 &\textbf{91}     &{95}            &\ldots          &\ldots          & 91\\
167756 &            &     & B0.5Ia     &  &  30452 &\textbf{86}     &{74}            &{78}            &\ldots          & 86\\
168905 &            &     & B2.5Vn     &  &  57019 &{200}           &\textbf{227}    &{212}           &{246}           &227\\
169022 &            &     & B9.5III    &  &  44754 &\ldots          &{207}           &{147}           &\textbf{148}    &148\\
169033 &  B8 V      &  HS & B8IV-Ve    &  &  19929 &\ldots          &{174}           &\textbf{144}    &{152}           &144\\
169467 &  B3 IV     &  HS & B3IV       &  &  39861 &{31}            &\textbf{25}     &{33}            &{$\leq$ 20}     & 25\\
170235 &            &     & B2IVpe     &  &  36202 &{143}           &\textbf{155}    &{159}           &{212}           &155\\
170740 &  B2 IV-V   & L68 & B2V        &  &  18314 &{40}            &\textbf{38}     &{72}            &{60}            & 38\\
171406 &  B4 V      & L68 & B4Ve       &  &  54738 &{175}           &\textbf{232}    &{211}           &{226}           &232\\
171961 &            &     & B8III      &  &  45848 &{83}            &{57}            &\textbf{45}     &{37}            & 45\\
172044 &            &     & B8II-IIIpHg & O&   3013 &\ldots          &{33}            &\textbf{25}     &{$\leq$ 20}     & 25\\
172167 &            &     &            &  &  42521 &\ldots          &{23}            &{$\leq$ 20}     &\textbf{$\leq$ 20}&$\leq$ 20\\
       &            &     &            &  &  45285 &\ldots          &{25}            &{$\leq$ 20}     &\textbf{23}     &\\
172910 &            &     & B2.5V      &  &  54739 &{37}            &\ldots          &\textbf{27}     &{$\leq$ 20}     & 27\\
173300 &            &     & B8III      &SB&  34012 &{59}            &{35}            &\textbf{26}     &{$\leq$ 20}     & 26\\
173948 &            &     & B2II-IIIe  &  &  33589 &{140}           &\textbf{134}    &{145}           &\ldots          &134\\
174237 &  B2.5 V    & L68 & B2.5Ve     & O&  14703 &{126}           &\textbf{128}    &{148}           &{146}           &139\\
       &            &     &            &  &  51040 &{144}           &\textbf{150}    &{159}           &{174}           &\\
174632 &            &     & B8V        &  &  44787 &\ldots          &{73}            &\textbf{65}     &{70}            & 65\\
174638 &            &     & B8IIpe     & O&  35836 &\ldots          &\textbf{68}     &\ldots          &{116}           & 68\\
175156 &  B5 II     & L68 & B5II       &  &   4479 &{33}            &\textbf{22}     &{$\leq$ 20}     &{$\leq$ 20}     & 22\\
       &            &     &            &  &  23985 &{$\leq$ 20}     &\textbf{22}     &{$\leq$ 20}     &{$\leq$ 20}     &\\
175362 &            &     & B8IVSi     &  &  44757 &{44}            &\ldots          &\textbf{$\leq$ 20}&{$\leq$ 20}     &$\leq$ 20\\
175640 &            &     & B9III      &  &  13510 &\ldots          &{21}            &\textbf{$\leq$ 20}&{$\leq$ 20}     &$\leq$ 20\\
176162 &  B5 IV     & L68 & B5IV       &SB&  21008 &{31}            &\textbf{24}     &{30}            &{$\leq$ 20}     & 24\\
176437 &  B9 III    &  HS & B9III      &  &  24358 &\ldots          &{63}            &{58}            &\textbf{58}     & 58\\
177003 &  B2.5 IV   & L68 & B2.5IV     &SB&  54674 &{40}            &\textbf{$\leq$ 20}&{24}            &{$\leq$ 20}     &$\leq$ 20\\
177756 &            &     & B9Vn       &  &  18018 &\ldots          &{206}           &\textbf{154}    &{150}           &154\\
177863 &            &     & B8III      &  &  48407 &{83}            &{70}            &\textbf{58}     &{57}            & 58\\
178125 &            &     & B8III      & O&  14866 &{88}            &{54}            &\textbf{51}     &{49}            & 51\\
178175 &  B2 Ve     & L68 & B2Ve       &SB&  15058 &{90}            &\textbf{128}    &{144}           &{177}           &123\\
       &            &     &            &  &  19928 &{106}           &\textbf{118}    &{122}           &{137}           &\\
178475 &  B6 IV     & L68 & B6IV       &  &  20851 &\ldots          &{202}           &\textbf{219}    &{202}           &219\\
179406 &  B3 V      & L68 & B3V        &  &  13865 &{127}           &\textbf{148}    &{138}           &{135}           &166\\
       &            &     &            &  &  36940 &{141}           &\textbf{184}    &{171}           &{154}           &\\
179761 &            &     & B8II-III   &  &  20176 &{72}            &{26}            &\textbf{$\leq$ 20}&{$\leq$ 20}     &$\leq$ 20\\
180163 &  B2.5 IV   & L68 & B2.5IV     &1O&  50600 &{48}            &\textbf{27}     &{26}            &{$\leq$ 20}     & 27\\
180554 &  B4 IV     & L68 & B4IV       &1O&  55228 &{103}           &\textbf{82}     &{83}            &{69}            & 82\\
180968 &  B1 IV     & L68 & B0.5IV     &  &  15214 &{255}           &\textbf{238}    &{234}           &\ldots          &234\\
       &            &     &            &  &  45641 &\textbf{229}    &{226}           &\ldots          &\ldots          &\\
181182 &            &     & B8III+K:   & O&  51044 &\ldots          &{135}           &\textbf{112}    &{99}            &112\\
181558 &            &     & B5V        &  &  47819 &{56}            &{24}            &\textbf{$\leq$ 20}&{$\leq$ 20}     &$\leq$ 20\\
181615 &            &     & B2Vpe+\ldots * & O&  26881 &{273}           &{182}           &\textbf{56}     &{49}            & 56\\
181858 &  B3 V      & W71 & B3IVp      &  &   6037 &{24}            &\textbf{24}     &{27}            &{$\leq$ 20}     & 24\\
181869 &  B9 III    &  HS & B8V        &SB&  44759 &\ldots          &{67}            &\textbf{57}     &{57}            & 57\\
182180 &            &     & B2Vn       &  &  39596 &\ldots          &\textbf{261}    &{263}           &{258}           &261\\
182308 &            &     & B9IVpHgMn  &  &  13512 &\ldots          &{24}            &\textbf{$\leq$ 20}&{$\leq$ 20}     &$\leq$ 20\\
183056 &            &     & B9pSi      & O&  14982 &\ldots          &{121}           &\textbf{77}     &{57}            & 77\\
183362 &  B3 Ve     & L68 & B3Ve       &  &  31218 &{242}           &\textbf{233}    &{250}           &{259}           &233\\
184171 &  B3 IV     & L68 & B3IV       &  &  48876 &{40}            &\textbf{26}     &{25}            &{$\leq$ 20}     & 26\\
184279 &  B1 IV     & W71 &            &  &   6034 &{150}           &\textbf{68}     &{63}            &{49}            & 68\\
184606 &            &     & B8IIIn     &  &   4559 &\ldots          &{188}           &\textbf{202}    &{244}           &202\\
185037 &  B8 V      &  HS & B8Vne      &  &  26897 &\ldots          &{239}           &\textbf{245}    &{300}           &245\\
185330 &            &     & B5II-III   &  &  18992 &{42}            &{$\leq$ 20}     &\textbf{$\leq$ 20}&{$\leq$ 20}     &$\leq$ 20\\
185859 &  B0.5 Ia   & L68 & B0.5Iae    &  &  14207 &\textbf{76}     &{63}            &{79}            &\ldots          & 76\\
185915 &  B6 IV     & L68 & B6IV       &SB&  14208 &{90}            &{65}            &\textbf{64}     &{57}            & 64\\
186122 &            &     & B9IIIpHgMn &  &   9528 &\ldots          &{$\leq$ 20}     &\textbf{$\leq$ 20}&{$\leq$ 20}     &$\leq$ 20\\
186500 &            &     & B8V        &  &  50991 &{149}           &{117}           &\textbf{96}     &{99}            & 96\\
186882 &  B9.5 III  &  HS & B9.5IV+F1V &SB&   9597 &\ldots          &{216}           &{134}           &\textbf{143}    &142\\
       &            &     &            &  &  14844 &\ldots          &{198}           &{127}           &\textbf{140}    &\\
186994 &  B0.2 IV   & W71 &            &  &   5654 &\textbf{149}    &{121}           &{131}           &\ldots          &149\\
187235 &            &     & B8Vn       &  &  26896 &\ldots          &{287}           &\textbf{304}    &{301}           &304\\
187567 &  B2.5 IVe  & L68 & B2.5IVe    &  &  36401 &{144}           &\textbf{174}    &{210}           &\ldots          &174\\
187811 &  B2.5 V    & L68 & B2.5Ve     & ?&  21488 &{167}           &\textbf{221}    &{231}           &{262}           &221\\
187879 &  B1 III    & L68 & B1III+B3V  & O&   6605 &{97}            &\textbf{87}     &{102}           &\ldots          & 92\\
       &            &     &            &  &  38552 &{107}           &\textbf{98}     &{123}           &\ldots          &\\
188439 &  B0.5 IIIn & L68 & B0.5IIIn   &SB&  10852 &{307}           &\textbf{229}    &\ldots          &\ldots          &229\\
188665 &  B5 V      & L68 & B5V        &  &   8919 &{118}           &{103}           &\textbf{105}    &{110}           &105\\
188892 &  B5 IV     & L68 & B5IV       &  &  20974 &{59}            &\textbf{38}     &{36}            &{30}            & 38\\
189103 &            &     & B3IV       & O&  52095 &{36}            &\textbf{24}     &{30}            &{$\leq$ 20}     & 24\\
189395 &            &     & B9Vn       &  &   4966 &\ldots          &{164}           &\textbf{151}    &{150}           &151\\
189687 &  B3 IV     & L68 & B3IVe      &SB&  19936 &{164}           &\textbf{206}    &{190}           &{185}           &206\\
190229 &            &     & B9pHgMn    & 1&   4962 &\ldots          &{23}            &\textbf{$\leq$ 20}&{$\leq$ 20}     &$\leq$ 20\\
190603 &  B1.5 Ia   & W71 & B1.5Ia     &  &  14942 &{82}            &\textbf{70}     &{82}            &\ldots          & 70\\
190993 &  B3 V      & L68 & B3V        &SB&   9960 &{105}           &\textbf{114}    &{125}           &{116}           &114\\
191456 &  B0.5 II-III & W71 &            &  &   6136 &\textbf{56}     &{53}            &{75}            &\ldots          & 56\\
191610 &  B2.5 V    & L68 & B2.5Ve     & ?&   8600 &{251}           &\textbf{260}    &{260}           &{304}           &245\\
       &            &     &            &  &  37154 &{185}           &\textbf{230}    &{242}           &{260}           &\\
191877 &  B1 Ib     & L68 & B1Ibe      &  &  14825 &{159}           &\textbf{124}    &{134}           &\ldots          &124\\
192044 &            &     & B7Ve       &  &  28251 &\ldots          &\ldots          &\ldots          &\ldots          &\ldots\\
192907 &            &     & B9III      &  &   5719 &\ldots          &{24}            &{$\leq$ 20}     &\textbf{$\leq$ 20}&$\leq$ 20\\
193237 &  B2 pe     & L68 & B2pe       &  &  43339 &{101}           &\textbf{70}     &\ldots          &\ldots          & 70\\
193432 &            &     & B9.5V      &  &  32300 &\ldots          &\ldots          &\ldots          &\ldots          &\ldots\\
193911 &            &     & B8IIIne    &  &  29406 &{118}           &{155}           &\textbf{148}    &{154}           &148\\
194092 &  B0.5 V    & W71 &            &  &  16213 &\textbf{$\leq$ 20}&{23}            &{143}           &\ldots          &$\leq$ 20\\
194335 &  B2 Vne    & L68 & B2Ven      &SB&  52946 &{245}           &\textbf{315}    &{334}           &\ldots          &315\\
194636 &            &     & B8II-III   &  &  47832 &{88}            &{38}            &\textbf{26}     &{$\leq$ 20}     & 26\\
195810 &  B6 III    & L68 & B6III      &  &  52067 &{98}            &{47}            &\textbf{44}     &{34}            & 44\\
196426 &            &     & B8IIIp     &  &  19759 &{48}            &{23}            &\textbf{$\leq$ 20}&{$\leq$ 20}     &$\leq$ 20\\
196519 &            &     & B9III      &  &  48309 &\ldots          &{198}           &{155}           &\textbf{140}    &140\\
196740 &  B5 IV     & L68 & B5IV       &  &  18017 &\ldots          &{226}           &\textbf{271}    &{261}           &271\\
196867 &  B9 IV     &  HS & B9IV       &SB&  24451 &\ldots          &{155}           &{112}           &\textbf{111}    &111\\
197018 &            &     & B6IIIpMn   &  &  14051 &{59}            &{42}            &\textbf{40}     &{30}            & 40\\
197511 &  B2 V      & L68 & B2V        &  &  10844 &{33}            &\textbf{37}     &{54}            &{36}            & 37\\
197702 &  B1 III(n) & W71 &            &  &  11290 &{277}           &\textbf{264}    &{359}           &\ldots          &264\\
197770 &  B2 III    & L68 & B2III      &SB&  49267 &{74}            &\textbf{67}     &{89}            &\ldots          & 67\\
198174 &            &     & B7IIIp     &  &  51082 &\ldots          &{47}            &\textbf{41}     &{31}            & 41\\
198183 &  B6 IV     & L68 & B5Ve       & ?&  14209 &{119}           &{108}           &\textbf{104}    &{108}           &102\\
       &            &     &            &  &  25780 &{113}           &{102}           &\textbf{101}    &{97}            &\\
198478 &  B2.5 Ia   & W71 & B3Ia       &  &  13907 &{77}            &\textbf{55}     &{46}            &{46}            & 53\\
       &            &     &            &  &  38688 &{70}            &\textbf{51}     &{39}            &{46}            &\\
198625 &  B4 V      & L68 & B4Ve       &  &  36195 &{150}           &\textbf{207}    &{232}           &{218}           &207\\
198667 &            &     & B9III      &  &  14045 &\ldots          &{32}            &\textbf{$\leq$ 20}&{$\leq$ 20}     &$\leq$ 20\\
198820 &  B3 III    & L68 & B3III      &  &   8331 &{39}            &\textbf{27}     &{26}            &{23}            & 32\\
       &            &     &            &  &  19872 &{57}            &\textbf{38}     &{41}            &{30}            &\\
199081 &  B5 V      & L68 & B5V        &2O&  10841 &\ldots          &{128}           &\textbf{146}    &{176}           &146\\
199140 &  B2 III    & L68 & B2IIIe     &  &   5600 &\textbf{52}     &{43}            &{88}            &\ldots          & 40\\
       &            &     &            &  &  52885 &{33}            &\textbf{29}     &{48}            &\ldots          &\\
199218 &            &     & B8Vnne     &  &  30071 &\ldots          &{281}           &\textbf{275}    &{237}           &275\\
199478 &            &     & B8Ia       &  &  15552 &{60}            &\textbf{46}     &{43}            &{31}            & 46\\
199661 &  B2.5 IV   & L68 & B2.5IV     &  &  10845 &{115}           &\textbf{110}    &{126}           &{372}           &110\\
199955 &            &     & B5Vn       &  &  10862 &\ldots          &{134}           &\textbf{99}     &{104}           & 99\\
200120 &  B1.5 Vnne & L68 & B1ne       &SB&  17094 &\textbf{341}    &{353}           &\ldots          &\ldots          &335\\
       &            &     &            &  &  52944 &\textbf{329}    &{356}           &{382}           &\ldots          &\\
200310 &  B1 Vn     & L68 & B1Ve       &SB&  10853 &{288}           &\textbf{267}    &{293}           &\ldots          &267\\
201345 &  ON9 V     & W76 &            &  &  15004 &\textbf{100}    &{95}            &{119}           &\ldots          &100\\
201908 &            &     & B8Vn       &SB&  45273 &\ldots          &{88}            &{82}            &\textbf{93}     & 93\\
202214 &  B0 V      & L68 & B0II       &  &  21011 &\textbf{38}     &{30}            &{55}            &\ldots          & 38\\
202654 &  B4 IV     & L68 & B4IV       &SB&  22646 &{114}           &\textbf{123}    &{127}           &{105}           &123\\
202753 &            &     & B5V        &  &  52920 &{36}            &{36}            &\textbf{34}     &{$\leq$ 20}     & 34\\
202850 &            &     & B9Iab      &SB&  13460 &\ldots          &\ldots          &\ldots          &\ldots          &\ldots\\
202904 &  B2 Ve     & L68 & B2Vne      & ?&   8601 &{138}           &\textbf{165}    &{177}           &{196}           &165\\
203245 &  B6 V      & L68 & B6V        &SB&  22755 &{109}           &{67}            &\textbf{66}     &{59}            & 66\\
203467 &  B3 IVe    & L68 & B3IVe      &SB&  42020 &{127}           &\textbf{142}    &{152}           &{148}           &142\\
203532 &            &     & B3IV       &  &  16399 &{65}            &\textbf{59}     &{63}            &{49}            & 59\\
203664 &  B0.5 III(n) & W71 &            &  &   7355 &\textbf{206}    &{172}           &{206}           &\ldots          &206\\
204172 &  B0 Ib     & W76 & B0Ib       &  &   6481 &\textbf{95}     &{78}            &{86}            &\ldots          & 95\\
       &            &     &            &  &  48946 &\textbf{95}     &{88}            &{83}            &\ldots          &\\
204770 &  B7 V      & L68 & B7V        &  &   8922 &\ldots          &{207}           &\textbf{172}    &{167}           &172\\
205021 &  B1 III    & L68 & B1IV       & O&   4609 &\textbf{31}     &{34}            &{96}            &\ldots          & 35\\
       &            &     &            &  &   6235 &\textbf{38}     &{33}            &{59}            &\ldots          &\\
       &            &     &            &  &  52941 &\textbf{36}     &{29}            &{88}            &\ldots          &\\
205139 &  B1 Ib     & L68 & B1II       &  &  46540 &\textbf{65}     &{51}            &{74}            &\ldots          & 65\\
205637 &  B2.5 Vp   & L68 & B2.5Vpe    & ?&  34400 &{176}           &\textbf{165}    &{139}           &{122}           &165\\
206165 &  B2 Ib     & W71 & B2Ib       &  &   6336 &{78}            &\textbf{63}     &{52}            &\ldots          & 63\\
206540 &            &     & B5IV       &  &  14023 &{57}            &{25}            &\textbf{$\leq$ 20}&{$\leq$ 20}     &$\leq$ 20\\
206672 &  B3 IV     & L68 & B3IV       & O&  24332 &{77}            &\textbf{75}     &{92}            &{74}            & 75\\
207330 &  B2.5 III  & L68 & B3III      & O&   9210 &{67}            &\textbf{45}     &{50}            &{38}            & 45\\
207857 &            &     & B9pHgMn    &  &  34436 &\ldots          &\ldots          &\ldots          &\ldots          &\ldots\\
207971 &  B8 III    &  HS & B8III      &  &  44768 &\ldots          &{51}            &\textbf{41}     &{33}            & 41\\
208057 &  B3 V      & L68 & B3Ve       &SB&   5909 &{101}           &\textbf{89}     &{98}            &{101}           & 87\\
       &            &     &            &  &  33664 &{90}            &\textbf{85}     &{94}            &{83}            &\\
208501 &            &     & B8Ib       &  &   4217 &\ldots          &\textbf{38}     &{33}            &{40}            & 38\\
208682 &  B2.5 Ve   & L68 & B2.5Ve     &  &  52942 &{224}           &\textbf{250}    &{270}           &\ldots          &250\\
209008 &  B3 III    & L68 & B3III      &  &  20593 &{37}            &\textbf{23}     &{26}            &{$\leq$ 20}     & 23\\
209014 &            &     & B8V        &  &  31219 &\ldots          &{264}           &\textbf{264}    &{228}           &264\\
209409 &  B7 IVe    & L68 & B7IVe      &  &   5912 &\ldots          &{184}           &\textbf{185}    &{206}           &185\\
       &            &     &            &  &  39200 &\ldots          &{223}           &\textbf{185}    &{155}           &\\
209419 &  B5 III    & L68 & B5III      &  &   8921 &{40}            &{27}            &\textbf{26}     &{$\leq$ 20}     & 26\\
209459 &            &     & B9.5V      &  &  21398 &\ldots          &{21}            &{$\leq$ 20}     &\textbf{$\leq$ 20}&$\leq$ 20\\
209522 &            &     & B4IVne     &  &  52358 &{274}           &\textbf{255}    &{286}           &{368}           &255\\
209952 &  B7 IV     &  HS & B7IV       &  &  48408 &\ldots          &{202}           &\textbf{192}    &{206}           &192\\
210129 &            &     & B7Vne      &  &  44770 &{132}           &{130}           &\textbf{106}    &{118}           &106\\
210191 &  B2.5 IV   & L68 & B2.5IV     &  &  23304 &{37}            &\textbf{24}     &{23}            &{$\leq$ 20}     & 24\\
210424 &            &     & B7III      &  &   9041 &{35}            &{22}            &\textbf{$\leq$ 20}&{$\leq$ 20}     &$\leq$ 20\\
211924 &  B5 IV     & L68 & B5IV       &  &  15217 &{53}            &{44}            &\textbf{44}     &{30}            & 44\\
212120 &  B6 V      & L68 & B6V        &2O&  26322 &{96}            &{67}            &\textbf{63}     &{59}            & 63\\
212454 &            &     & B8III-IV   &  &  14974 &{57}            &{41}            &\textbf{31}     &{30}            & 31\\
212571 &  B1 Ve     & L68 & B1Ve       &  &   7009 &\textbf{268}    &{276}           &{283}           &\ldots          &272\\
       &            &     &            &  &  54752 &\textbf{275}    &{291}           &\ldots          &\ldots          &\\
212581 &  B9.5 V    &  HS & B9.5V      &  &  26185 &\ldots          &{234}           &{199}           &\textbf{194}    &194\\
212710 &            &     & B9.5Vn     & 2&  48238 &\ldots          &\ldots          &\ldots          &\ldots          &\ldots\\
212978 &  B1.5 V    & L68 & B2V        &  &   6237 &{113}           &\textbf{102}    &{121}           &{140}           &102\\
213420 &  B2 IV     & L68 & B2IV       & O&  20267 &{79}            &\textbf{60}     &{85}            &{82}            & 60\\
214168 &  B1.5 V    & L68 & B2Ve       &SB&  49149 &{288}           &\textbf{295}    &{274}           &\ldots          &295\\
214240 &  B3 IV     & L68 & B3V        &2O&  44796 &{80}            &\textbf{56}     &{54}            &{48}            & 56\\
214923 &  B8 V      &  HS & B8V        &  &  42071 &\ldots          &{185}           &{129}           &\textbf{134}    &134\\
214993 &  B1.5 III  & W71 & B2III      &SB&   5407 &{74}            &\textbf{60}     &{106}           &\ldots          & 60\\
215573 &            &     & B6IV       &  &  29817 &{39}            &{22}            &\textbf{$\leq$ 20}&{$\leq$ 20}     &$\leq$ 20\\
216200 &  B4 III    & L68 & B3IV:      & 2&   9205 &{134}           &\textbf{161}    &{177}           &{194}           &161\\
216494 &            &     & B9III      &2O&  55266 &\ldots          &{85}            &\textbf{64}     &{60}            & 64\\
216916 &  B2 IV     & L68 & B2IV       &1O&   5361 &\textbf{24}     &\ldots          &{46}            &\ldots          & 24\\
217101 &  B2 IV-V   & L68 & B2IV-V     &  &  18309 &{99}            &\textbf{109}    &{127}           &{156}           &109\\
217543 &  B3 Vp     & L68 & B3Vpe      &SB&  33666 &{183}           &\textbf{270}    &{245}           &{194}           &270\\
217675 &            &     & B6IIIpe+A2p &2O&  52943 &{113}           &\textbf{181}    &{164}           &{162}           &181\\
217833 &            &     & B9IIIHe wk &  &  14972 &{39}            &{30}            &\textbf{25}     &{$\leq$ 20}     & 25\\
217891 &  B6 Ve     & L68 & B6Ve       &  &  15512 &{61}            &{81}            &\textbf{82}     &{72}            & 82\\
218045 &  B9.5 III  &  HS & B9V        &SB&  41819 &\ldots          &{263}           &{100}           &\textbf{107}    &107\\
218376 &  B0.5 III  & W71 & B0.5IV     &  &   5363 &\textbf{55}     &{46}            &{85}            &\ldots          & 55\\
219188 &  B0.5 \ldots * & W71 &            &  &   7711 &\textbf{278}    &{230}           &\ldots          &\ldots          &278\\
219688 &  B5 Vn     & L68 & B5V        &  &  10385 &\ldots          &\textbf{200}    &{217}           &{290}           &200\\
219749 &            &     & B9pSi      &1O&   6095 &\ldots          &{74}            &\textbf{73}     &{61}            & 73\\
220599 &            &     & B9III      &  &  48983 &\ldots          &\ldots          &\ldots          &\ldots          &\ldots\\
220885 &            &     & B9III      &SB&  14970 &\ldots          &{66}            &\textbf{53}     &{47}            & 53\\
221253 &  B3 IV     & L68 & B3IV       &1O&  51974 &{129}           &\textbf{133}    &{127}           &{133}           &133\\
221507 &            &     & B9.5IVpHgMnEu &  &  18991 &\ldots          &{35}            &\textbf{$\leq$ 20}&{$\leq$ 20}     &$\leq$ 20\\
222173 &            &     & B8V        &SB&  10376 &\ldots          &{60}            &\textbf{51}     &{51}            & 51\\
222439 &            &     & B9IVn      &  &  52658 &\ldots          &{196}           &{144}           &\textbf{143}    &143\\
222661 &  B9.5 V    &  HS & B9.5V      &SB&  24450 &\ldots          &{243}           &{131}           &\textbf{129}    &129\\
223640 &            &     & B9pSiSrCr  &  &   5723 &\ldots          &{40}            &\textbf{26}     &{$\leq$ 20}     & 26\\
223987 &  B1 II-III & W71 &            &  &  13763 &{105}           &\textbf{74}     &{67}            &\ldots          & 74\\
224112 &            &     & B8V        &  &   5718 &\ldots          &{37}            &\textbf{52}     &{30}            & 52\\
224113 &            &     & B6V        &1O&   5738 &{89}            &{67}            &\textbf{63}     &{57}            & 63\\
224151 &  B0.5 II-III & L68 & B0.5II+B0.5II & O&  26321 &\textbf{102}    &{77}            &{82}            &\ldots          &102\\
224572 &  B1 V      & L68 & B1V        & 2&  20971 &\textbf{142}    &{141}           &{164}           &\ldots          &142\\
224686 &  B9 IV     &  HS & B9IV       &  &  22848 &\ldots          &{257}           &{240}           &\textbf{207}    &207\\
225094 &  B3 Iab    & L68 & B3Ia       &  &   9415 &{113}           &\textbf{74}     &{62}            &{63}            & 65\\
       &            &     &            &  &  18688 &{78}            &\textbf{56}     &{52}            &{51}            &\\
225253 &            &     & B8IV-V     &  &   4413 &{$\leq$ 20}     &{25}            &\textbf{$\leq$ 20}&{$\leq$ 20}     &$\leq$ 20\\
269128 &            &     &            &  &   7138 &{130}           &\textbf{122}    &{140}           &{92}            &122\\


\end{longtable}
\normalsize

\noindent
\tref{GG -- \ybcite{gg}; HS -- \ybcite{hs}; L68 -- \ybcite{l68}; W71
-- \ybcite{wa:71supp}; -- W76 -- \ybcite{wa:76}}

\noindent
\emph{HD 358}. GG spectral type is  kB9hB8HeB9 III. \\
\emph{HD 12767}. GG spectral type is  kB8hB8heB9 III. \\
\emph{HD 21699}. GG spectral type is  kB8hB7HeB9.5 III. \\
\emph{HD 181615}. BSC spectral type is B2Vpe+A2IaShell.\\
\emph{HD 219188}. W71 spectral type is  B0.5 II-III(n). \\








