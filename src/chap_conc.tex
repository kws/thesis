\chapter{Summary}

\section{Physical parameters of O stars}

Modern computer models are probably sufficiently advanced for reliable
--- or, at least, consistent --- surface parameters to be derived by
comparing observations to models over much of the H--R diagram. In any
case, the choice of stars over a very limited part of the H--R diagram
allows us to derive good differential parameters for these objects. In
chapters \ref{ch:mod} and \ref{ch:casestudy} we developed methods to
find the `best-fit' model spectrum from a four-dimensional grid of
models. By comparing equivalent widths of helium lines (both \hei\ and
\heii) and fitting the wings of hydrogen lines, fit-diagrams were
constructed that gave possible values for the physical parameters of
the stars. Series of such diagrams were used to constrain the solution
in all four model dimensions, namely effective temperature, surface
gravity, helium number fraction and microturbulence.

%%%%%%%% SUBSECTION
\subsection{Temperatures and surface gravities}

Effective temperatures and surface gravities were derived for the 23
WHT/UES O-star sample (chapter \ref{ch:analyses}). Table
\ref{ta:conc:params} lists the derived temperatures and gravities for
each spectral type. Comparisons with other studies using non-LTE
models show good agreement, but it appears that the inclusion of
microturbulence together with an objective procedure to locate the
`best' parameters, causes us to derive temperatures that are slightly
higher than in other studies.

Using the derived surface parameters and observed/estimated absolute
magnitudes, $M_V$, it is possible to calculate the stars'
luminosities, masses and radii. These values have been compared to
`traditional' theoretical evolutionary models, and `evolutionary masses'
derived. There appears to be a discrepancy between evolutionary
predictions and masses derived, except for a small subgroup of
nitrogen deficient stars that all are in good agreement with
theoretical models (section \ref{sec:mass_discrepancy}). Recent models
(\cite{ma:98}; \cite{me:98}) including the effects of stellar
rotation, go some way towars resolving this `mass discrepancy'. They
predict that rotating stars will show an increase in luminosity,
which would in turn reduce the  masses derived using the above
method. This, coupled with advances in stellar atmospheric modelling,
will probably bring evolutionary and spectroscopic masses to a better
agreement.

\begin{table} %%%%%%%%%%%%% TABLE
\begin{center}
\caption{Summary of observed parameters}
\label{ta:conc:params}
\vspace{\abovecaptionskip}
\begin{tabular}{l|rrr|rrr|rrr}
\hline\hline
\multicolumn{1}{c}{}&
\multicolumn{3}{c}{I}&
\multicolumn{3}{c}{II -- III} &
\multicolumn{3}{c}{IV -- V} \\
\multicolumn{1}{c}{Sp. type} &
$N$ & \multicolumn{1}{c}{\teff} & \multicolumn{1}{c}{\logg} &
$N$ & \multicolumn{1}{c}{\teff} & \multicolumn{1}{c}{\logg} &
$N$ & \multicolumn{1}{c}{\teff} & \multicolumn{1}{c}{\logg} \\
\hline
O8   & 1 & 38.0 & 3.4 & & & & & & \\
O9   & 2 & 36.0 & 3.3 & 1 & 37.0 & 3.6 & 3 & 36.7 & 4.1 \\
O9.5 & 5 & 33.8 & 3.2 & 3 & 34.0 & 3.4 & 1 & 37.0 & 4.2 \\
O9.7 & 5 & 32.0 & 3.1 & 2 & 32.8 & 3.3 & & & \\

\hline\hline
\end{tabular}
\end{center}

\tnote{The table shows the average (mean) \teff\ and \logg\ for
each spectral type and luminosity class. $N$ is the number of stars
observed in each group. Effective temperatures are in kK.}
\end{table}  %%%%%%%%%%%%% END TABLE


%%%%%%%% SUBSECTION
\subsection{Helium abundances and the evolutionary connection}

Helium abundances have also been derived, and although it is difficult
to evaluate the accuracy of the absolute values, the differential
results should be accurate to about 0.02. A link between the
morphologically peculiar ON/OC stars and the helium abundance was found
(section \ref{sec:cno_anomalies}), suggesting the unusual surface composition
is caused by the exposure of CNO processed material on the surface of
the stars. Estimates of the stellar ages from comparisons with
evolutionary models suggest there could be a link between age and
observed increased abundances; however the OC supergiants and ON
main-sequence stars do not appear to follow such trends.

A relation was also found between the line broadening velocity (\vsini\
or macroturbulence) and the helium abundance (section
\ref{sec:odiscuss:rotation}). Several different theories which could
explain these enhanced abundances as `illusions' caused by blending
effects or stellar asymmetries have been explored, but none was found that
is able to reproduce the observed abundances. It therefore appears
that rotation and surface chemistry are linked, something which is
also indicated by the new evolutionary models mentioned above.


%%%%%%%% SUBSECTION
\subsection{Microturbulence}

The inclusion of microturbulence in the detailed line-formation
calculations has several implications on the derived results. The
strengthening of the helium lines when microturbulence is included
leads to derived helium abundances that are closer to solar
values. The lower helium abundances lead to slightly higher
temperatures, since the increase in equivalent width predicted for the
\heii\ lines, which are the main temperature indicators, is less than
that predicted for the \hei\ lines. A lower helium abundance leads to
weaker \heii\ lines, and the model temperatures have to be
increased (typically by about 1~--~2~kK) to match the observed
equivalent widths (see also section \ref{sec:analyses:mturb}). 

Microturbulence is more important in some \hei\ lines than
others. This relative strengthening improves the model fits in the
lines that are affected by the so-called \emph{generalized dilution
effect}, although the models, in general, still underestimate the
equivalent widths of these lines.

We find high microturbulence ($\sim 15$~\kms) for all the stars in the
sample, including the main-sequence stars (although the
microturbulence is of little importance in these stars).


%%%%%%%%%%%%%%%%%%%%%%%%%%%%%%%%%%%%%%%%%%%%%%%%%%%%%%%%
%%%%%%%%%%%%% LINE BROADENING %%%%%%%%%%%%%%%%%%%%%%%%%%
%%%%%%%%%%%%%%%%%%%%%%%%%%%%%%%%%%%%%%%%%%%%%%%%%%%%%%%%
\section{Line broadening properties}

%%%%%%%% SUBSECTION
\subsection{B-star survey}

An automated survey of the cross-correlation characteristics of a
large sample ($> 700$) B-stars was undertaken (chapter
\ref{ch:xcorr}). Line-broadening velocities were calculated and the
velocity distributions of subsamples of the dataset examined. The
velocity distributions for main-sequence stars and supergiants appear
to differ significantly. The main-sequence stars have a broad velocity
distribution, containing both the slowest and the fastest rotators of
the sample. The supergiants were found to have a much smaller range of
velocities, with a lack of slow rotators among the early-type
stars. The minimum velocity for the supergiants appears to be a
function of spectral type, increasing towards earlier types. This is
consistent with the existence of macroturbulent velocity fields in
early-type supergiants. These velocity fields (of unknown origin)
would not show the same inclination dependence as rotational velocity
(\vsini), and would therefore be the lower limit to the velocity
distributions.


%%%%%%%% SUBSECTION
\subsection{Rotation vs. macroturbulence}

As part of the spectroscopic analyses of the 23 O-stars, the model
spectra were convolved with two different functions (section
\ref{sec:casestudy:broadening}), one being the theoretical function for
a line broadened by stellar rotation, and the other the broadening
caused by a Gaussian velocity field in the stellar atmosphere
(simulating macroturbulence). These convolved models were compared to
the observations using a $\chi^2$ method to estimate the
goodness-of-fit. The results for the main-sequence stars were
inconclusive, due to the low number of stars, but tentatively suggest
that rotation is the dominant source of line broadening in these
stars. For the supergiants, all but the two very fastest rotators have
macroturbulent line profiles, and we conclude that in early-type stars
there appears to be some process, apart from rotation, that
contributes to the line broadening.


