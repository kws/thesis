\chapter{Observations and data reduction}
\label{ch:obs}

\section{WHT observations of late O-type stars}

The high-resolution, high signal-to-noise data needed for the line
analysis were acquired using the Utrecht Echelle Spectrograph (UES)
during four nights in August 1995. The UES is permanently mounted on
one of the nasmyth platforms at the f/11 focus of the 4.2 metre
William Herschel Telescope (WHT), which is part of the Isaac Newton
Group (ING) of telescopes at La Palma.

One of the main criteria for selecting target stars was brightness. In
order to get high-quality spectra of as many stars as possible, it was
important to keep exposure times to a minimum. After the second night
of observing, it was decided to remove the image de-rotator from the
instrument. This disabled the WHT's auto-guider, forcing us to guide
the telescope manually, but it also increased the throughput (i.e. reduced
the exposure times) by about 30 per cent.

Most of the four nights were spent observing with the UES, however
several lower resolution spectra were acquired with both the red and blue
arms of the ISIS double beam spectrograph. These spectra were not used
for the analysis discussed in this thesis, but they were obtained to
help define the wings of the broad hydrogen lines. They have also
sometimes been used as a comparison to check for faults in the reduction of
the high-resolution data, and to help correct for the echelle blaze in
the data. A special routine was later developed to automatically
de-blaze the spectra (see section \ref{sec:blaze}).

\subsection{The sample}

One of the main aims of this thesis is to look for object-to-object
variations in the helium abundance of O stars. To look for
evolutionary tracers, we wanted a large sample of stars that populate
a small part of the HR diagram, and with a bias towards stars that
have been proposed to show abnormal surface abundances. By chosing
stars with similar effective temperatures and surface gravities, we
hope that reliable \emph{differential} parameters for the stars can be
derived.  To test the hypothesis that OC -- O -- ON is an evolutionary
sequence, and that the nitrogen enhancements should be reflected in
the helium abundance, several stars that were classified as OC and ON
by \ybcite{wa:76} were chosen. The ``new'' ON star identified in
\ybcite{ma:89}, BD+36~4063, was also observed, and recent observations
show that this star is a single-lined spectroscopic binary (Howarth,
private communication).  See table
\ref{tab:obs_targets} for a complete list of the UES target stars.

These stars are probably not a representative sample of the O
stars. There are only 4 stars (roughly 17~\% of the sample) classified
as belonging to luminosity class V, corresponding to a main sequence
star. It is difficult to say what the ``true'' distribution of
luminosity classes are, but as a quick estimate we can check with
Garmany's unpublished catalogue of O stars. This catalogue contains a
more or less complete sample out to approximately 2~kpc
(\cite{ga:imf}) and although the spectral types are not always
consistent with those we have adopted (see table
\ref{tab:obs_targets}) there are probably no important differences. For the
list containing about 1000 stars (some without luminosity
classifications), there are 375 stars classified as either If, I, II
or III and 394 as IV or V.

The sample does, however, contain a mix of morphologically normal
stars, and stars which have been at one point or another been
classified as chemically peculiar. These stars include the ON and OC
stars, the Orion belt stars, $\delta$ Ori and $\zeta$ Ori, and several
other stars which have been noted to have moderate chemical
anomalies. In particular, the Orion belt stars are interesting
because, although not classified as OCs, they both have been noted by
\ybcite{wa:76} to be slightly N deficient.


%%%%%%%%%%%%%%%%%%%%%
\begin{figure}
\epsfxsize=\figwidth
\setlength{\cen}{(\textwidth / 2) - (\epsfxsize / 2)}
\hspace{\cen}\epsfbox{fig_obs/hrd/hrd.eps}
\caption[HR-diagram]
{\fcfont HR Diagram showing the positions of the sample stars as open
symbols. The physical parameters for the stars are calculated in this
thesis. Also indicated are the core hydrogen (H-b) and helium burning
(He-b) phases of the evolutionary tracks, and predicted positions of
the zero-age main-sequence (ZAMS), dredge-ups (D-up), extended part of
asymptotic giant branch (EAGB), thermal pulsing part of the AGB
(TPAGB), and the positions of planetary nebulae (PNs), Wolf-Rayet
stars (WRs), luminous blue variables (LBVs) and the white dwarf (WD)
cooling track. The thick black lines denote `slow' parts of the
evolutionary tracks, i.e. where the stars are most likely to be
observed. Diagram adapted from Chiosi (1998).}
\label{fig:hrd1}
\end{figure}
%%%%%%%%%%%%%%%%%%%%%%

%%%%%%%%%%%%%%%%%%%%%
\begin{figure}
\epsfxsize=\figwidth
\setlength{\cen}{(\textwidth / 2) - (\epsfxsize / 2)}
\hspace{\cen}\epsfbox{fig_obs/hrd/hrd2.eps}
\caption[Spectral types and absolute magnitudes of WHT sample]
{\fcfont Absolute magnitudes and spectral types for the sample
stars. The magnitudes are from Garmany's unpublished catalogue, and
the spectral types are taken from the various references discussed in
the text. The lines show the loci of absolute magnitudes adopted by
Howarth and Prinja (1989).}
\label{fig:hrd2}
\end{figure}
%%%%%%%%%%%%%%%%%%%%%%

%%%%%%%%%%%%%%%%%%%%%%%%%%%%%%%%%%%%%%%%%%%%%%%%%%%%%%%%%%%%%%%%%%%%%%%%%%%%%%%%
\begin{table}
\begin{center}
\small % SMALL!!!!!!!!!!
\caption{UES target stars}
\begin{tabular}{rllrccrrrr}
\\ \hline\hline
HD/BD & \centering{Name} & \multicolumn{2}{c}{Spectral Type} & V &
\multicolumn{2}{c}{No. exposures} & \multicolumn{2}{c}{Tot. time} & S/N\\
& & & & & 4190\AA & 5426\AA &
4190\AA & 5426\AA \\

\hline

10125   &            	& O9.7 II       & W76 & 8.22 & 2 & 2 & 3000 & 2700 & 300 \\
12323   & 		& ON9 V         & W76 & 8.90 & 2 & 2 & 2700 & 2400 & 300 \\
13745   &         	& O9.7 II((n))  & W76 & 7.88 & 2 & 2 & 1600 & 1500 & 450 \\
16429   &               & O9.5 II((n))  & W76 & 7.67 & 2 & 2 & 2700 & 1500 & 300 \\
30614   & $\alpha$ Cam	& O9.5 Ia       & W76 & 4.26 & 2 & 2 &  160 &  100 & 400 \\
34078   & AE Aur 	& O9.5 V        & W73 & 5.81 & 2 & 2 &  420 &  350 & 300 \\
36486   & $\delta$ Ori  & O9.5 II       & W76 & 2.20 & 3 & 2 &   30 &   10 & 350 \\
37742   & $\zeta$ Ori   & O9.7 Ib       & W76 & 1.75 & 2 & 3 &   24 &   19 & 300 \\
188209  & 		& O9.5 Iab      & W76 & 5.64 & 3 & 3 &  550 &  400 & 350 \\
189957  &	        & O9.5 III      & W73 & 7.22 & 3 & 2 & 1200 & 1000 & 300 \\
191781  & 	        & ON9.7 Iab     & W76 & 9.54 & 3 & 2 & 4500 & 2500 & 300 \\
194280  &               & OC9.7 Iab     & W76 & 8.39 & 3 & 2 & 4500 & 3000 & 200 \\
195592  &               & O9.7 Ia       & W76 & 7.08 & 2 & 2 & 2000 &  500 & 300 \\
201345  &               & ON9 V         & W76 & 7.66 & 2 & 3 & 2000 & 3000 & 250 \\
202124  &               & O9.5 Iab      & W76 & 7.80 & 2 & 2 & 1600 & 1000 & 400 \\
207198  &               & O9 Ib-II      & W76 & 5.96 & 3 & 3 & 1300 &  600 & 350 \\
209975  & 19 Cep	& O9.5 Ib       & W76 & 5.11 & 2 & 3 &  300 &  300 & 350 \\
210809  &               & O9 Iab        & W76 & 7.55 & 3 & 2 & 2400 & 1000 & 400 \\
214680  & 10 Lac	& O9 V          & W73 & 4.87 & 3 & 3 &  150 &  150 & 350 \\
218195  &               & O9 III        & W76 & 8.34 & 2 & 2 & 1600 & 1400 & 300 \\
218915  &               & O9.5 Iab      & W76 & 7.18 & 3 & 3 & 2600 & 3000 & 400 \\
225160  &               & O8 Ib(f)      & W71 & 8.19 & 2 & 2 & 2400 & 2000 & 300 \\
+36 4063&               & ON9.7 I       & M89 & 9.71 & 5 & 2 & 6900 & 3000 & 150 \\
\hline\hline
\end{tabular}
\label{tab:obs_targets}
\end{center}
\tnote{Columns three and four list classifications and published
reference. Column 5 gives the V magnitude listed in Garmany's
catalogue (see text). The wavelengths for columns 6 to 9 represent the
central wavelength setting of the spectrograph (see
text). Signal-to-noise (column 10) is calculated for a small spectral
region between 4200\AA\ and 4300\AA\ and is not necessarily
representative for the entire wavelength range.}
\tref{W71, W73, W76 -- Walborn \cite*{wa:71supp,wa:73,wa:76}
respectively, M89 -- \cite{ma:89}.}
\end{table}
%%%%%%%%%%%%%%%%%%%%%%%%%%%%%%%%%%%%%%%%%%%%%%%%%%%%%%%%%%%%%%%%%%%%%%%%%%%%%

\subsection{The Utrecht Echelle Spectrograph}

The UES is a cross-dispersed echelle, used in our observations with
the 31.6 grooves/mm echelle grating.  The slit was opened to 1.5$''$,
which gave a resolution element of about 65 microns at the detector
(approximately 3 pixels). This corresponds to a resolving power $R =
\lambda /\Delta\lambda = 40000$ (about 8~\kms). The detector was the 
Tektronix 5 (TEK5) with $1124\times1124$ pixels ($1024\times1024$
usable imaging size).

Several exposures with two spectrograph settings were obtained for all
the target stars. Centering the grating at \lcen\ = 4190\AA\ in order
136 and \lcen\ = 5426\AA\ in order 105 gave complete spectral coverage
\label{spectral_coverage} from roughly $\lambda3780$\AA\ to
$\lambda5670$\AA, and further coverage with inter-order gaps up to
$\lambda6980$\AA. These wavelengths are in the rest-frame of the
observer, and so the actually stellar wavelengths depend slightly on the 
radial velocity of the stars observed and on heliocentric corrections.


\subsection{Data reduction}

The echelle data reduction was performed using the Starlink packages
\echomop\ (\cite{sun:152}), \figaro\ (\cite{figaro}), and
\dipso\ (\cite{dipso}). \echomop\ was originally developed to extract data
from the UES' ``twin'' instrument, the UCL Echelle Spectrograph
(UCLES) at the Anglo Australian Telescope (AAT). The 2-D images were
first bias subtracted and cleaned of cosmic rays using the Starlink
software \prog{bclean}. The data were then transfered to the \echomop\
environment. \echomop\ is a multi-step reduction package that can
either be run interactively or automated through shell scripts. For
this project, \echomop\ was used interactively on the first
exposure of each spectrograph setting of each star; the resulting
`reduction data structure' was then used to automatically reduce the
remaining spectra at the same settings. The data structure contains
information about the extracted spectra, such as the order positions and
shapes, the cross-dispersion profile and so on. It did, however,
sometimes turn out to be necessary to reduce each exposure manually,
usually because of slight changes to telescope positioning, or
instrument setup.

There is a good summary of the general properties of echelle
spectra in \emph{Starlink Guide 9} (\cite{sg:9}). Below is a brief
description of the steps involved in the reduction and extraction of
these data.

\subsubsection{Order identification and tracing}
Before the stellar spectra can be extracted, the orders have to be
identified. \echomop\ proved very good at locating the orders in the
stellar frames, and no manual intervention was necessary. The orders
were then traced using a centroid tracing routine, and polynomials of
the 5th degree were fitted to the orders. For each order, the
differences between the centroid trace and the fitted polynomial were
plotted, and points that fell more than 0.3 pixels away from the
polynomial were then removed and the order re-fitted. This was
iterated until no points fell outside the 0.3 pixel limit.

\subsubsection{Sky \& background fitting}
An averaged flat-field frame was then used to determine the dekker
limits, which mark the extent of each order in the spatial
(cross-dispersed) direction. When the dekker limits had been
determined, the orders were examined manually and each pixel along the
slit was marked as either belonging to the star or sky. The mean (in
the spatial direction) of the ``sky'' pixels was used to correct for
sky emission/scattered light. From measuring the FWHM of the arc lines
we can estimate the spectrograph resolution, $R = \lam /\Delta\lam$
which we find to be about 40000 -- 50000 dependent on the wavelength.

\subsubsection{Spectrum extraction}
Finally, the object profile was calculated using the star mask created
above, and the stellar and arc spectra were extracted using an optimal
extraction routine. This routine weights each pixel by the product of
the calculated profile, $P(i,j)$, and an estimate of the uncertainty
of the pixel intensity (see next section). The star and arc spectra
were extracted simultaneously to ensure that the same weights were
used in both cases. Arc lines were identified using \prog{echarc} and
the wavelength information mapped on to the extracted spectra.

\subsection{Flat-fielding}
\label{sec:flatfield}
Flat-fielding of echelle spectra is far more complicated than for
normal grating spectra. Although the flat-field frames should be quite
sensitive to pixel-to-pixel variations in the detector, it is far more
difficult to model the variations caused by the instrumental profile
(see section \ref{sec:blaze}). Since both of these are present in the
flat-field frames, it is difficult to extract any useful information
from them.

Several tests were performed using flat-field frames obtained in the
afternoon before each night's observing. Cross-dispersion `drift' in
the spectrograph (perhaps due to temperature changes in the
spectrograph\footnote{The refractive index of the cross-dispersion
prism is temperature sensitive, and although the UES is in a
temperature controlled area, there does appear to be a drift in the
cross-dispersion direction throughout the nights of observing.}) meant
that a straight division of the flat-field frames into the data frames
degraded rather than improved the quality of the extracted spectra. We
also tried to extract the flat-field frames using the same object
profile obtained during the spectrum extraction and several different
\emph{weighting schemes}: simple, profile and optimal.
\begin{itemize}
\item	\textbf{Simple:} all object pixels are assigned equal weights.
\item	\textbf{Profile:} Weights each pixel by $P(i,j)^2$ where $P(i,j)$ is
	the calculated normalisation profile at spatial offset $j$ from the
	trace centre, and $i$ is the column number.
\item 	\textbf{Optimal:} Each pixel is weighted by the product of $P(i,j)$
	and an estimate of the uncertainty of the pixel intensity.
\end{itemize}
However, none of these methods seemed to be able to correct
satisfactorily for pixel defects in the detector or the echelle
blaze. No flat-fielding was carried out, therefore. This was
acceptable from the point of view of pixel-to-pixel sensitivity
differences, which were small, but left the problem of correcting the
echelle blaze.

\subsection{Echelle blaze correction}
\label{sec:blaze}
The shape of the extracted spectra is dominated by the so-called
echelle blaze function. This is just the angle (i.e. wavelength)
dependence of the light reflected by the grating. There are several
ways of correcting for the blaze, but none of them are
straight-forward. One of the most common ways is to divide through by
the normalised flat-field, but, as noted above, this method did not
work well for these data -- possibly due to difference in the optical
paths in the spectrograph between the flat-fields and the actual
observations (see also section
\ref{sec:flatfield}). However, several other methods of correcting for
the echelle blaze were tried. In particular, two different routines
were written in
\prog{fortran 77} and incorporated into the \dipso\ environment.

\subsubsection{Barker method}
\label{sec:barker}
The method of \ybcite{ba:blaze} involves dividing the spectral
orders by a theoretical ripple function, $R(\lambda)$. For a plane
grating used in near-Littrow mode (basically when the angle of
incidence is not very different from the facet normal; see
\cite{gray}), the blaze function is given by
\begin{equation}
R(\lambda) = \frac{\sin^2 \pi\alpha X}{(\pi\alpha X)^2}
\end{equation}
where $X = m[1-\lambda_c(m)/\lambda]$. The order number, $m$, and the
central wavelength of order $m$, $\lambda_c(m)$, are related through
the grating constant, $k = m\lambda_c(m)$, and $\alpha$ is another grating
constant which depends on the facet width, the spatial frequency of
facets, and the blaze angle (\cite{ba:blaze}).

The above equation for the blaze function fits well, but not well
enough to remove all the ripple. The Barker routine finds a correction
to the grating constant, $k$, for each order -- allowing $k$ to vary
as a function of the order, $m$. The correction is found by examining
the overlap region between two orders after they have been de-blazed
with an initial estimate of the value of $k$ (usually
$m\lambda_c(m)$). For each overlap region, the orders should match
up. The fluxes, $F_+$ and $F_-$, of the de-blazed orders, are then
measured in the overlap region. $F_+$ and $F_-$ just denotes the flux
of the order with the higher flux and lower flux respectively (see fig
\ref{fig:barker}).

A differential correction, $\Delta k$, to the grating constant is then calculated
for each order from
\begin{equation}
\Delta k_m = \frac{\frac{F_+}{F_-} - 1}
{
\frac{1}{R_m}\frac{\partial R_m}{\partial k}
- \frac{F_+}{F_-}\frac{1}{R_{m+1}}
\frac{\partial R_{m+1}}{\partial k}
}
\end{equation} 
The process is then repeated until $\Delta k$ is sufficiently small.

The Barker routine works well when the correct value for $\alpha$ has
been found, but the major problem is the need for order overlap. As
mentioned in section \ref{spectral_coverage}, redwards of
$\lambda$5670\AA\ there is only partial coverage of the full spectral
range. It is possible that a function could be fitted to the corrected
values of $k$ and extrapolated into the region without overlap. This
has not been tried for this project, but \ybcite{ho:winds} smoothed
the $k$ values and introduced a 2$\sigma$ rejection. They found that
this greatly reduced the chance of erroneous values of $k$ caused by
cosmic-rays and phosphor decay spikes.

%%%%%%%%%%%%%%%%%%%%%
\begin{figure}
\epsfxsize=\figwidth
\setlength{\cen}{(\textwidth / 2) - (\epsfxsize / 2)}
\hspace{\cen}\epsfbox{fig_obs/barker.eps}
\caption[Schematic describing blaze-removal method of
Barker]{\fcfont Schematic of three adjacent orders in an echelle spectrum with
incorrect blaze removal. From Barker, 1984.}
\label{fig:barker}
\end{figure}
%%%%%%%%%%%%%%%%%%%%%%

\subsubsection{Splif}
A very different approach was taken in a previously empirical method
by Howarth (private communication), called \prog{splif}. The extracted
echelle orders are treated as a surface in the spectral vs. spatial
(order number) plane. This surface is then fitted using a very simple
approach. Briefly, this involves:

\begin{enumerate}

\item 	Each order is ``normalised'' by dividing by the mean intensity
of that order. This step removes the large order to order variations in
intensity and allows the use of a simple fit surface. This step can
also be slightly improved by fitting a polynomial to a plot of mean
intensity against order and dividing through by the fitted values,
since orders with strong lines, or many weak lines, will lower the
mean intensity of that order.

\item 	For each pixel in the dispersed direction, a straight line is
fitted to a plot of intensity as a function of order number. This
returns (for the TEK5) two arrays of 1024 elements, one holding the
gradients of each fitted line, and the other the intercepts.

\item	The two arrays are then optionally fitted with polynomials to create a
smoothed surface.

\item	The spectral data from step (1) are divided by this surface.

\item	A linear `ramp' is applied to each order to ensure agreement
in regions of order overlap.

\item	Orders are merged with a ramped weighting function.

\end{enumerate}

\begin{figure} %%%%%%%% FIGURE

\setlength{\cen}{\figwidth/2}

\begin{tabular}{cc}
\epsfxsize=\cen\epsfbox{fig_obs/splif_3D/orders_raw.eps}&
\epsfxsize=\cen\epsfbox{fig_obs/splif_3D/orders_nor.eps}\\
\epsfxsize=\cen\epsfbox{fig_obs/splif_3D/orders_splif.eps}&
\epsfxsize=\cen\epsfbox{fig_obs/splif_3D/splif_spectra.eps}\\
\end{tabular}

\caption[Splifing]
{\fcfont The four steps of \prog{splif}ing: \textbf{top left} --- raw
extracted orders; \textbf{top right} --- `normalised' orders;
\textbf{bottom left} --- \prog{splif}ed orders; \textbf{bottom right}
--- almost done, spectral orders just need to be normalised before merging.}
\end{figure} %%%%%%%

Because of the several layers of smoothing, the method is fairly
robust against spectral features, and does not require order overlap.
As shown in figure \ref{fig:splif_wide}, even wide spectral features which
span several orders are preserved in the fitting process.

\begin{figure} %%%%%%%%%% FIGURE
\epsfxsize=\figwidth
\setlength{\cen}{(\textwidth / 2) - (\epsfxsize / 2)}
\hspace{\cen}\epsfbox{fig_obs/splif_dib/hd218195_dib.eps}

\epsfxsize=\figwidth
\setlength{\cen}{(\textwidth / 2) - (\epsfxsize / 2)}
\hspace{\cen}\epsfbox{fig_obs/splif_ha/hd195592_ha.eps}

\caption[Wide spectral features preserved by SPLIFing]{\fcfont The SPLIF method
preserves wide spectral features that span several echelle orders. The
\textbf{top panel} shows the strong diffuse interstellar band at \lam
4430\AA\ in HD~218195.  The merged spectrum is shown with the
individual orders offset underneath. The spectra have been clipped at
a rectified intensity of 0.85 for clarity.  The \textbf{bottom panel}
shows H$\alpha$ emission in HD~195592. In this wavelength region there
is a large inter order gap. The strong emission meant that we had to
saturate H$\alpha$ to get decent S/N for the rest of the
spectrum. Scattered light from this strong emission might have caused
the small features seen at approximately \lam\lam 6480 and 6640\AA\ in
the adjacent orders.}
\label{fig:splif_wide}
\end{figure}
%%%%%%%%%%%%

All the UES data were de-blazed using \prog{splif} and all the orders
containing lines used in the analysis were carefully inspected
before they were merged to form a single, continuous
spectrum. Neighbouring orders were usually in good agreement in the
overlap region, but in a few cases there were differences of the order
of a few per cent of the continuum intensity. The orders were merged
using a gradient function in the overlap region. This would slope
linearly from 1 to 0 and give less weight to the end points in each
order.

The \prog{splif}ed, merged and normalised WHT spectra are included on
the attached CD-ROM. See appendix \ref{app:cdrom} for details.

\section{\iue\ data}

The second part of this thesis (chapter \ref{ch:xcorr}) investigates
photospheric line broadening in UV spectra of O and B stars. The O
(and some B) stars were initially analysed as part of an undergraduate project
undertaken before starting this Ph.D. research, and subsequently
published in \ybcite{ho:xcorr}. We have subsequently automated the analysis
procedures and have expanded the sample to include later-type B
supergiants than those investigated in \acite{ho:xcorr}, and also B
main-sequence stars. 

As before, UV spectra taken with the
\emph{International Ultraviolet Explorer} (\iue)
satellite's Short Wavelength Prime (SWP) camera were used for the
analysis. \iue, launched in January 1978 by a collaboration of SERC,
ESA and NASA, had four UV sensitive cameras. The Long Wavelength Prime
and Redundant (LWP and LWR) cameras observed in the range
1900-3200\AA, and the Short Wavelength Prime camera in the range
1150-2000\AA\ (the Short Wavelength Redundant camera had a faulty
readout section and was not used). 

Each of the two spectrographs on board (LW \& SW) could operate in
high or low-resolution mode. The low-resolution spherical gratings
produced spectra with a resolution of about 6\AA, and the
high-resolution echelle spectrographs produced spectra with a
resolution of approximately 0.1\AA. Only high-resolution spectra have
been used in this study.

\subsection{The \iue\ sample}

Using the electronic version of the Bright Star Catalogue
(\cite{bsc5}), all stars classified as type B were extracted and
cross-checked with the IUE catalogue. Where appropriate observations
existed, these were downloaded from the IUE archives available at
\texttt{http://ast.star.rl.ac.uk/iues/iues.html}. Some additional stars from
\ybcite{ho:xcorr} were included to test the automatic analysis
routines. The sample of stars is discussed in more detail in chapter
\ref{ch:xcorr}.

\subsection{Data extraction}

The \iue\ data were extracted in a very similar manner to that
described in \ybcite{ho:winds} using the Starlink \iue\ data reduction
package \prog{iuedr} (\cite{iuedr}). A good review of \iue\ data
reduction and analysis can be found in \emph{IUE Analysis --- A
tutorial} (\cite{sg:7}). Briefly, the steps involved include:

\begin{itemize}
\item	Using the commands \comm{scan}\ and \comm{agshift}\ successively
	to find the geometric shift used to trace the orders.

\item	Performing centroid tracing and order extraction using
	\comm{trak}.

\item	Applying a wavelength correction of the form $m \Delta\lambda =
	\mathrm{constant}$ to bring interstellar lines in the spectrum
	to their laboratory wavelengths.

\item 	Correcting for Echelle ripple using the method of Barker (see
	section \ref{sec:barker}).

\item	Subtracting inter-order background (approximately 12\% at
	Ly$\alpha$ for the B stars and 15\% for the O stars).

\item	Merging and mapping the orders onto a uniform wavelength grid.
\end{itemize}

The reduction process was greatly simplified by using modified
procedures created by I. D. Howarth and M. Clayton (private
communications) with some new additions, mainly to try to
automatically detect if the archive images are both geometrically and
photometrically corrected (GPHOT) or just photometrically\footnote{For
the PI images, the photometric calibration frames were corrected to
match the image geometry, not the other way around as for the GPHOTs}
corrected (PHOT/PI). These corrections were necessary to transform the
faceplate charge of the detectors to linear intensity units. Around
the end of 1980, the correction type changed from GPHOT to the
superior PI. Unfortunately the exact date of change depends on which
ground station processed the images. Thus, for stars where there is
doubt about the correction type, the spectra were extracted twice,
trying both corrections (\prog{iuedr} needs to know the correction
type to set default extraction parameters). In most cases it is very
obvious when the wrong type has been used, as seen in figure
\ref{fig:iue:red}. The spectra are tested by calculating the pixel
intensity range over small areas of the spectrum around order overlap
regions. The correction type that gives the smallest spread of
intensities (i.e. less noisy) is assumed to be the correct one. This
seemed to work well and only failed where there were problems with the
data reduction (in some cases the wrong aperture was listed in the
logs). Finally, all the spectra were visually examined to check for
flaws.

\begin{figure} %%%%%%%%%FIGURE
\epsfxsize=\figwidth
\setlength{\cen}{(\textwidth / 2) - (\epsfxsize / 2)}
\hspace{\cen}\epsfbox{fig_obs/autoiue/autoiue_swp5652.eps}
\caption[Example of IUE spectrum extracted assuming the wrong
correction type] {\fcfont Example of IUE spectrum extracted assuming
the wrong correction type. This spectrum is clearly of the older GPHOT
type.}
\label{fig:iue:red}
\end{figure}   %%%%%%%%%%%%%%%



